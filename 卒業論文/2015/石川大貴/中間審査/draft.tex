%中間審査概要テンプレート ver. 3.0

\documentclass[uplatex,twocolumn,dvipdfmx]{jsarticle}
\usepackage[top=22mm,bottom=22mm,left=22mm,right=22mm]{geometry}
\setlength{\columnsep}{10mm}
\usepackage[T1]{fontenc}
\usepackage{txfonts}
\usepackage[expert,deluxe]{otf}
\usepackage[dvipdfmx,hiresbb]{graphicx}
\usepackage[dvipdfmx]{hyperref}
\usepackage{pxjahyper}
\usepackage{secdot}





%タイトルと学生番号,名前だけ編集すること
\title{\vspace{-5mm}\fontsize{14pt}{0pt}\selectfont GitHub APIを活用するプロジェクトマネジメントツールの提案と実装 }
\author{\normalsize プロジェクトマネジメントコース 矢吹研究室 1342011 石川大貴}
\date{}
\pagestyle{empty}
\begin{document}
\fontsize{10.5pt}{\baselineskip}\selectfont
\maketitle





%以下が本文
\section{背景}

ソフトウェア開発の現場では,主にウォーターフォール型開発が採用されていたが,現在ではアジャイル型開発が普及してきている.ウォーターフォール型開発は仕様を最初にすべて決めてから機能を実装するため,開発着手までに時間がかかる.さらに,テストで不具合が発生すると,後半になるほど手戻りの工数が大きくなってしまうため,途中での仕様変更は困難となる.その一方で,アジャイル型開発は開発対象を多数の小さな機能に分割し,短い期間で実装とテストを繰り返して徐々に開発を進めていくので,ウォーターフォール型開発に比べて開発途中の仕様変更が容易である.

アジャイル型開発では,テスト駆動開発がよく採用される.これは,プログラムに必要な各機能について最初にテストを書き,そのテストが動作する必要最低限な実装を行った後,コードを洗練させるという短い工程を繰り返し行う手法である.\cite{shimizu2012}

テストを実施するにあたって,カバレッジを測定・分析することがソフトウェアの品質向上に大きく関わる.カバレッジとはテスト対象となる部分のうち,テストした部分がどれだけ占めているかの割合である.カバレッジを測定する方法は,コードや仕様,要件,設計など,さまざまな側面から計測する方法があるが,単体テストの段階では,コードベースのカバレッジでテストの品質を測ることが一般的である.コードカバレッジを測定し,テストが実施されていないコードを確認することにより,テストのミスや不具合を検出しやすくなる.\cite{watanabe2014}

よく利用されているホスティングサービスの一つにGitHubがある.GitHubではプロジェクトのバグ管理に使えるIssuesや,コードレビューを効率化するPull Requestなどの開発に役立つ機能が多くある.さらに,連携が可能な開発ツールやサービスも多くある.それらを活用してテスト工程を管理できれば,より円滑にプロジェクトを進めることが可能であると考える.

\section{目的}

本研究では,ソフトウェア開発のテスト工程に着目し,GitHub上のプロジェクトを調査する.そこからテスト工程で使えるGitHubを活用したマネジメントツールを提案する.

\section{手法}

GitHubで利用できる機能やAPI,連携できるツールを調査する.また,GitHubにホスティングされているプロジェクトを調査し,現在のソフトウェア開発のテストについての実情を知る.調査の例として,テストコードのカバレッジを測定して,テストがどれほど網羅されているか知る.その上で,より管理しやすくなるようなツールを作成する.

\section{想定される成果物}

GitHubを活用したマネジメントツール

\section{進捗状況}

現在,GitHubの機能や使い方を調査するとともに,テストコードがあるソフトウェアを調査し,GitHubを活用したマネジメントツールを考案中である.

\section{今後の計画}

今後は以下のように研究を行う.
\begin{enumerate}
\item GitHubを活用したマネジメントツールを作成する.
\item 作成したツールがテスト工程で活用できるか実際に使用し検討する.
\end{enumerate}

\bibliographystyle{junsrt}
\bibliography{biblio}%「biblio.bib」というファイルが必要.

\end{document}
