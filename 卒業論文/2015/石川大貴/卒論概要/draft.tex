\documentclass[uplatex,twocolumn,dvipdfmx]{jsarticle}
\usepackage[top=22mm,bottom=22mm,left=22mm,right=22mm]{geometry}
\setlength{\columnsep}{10mm}
\usepackage[T1]{fontenc}
\usepackage{txfonts}
\usepackage[expert,deluxe]{otf}
\usepackage[dvipdfmx,hiresbb]{graphicx}
\usepackage[dvipdfmx]{hyperref}
\usepackage{pxjahyper}
\usepackage{secdot}





\title{\vspace{-5mm}\fontsize{14pt}{0pt}\selectfont OSS開発プロジェクトにおけるメンバの貢献度調査}
\author{\normalsize プロジェクトマネジメントコース 矢吹研究室 1342011 石川大貴}
\date{}
\pagestyle{empty}
\begin{document}
\fontsize{10.5pt}{\baselineskip}\selectfont
\maketitle





\section{序論}

ソフトウェア開発の現場では,主にウォーターフォール型開発が採用されていたが,現在ではアジャイル型開発が普及してきている.アジャイル型開発では,テスト駆動開発がよく採用される.これは,プログラムに必要な各機能について最初にテストを書き,そのテストが動作する必要最低限な実装を行った後,コードを洗練させるという短い工程を繰り返し行う手法である\cite{shimizu2012}.

開発プロセスに関する情報を誰でも見ることができる,オープンソースソフトウェア(OSS)開発というものがある.OSS開発には,OSSホスティングサービスを利用して開発されることが多い.最もよく利用されているOSSホスティングサービスの一つがGitHubである.実際,1000万以上のユーザがGitHubを活用しており,リポジトリ数は2430万以上もある\cite{github}.近年のOSS開発について調査するにあたり,多くのプロジェクトをホストするGitHubが適切だと考える.


\section{目的}

本研究ではGitHub上のプロジェクトを調査する.テスト駆動開発において,個人の貢献度がメインコードとテストコードで違いがあるのかを調査する.


\section{手法}
以下の手順で研究を進める.
\begin{enumerate}
\item GitHubのプロジェクトでテストコードがあるプロジェクトを探す.
\item プロジェクトのコミットごとにユーザ名,日時,メインコードの行数,テストコードの行数をそれぞれ取得できるプログラムを作成し,実行する.
\item 取得したデータをメンバごとにソートし,メインコードとテストコードの行数を集計する.
\item メンバごとに行数を比較し,プロジェクトへの貢献度の違いを見る.
\end{enumerate}


\section{結果}

22件のプロジェクトでメンバごとの行数を比較した.パレートの法則が成り立つ可能性が考えられたため,さらに調査を行った.パレートの法則が成り立つプロジェクトはメインコードでは22件中13件であり,テストコードでは22件中7件であった.また,メインコードとテストコードの両方で貢献度の高いメンバは1から3人ほどであり,その他はどちらかで貢献していることが多かった.このように,メインコードとテストコードで,メンバごとのプロジェクトへの貢献度の違いを見ることができた.


\section{考察}

OSS開発でもパレートの法則は成り立つ.しかし,同じプロジェクトでもメインコードよりテストコードの方が成り立つ可能性が少ない.メインコードもテストコードも貢献度が高いメンバの人数はあまり変わらないのに対し,メインコードよりテストコードの方が,参加メンバが少ないためだと考える.また,少数のメンバがメインコードとテストコードの両方で貢献度が高く,そのメンバが中心的に活動していると考える.


\section{結論}

メンバによる貢献度の違いが明らかになったため,貢献度の高いメンバの開発手法を参考にすることで,より効率的な開発が可能になるだろう.全体的にテストコードを書くメンバが少ないため,テストコードを書くことでプロジェクトへの貢献度は高まるはずだ.


\bibliographystyle{junsrt}
\bibliography{biblio}%「biblio.bib」というファイルが必要.

\end{document}
