\documentclass[uplatex,twocolumn,dvipdfmx]{jsarticle}
\usepackage[top=22mm,bottom=22mm,left=22mm,right=22mm]{geometry}
\setlength{\columnsep}{10mm}
\usepackage[T1]{fontenc}
\usepackage{txfonts}
\usepackage[expert,deluxe]{otf}
\usepackage[dvipdfmx,hiresbb]{graphicx}
\usepackage[dvipdfmx]{hyperref}
\usepackage{pxjahyper}
\usepackage{secdot}





\title{\vspace{-5mm}\fontsize{14pt}{0pt}\selectfont OSS開発プロジェクトにおけるメンバの貢献度調査}
\author{\normalsize プロジェクトマネジメントコース 矢吹研究室 1342011 石川大貴}
\date{}
\pagestyle{empty}
\begin{document}
\fontsize{10.5pt}{\baselineskip}\selectfont
\maketitle





\section{序論}

ソフトウェア開発の現場では,主にウォーターフォール型開発が採用されていたが,現在ではアジャイル型開発が普及してきている.アジャイル型開発では,テスト駆動開発がよく採用される.これは,プログラムに必要な各機能について最初にテストを書き,そのテストが動作する必要最低限な実装を行った後,コードを洗練させるという短い工程を繰り返し行う手法である\cite{shimizu2012}.

開発プロセスに関する情報を誰でも見ることができる,オープンソースソフトウェア(OSS)開発というものがある.OSS開発には,OSSホスティングサービスを利用して開発されることが多い.最もよく利用されているOSSホスティングサービスの一つがGitHubである.実際,1000万以上のユーザがGitHubを活用しており,リポジトリ数は2430万以上もある\cite{github}.近年のOSS開発について調査するにあたり,多くのプロジェクトをホストするGitHubが適切だと考える.

本研究でGitHub上のプロジェクトを調査し,テスト駆動開発において個人がプロジェクトに与える影響を調査したい.今回はメンバが追加したコードの行数を貢献度とし,メインコードとテストコードで違いがあるのか調査する.パレートの法則が成り立つ可能性が考えられる.パレートの法則とは,全体の数値の大部分は,全体を構成するうちの一部の要素が生み出しているという理論であり,80:20の法則とも呼ばれる.


\section{目的}

本研究ではGitHub上のプロジェクトにおいて,パレートの法則が成り立つか個人の貢献度をメインコードとテストコードで調査する.


\section{手法}

まず,GitHubのプロジェクトでテストコードがあるプロジェクトを探す.プロジェクトのコミットごとにユーザ名,日時,コードの行数をそれぞれ取得できるプログラムを作成し,実行する.取得したデータをメンバごとにソートし,メインコードとテストコードの行数を集計する.メンバごとに行数を比較し,パレートの法則が成り立つか調査する.


\section{結果}

パレートの法則が成り立つプロジェクトはメインコードでは22件中13件であり,テストコードでは22件中7件であった.また,メインコードとテストコードの両方で貢献度の高いメンバは1から3人ほどであり,その他はどちらかで貢献していることが多かった.このように,メインコードとテストコードで,メンバごとのプロジェクトへの貢献度の違いを見ることができた.


\section{考察}

OSS開発でもパレートの法則は成り立つ.しかし,同じプロジェクトでもメインコードよりテストコードの方が成り立つ可能性が少ない.メインコードもテストコードも貢献度が高いメンバの人数はあまり変わらないのに対し,メインコードよりテストコードの方が,参加メンバが少ないためだと考える.また,少数のメンバがメインコードとテストコードの両方で貢献度が高く,そのメンバが中心的に活動していると考える.


\section{結論}

メンバによる貢献度の違いが明らかになったため,貢献度の高いメンバの開発手法を参考にすることで,より効率的な開発が可能になるだろう.全体的にテストコードを書くメンバが少ないため,テストコードを書くことでプロジェクトへの貢献度は高まるはずだ.


\bibliographystyle{junsrt}
\bibliography{biblio}%「biblio.bib」というファイルが必要.

\end{document}
