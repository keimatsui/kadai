%卒論概要テンプレート ver. 3.0

\documentclass[uplatex,twocolumn,dvipdfmx]{jsarticle}
\usepackage[top=22mm,bottom=22mm,left=22mm,right=22mm]{geometry}
\setlength{\columnsep}{10mm}
\usepackage[T1]{fontenc}
\usepackage{txfonts}
\usepackage[expert,deluxe]{otf}
\usepackage[dvipdfmx,hiresbb]{graphicx}
\usepackage[dvipdfmx]{hyperref}
\usepackage{pxjahyper}
\usepackage{secdot}





%タイトルと学生番号,名前だけ編集すること
\title{\vspace{-5mm}\fontsize{14pt}{0pt}\selectfont 誰がSNSを炎上させるのか}
\author{\normalsize プロジェクトマネジメントコース 矢吹研究室 1342069 氏名 下村渉}
\date{}
\pagestyle{empty}
\begin{document}
\fontsize{10.5pt}{\baselineskip}\selectfont
\maketitle





%以下が本文
\section{序論}
ネットサービスが急速な発展をし,ソーシャルメディアを利用すれば,不特定多数の相手に対して,個人が情報を発信することも容易になっている.しかし,ネット上でのコミュニケーションが活発になる一方で,ある人物が発言した内容や行った行為について,ソーシャルメディアに批判的なコメントが殺到する現象「ネット炎上」が多発するようになってきている\cite{a}.炎上が日本で認識されるようになったのは2004年に無料ブログやSNSが多くサービスを開始したこととされ,2011年頃から顕著に増加している\cite{b}.

 短文の投稿を共有するウェブ上の情報サービスであるTwitterでは日常的に悪質なツイートがされる.例えば一般ユーザーがファッションセンターしまむらの店員に対し土下座を強要し,その光景を写真に収めツイートした件やコンビニエンスストアであるローソンの従業員がアイスケースの中に入った写真をツイートした件などがある.

日常的に悪質なツイートがされることに対し,悪ふざけや犯罪を自慢するツイート,情報モラル,情報リテラシーが低いツイートを見過ごさず,通報やリツイートをする正義感溢れる人達がいる.彼らはそれ相応の罰を受ける必要があるという正義感から通報やリツイートをする.リツイート数が伸びると便乗してリツイートするユーザが増え,結果事態が大きくなってしまい炎上してしまう場合がある.
 炎上しないための対策はいくつかある.例えば,発言は誰でも見ることができることを意識する,プライベートなアカウントでも用心する,発言の前に友人や知人に確認するなどの方法がある .そのため,多くの炎上をリツイートしているユーザの特定をすることは炎上するリスク対策につながると考えた.



\noindent




\section{目的}
本研究の目的はTwitter上で悪ふざけの投稿や犯罪自慢投稿,情報モラル,リテラシーの低いツイートをより多くリツイートしているユーザの特定し,炎上のリスク対策ができるような指標を作成する.
\section{手法}
本研究はTwitterとStreaming APIを使用しデータの集計をする.Streaming APIとはツイートされた内容をほぼリアルタイムで自動で取得するものである.Streamingデータから,リツイートだけを取り出し,データベースに格納する.リツイートした人,された人のフォロー関係を調べ,データベースに記録する.


\section{結果}
炎上ツイートの定義をリツイート数が5000以上されたものとし,ランダムサンプリングしたツイートの中からリツイート数が5000以上ものを100件取得した.その100件のツイートをリツイートしているユーザーを取得した結果,2件以上リツイートしたユーザーは366001人中146人で,ツイートした人をフォローしている人は245人だった.

\section{考察}
5000以上リツイートされたものを炎上としたとき,炎上ツイートした人をフォローしている人が炎上を拡散することは少なく,フォローをしていない人の方が炎上を拡散していると考えられる.

\section{結論}
本研究では,リツイートした人とリツイートされた人の関係性を調べることができた.炎上したツイートは消されることが多くデータを収集するのが艱難なので,どのようにデータを収集するのかが今後の課題である.


\bibliographystyle{junsrt}
\bibliography{biblio}%「biblio.bib」というファイルが必要.

\end{document}
