%卒論概要テンプレート ver. 3.0

\documentclass[uplatex,twocolumn,dvipdfmx]{jsarticle}
\usepackage[top=22mm,bottom=22mm,left=22mm,right=22mm]{geometry}
\setlength{\columnsep}{10mm}
\usepackage[T1]{fontenc}
\usepackage{txfonts}
\usepackage[expert,deluxe]{otf}
\usepackage[dvipdfmx,hiresbb]{graphicx}
\usepackage[dvipdfmx]{hyperref}
\usepackage{pxjahyper}
\usepackage{secdot}





%タイトルと学生番号,名前だけ編集すること
\title{\vspace{-5mm}\fontsize{14pt}{0pt}\selectfont ソーシャルブックマーキングサービスのデータ分析によるウェブマーケティング}
\author{\normalsize プロジェクトマネジメントコース 矢吹研究室 1342029  遠藤一輝}
\date{}
\pagestyle{empty}
\begin{document}
\fontsize{10.5pt}{\baselineskip}\selectfont
\maketitle





%以下が本文
\section{序論}
私たちの周りには様々な情報やニュースがあふれている.情報の入手や発信を行う際,利用する主なものとしてインターネットがあげられる.インターネットでは手軽に情報の入手及び発信が行えるが,インターネット上にある情報は膨大なため,情報の取捨選択が求められる.そのため多くの人に自身の発信する情報を効果的に伝えることが重要となる.
そこで多くの人に閲覧してもらえるための効果的なウェブマーケティングの分析を行う.
本研究ではウェブマーケティングの例としてソーシャルブックマーキングサービスを用いる.
ソーシャルブックマーキングサービスを用いるのは,利用者が様々なジャンルのサイトを登録するため幅広いジャンルのデータが収集でき,ブックマーク数という共通の数値が得られるため比較が容易となるからである.

\section{目的}
ソーシャルブックマーキングサービスにおけるサイトのブックマーク数の推移を調査し分析することで,ブックマーク数の増加傾向や伸びやすい時間帯の考察を行う.
それをもとにウェブマーケティングにおける指標の一つを作成することを本研究の目的とする.


\section{手法}

\subsection{データの収集方法}
ソーシャルブックマーキングサービスの一つであるはてなブックマークを対象に,はてなブックマークが公開しているのAPI\cite{hatena}を使用し,人気記事の情報を取得する.
APIで取得した情報からブックマークの詳細,記事のジャンル,記事の掲載日時等の要素を抜き出し,グラフを作成する.

\subsection{分析}
1日のブックマークの増加数を1時間ごとに分け,ブックマークされやすい時間帯を調べる.
作成したグラフの推移に関してパターン分けを行い,各パターンについての考察を行う.

\section{結果}
分析の結果,ブックマーク数の伸びやすい時間帯は8時から12時までであることが分かった.特に12時が最もブックマーク数の伸びやすい時間であった.逆にブックマーク数の伸びにくい時間帯は2時から6時までであり,特に4時が最もブックマーク数が伸びにくいという結果となった.
作成したグラフを分類した結果,大きく分けて以下の3つのパターンに分類した.
\begin{enumerate}
\item 平均型…安定して伸び続け,ほぼ直線のグラフを描く.
\item 急上昇型…初期はあまり数値が伸びないものの,ある一点から数値が伸び始める.
\item 不規則型…上昇と停滞を繰り返し波のような曲線を描く. 
\end{enumerate}


\section{考察}
深夜時はブックマーク数が伸びづらいためグラフの曲線が停滞しやすいと考えられる.そのため短期でブックマーク数が伸びきった場合は平均型となり,短期でブックマーク数が伸びたが深夜を挟むか深夜にブックマークが伸び始めた場合は急上昇型となりやすい.長期にわたってブックマーク数が増えている場合は,停滞しやすい深夜を数回挟むためグラフの曲線が不安定になりやすく,結果として不規則型となりやすい.

\section{結論}
本研究では時間とブックマーク数の推移に着目し分析を行った.この要素のみの分析ではまだ確実なウェブマーケティングの予測は難しいが,取得できる他の要素についても分析を行うことでより確実なウェブマーケティングの予測が可能となる.


\bibliographystyle{junsrt}
\bibliography{biblio}%「biblio.bib」というファイルが必要.

\end{document}
