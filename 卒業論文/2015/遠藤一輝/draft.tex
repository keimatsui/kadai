%中間審査概要テンプレート ver. 3.0

\documentclass[uplatex,twocolumn,dvipdfmx]{jsarticle}
\usepackage[top=22mm,bottom=22mm,left=22mm,right=22mm]{geometry}
\setlength{\columnsep}{10mm}
\usepackage[T1]{fontenc}
\usepackage{txfonts}
\usepackage[expert,deluxe]{otf}
\usepackage[dvipdfmx,hiresbb]{graphicx}
\usepackage[dvipdfmx]{hyperref}
\usepackage{pxjahyper}
\usepackage{secdot}






%タイトルと学生番号,名前だけ編集すること
\title{\vspace{-5mm}\fontsize{14pt}{0pt}\selectfont ソーシャルブックマーキングサービスのデータ分析によるウェブマーケティング}
\author{\normalsize プロジェクトマネジメントコース 矢吹研究室 1342029  遠藤一輝}
\date{}
\pagestyle{empty}
\begin{document}
\fontsize{10.5pt}{\baselineskip}\selectfont
\maketitle





%以下が本文
\section{背景}

私たちの周りには様々な情報やニュースがあふれている.その情報量は膨大であり,私たちは日々その膨大な情報の中から自身に必要な情報を取捨選択することで入手している.その取捨選択を行う際に私たちはその情報の様々な要素から,自身にとって必要な情報かを総合的に判断している\cite{yahoo}.\par
情報の要素とはタイトルや掲載された日時などの発見するときに必要なものから,内容やコメントなどの詳細に調べたときまでの様々なものである.\par
そこで数多くある要素のうちどれが決定において大きな影響をもたらしたのかを調査し,人はどのような要素に興味を持つのか,どのような情報の発信をすれば多くの人に見てもらえるのかを分析し可視化することでウェブマーケティングの一つの指標とすることを目標とする.\par
本研究ではウェブマーケティングの例としてソーシャルブックマーキングサービスを用いる.理由としては,様々なジャンルの情報を利用者が登録するため幅広い分野のデータが収集でき,ブックマーク数という共通の数値が得られるため比較が容易となることがあげられる.

\section{目的}

本研究の目的はソーシャルブックマーキングサービスにおける登録ユーザの推移を分析し,その増加率に大きく貢献している要素を算出する.
それをもとに効果的なウェブマーケティング方法を考察する.                                           

\makeatletter
  % sectionの下マージンを小さく
  \renewcommand{\section}{%
    \@startsection{section}{1}{\z@}%
    {0.1\Cvs}{0.1\Cvs}%
    {\normalfont\large\headfont\raggedright}}
\makeatother


\section{手法}

\subsection{サンプルの選出方法}
本研究のサンプルデータは,ソーシャルブックマーキングサービスの一つであるはてなブックマークのAPI\cite{hatena}を使用し取得する.\par
各ジャンルの人気エントリの要素を一日に一回取得し,記録することでサンプルとする.一つのエントリの取得期間は一週間とし,同時に取るエントリの数はapiの上限数である50とする.
\subsection{分析}
得られたサンプルデータを様々な点から分析することで各要素の重要度を算出する.\par
サンプルから得られた要素の中から掲載日時,ブックマーク数,タイトル,ジャンル,タグ,内容,コメント数の七項目について主に考察を行い,その他の要素である記録日時,URL,はてなブックマーク上でのURL,ブックマークタイムスタンプ,記事ID,ブックマークをしたユーザーネームの六項目は研究を進める上での参考データとして活用する.\par
その後サンプルデータに対し分析を行い,得られた結果から効果的なウェブマーケティング方法を考察する.


\section{想定される成果物}

閲覧者に興味を抱かせやすい効果的なウェブマーケティング手法を考案する.
また,それに付随してウェブマーケティングを行う上での各要素の重要度を算出する.


\section{進捗状況}

本研究に用いるデータを収集するために,はてなブックマークのapiを用いた情報を取得するプログラムを制作している.

\section{今後の計画}
以下のように研究を進める計画である.

\begin{enumerate}
\item はてなブックマークから各ジャンルのエントリからデータを記録する.
\item 記録した要素をそれぞれ分析をし,重要度の算出を行う.
\item 分析した結果をもとにウェブマーケティング手法の提案をし,その信憑性の向上を図る.
\end{enumerate}

\bibliographystyle{junsrt}

\bibliography{biblio}%「biblio.bib」というファイルが必要.

\end{document}
