%中間審査概要テンプレート ver. 3.0

\documentclass[uplatex,twocolumn,dvipdfmx]{jsarticle}
\usepackage[top=22mm,bottom=22mm,left=22mm,right=22mm]{geometry}
\setlength{\columnsep}{10mm}
\usepackage[T1]{fontenc}
\usepackage{txfonts}
\usepackage[expert,deluxe]{otf}
\usepackage[dvipdfmx,hiresbb]{graphicx}
\usepackage[dvipdfmx]{hyperref}
\usepackage{pxjahyper}
\usepackage{secdot}





%タイトルと学生番号,名前だけ編集すること
\title{\vspace{-5mm}\fontsize{14pt}{0pt}\selectfont GitHubにおける人的資源マネジメント}
\author{\normalsize プロジェクトマネジメントコース 矢吹研究室 1342081 氏名 辻岡 大知}
\date{}
\pagestyle{empty}
\begin{document}
\fontsize{10.5pt}{\baselineskip}\selectfont
\maketitle





%以下が本文
\section{背景}

ソフトウェア開発では,GitHubを用いることが多い.GitHubとはコンピュータ上で作成,編集されるファイルの変更履歴を管理するためのバージョン管理システムである.複数人でプログラミングを行う場合,ソースコードを効率的に管理,運用する必要がある.GitHubはこのような管理を行うために作られたツールであり,システム開発の現場で一般的に使われているツールの一つである\cite{a}.またGitHubは公開されているソースコードの閲覧や簡単なバグ管理機能,SNS機能を備えている.私たちはGitHubで公開されているプロジェクトを閲覧することにより手軽にプロジェクトについての研究や学習をすることができる.

システムエンジニアはコミュニケーション能力が必要とされる仕事である.私はGitHubに公開されているプロジェクトを調べているうち,活発に活動しているユーザはGoogle+やTwitterで交友関係が広い場合が多いことに気が付いた.そこで私はGitHubを用い,活発に活動するシステムエンジニアはコミュニケーション能力が高いというのではないかという仮説を立て,その仮説を検証するため本研究を行った.


\section{目的}

GitHubで活発に活動しているユーザの交友関係や投稿頻度をGoogle+を用いて調査する.ソフトウェア開発で活発に活動しているユーザはコミュニケーション能力が高いのではないかという仮説を検証する.

\section{手法}

昨年のcontributionが300以上のユーザを活発に活動しているユーザとしてデータの取得を行う.contributionとはユーザがどの程度GitHub上で活動しているのかを表すグラフである.

以下の手法を用いて研究を進める.

\begin{enumerate}
 \item GHTorrent\cite{GHTorrent}を使用しGmailアドレスを登録しており,活発に活動しているユーザのみを抽出する.
 \item 抽出したユーザのGmailアドレスを活用し,Google+におけるユーザのフォロワー数と投稿頻度を調査する.
 \item 抽出したユーザのGoogle+におけるフォロワー数,投稿頻度とGitHubにおけるcontributionとの関係性を調査する.
\end{enumerate}

\section{想定される成果物}

ソフトウェア開発で活発に活動しているシステムエンジニアはコミュニケーション能力が高いというという検証結果が得られる.

\section{進捗状況}

活発に活動しているユーザのデータを取得するためGHTorrentを使用した.しかし現時点でユーザのデータを取得することはできていない.

\section{今後の計画}


以下の順序で研究を進めていく.

\begin{enumerate}
 \item GHTorrentを使用できるよう準備を行う.
 \item GHTorrentを使用しGmailアドレスを登録しており,活発に活動しているユーザのみを抽出する.
 \item 抽出したユーザのGmailアドレスを活用し,Google+におけるユーザのフォロワー数と投稿頻度を調査する.
 \item ソフトウェア開発で活発に活動しているユーザのcontributionとGoogle+でのフォロワー数と投稿頻度の関係性を調査する.
\end{enumerate}




\bibliographystyle{junsrt}
\bibliography{biblio}%「biblio.bib」というファイルが必要.

\end{document}
