%中間審査概要テンプレート ver. 3.0

\documentclass[uplatex,twocolumn,dvipdfmx]{jsarticle}
\usepackage[top=22mm,bottom=22mm,left=22mm,right=22mm]{geometry}
\setlength{\columnsep}{10mm}
\usepackage[T1]{fontenc}
\usepackage{txfonts}
\usepackage[expert,deluxe]{otf}
\usepackage[dvipdfmx,hiresbb]{graphicx}
\usepackage[dvipdfmx]{hyperref}
\usepackage{pxjahyper}
\usepackage{secdot}





%タイトルと学生番号,名前だけ編集すること
\title{\vspace{-5mm}\fontsize{14pt}{0pt}\selectfont ボットを活用するプロジェクトマネジメントツールの提案}
\author{\normalsize プロジェクトマネジメントコース 矢吹研究室 1342097 浜野太豪}
\date{}
\pagestyle{empty}
\begin{document}
\fontsize{10.5pt}{\baselineskip}\selectfont
\maketitle





%以下が本文
\section{背景}
2016年3月24日米マイクロソフト社が開発した人工知能ボットが公開された.ツイッターなどのSNSでユーザとの「会話」を開始した.「Tay(テイ)」と名付けられたこの人工知能ボットは、ユーザのツイートや質問に返信する中で、新しい言葉や会話を学習していくことを目的に開発され、衝撃を与えた.\cite{tay}

他の企業もボットに力を入れている.2016年に発表されたボット開発フレームワークは数多く、有名な企業ではFacebookやLineがあげられる.それらの企業はプラットフォーム用のボット開発フレームワークを公開した.ボット開発フレームワークが公開されたことによってコミュニケーションツールとボットの連携が容易になった.そのため技術者はボットの活用が求められる.

またコミュニケーションツールにSlackというものがある.Slackはリリース当初から爆発的な勢いで世界中に広まっている.当初はスタートアップ企業を中心に利用されていたが,最近ではGoogle,Microsoft,IBM,Sonyなど,世界の名だたる企業も次々に導入し始めている.Slackの利点はサービスとの連携にある.Google DriveやDropbox,Heroku,Githubをはじめ,449ものサービスと連携が可能で、その数は日々増え続けている.\cite{slack}
 




\section{目的}
コミュニケーションツールSlackを利用してChatOpsを実現する.
そしてシステム開発における,複雑な開発環境をSlackに統合する.
統合することによって情報の共有,操作の可視化を行う.



\section{手法}
ボット開発フレームワークbotkitの導入とボットの実装を行う.
ボットの実装では以下の技術を利用する.
\begin{enumerate}
\item Botkit
\item Node.js
\item Heroku
\end{enumerate}


\subsection{ボットの実装}
Slack APIを用いてJavaScriptで作成する.

SlackAPIの例を以下に示す.

hears(コマンド名, メッセージの種類, コマンドの処理)

コマンド名ではユーザからどのようなメッセージに反応するか記述する.メッセージの種類では個人やチャンネルからのメッセージから処理するか記述する.コマンドの処理ではユーザからのメッセージに対しての処理を記述する.





\section{想定される成果物}
運用にかかわる様々なタスクを自動化する

\section{進捗状況}
Slack上の発言からGitHubのIssueを作成するボットの作成を行うことができた.そのためGitHubAPI,Node.jsライブラリの活用方法を理解した.




\section{今後の計画}
以下の順で実行する.
\begin{enumerate}
\item チーム活動に必要になる要件をまとめる.
\item JavaScriptを用いて機能を実装する.
\item ボットを実際に利用してもらう.
\end{enumerate}



\bibliographystyle{junsrt}
\bibliography{biblio}%「biblio.bib」というファイルが必要.

\end{document}
