%中間審査概要テンプレート ver. 3.0

\documentclass[uplatex,twocolumn,dvipdfmx]{jsarticle}
\usepackage[top=22mm,bottom=22mm,left=22mm,right=22mm]{geometry}
\setlength{\columnsep}{10mm}
\usepackage[T1]{fontenc}
\usepackage{txfonts}
\usepackage[expert,deluxe]{otf}
\usepackage[dvipdfmx,hiresbb]{graphicx}
\usepackage[dvipdfmx]{hyperref}
\usepackage{pxjahyper}
\usepackage{secdot}





%タイトルと学生番号,名前だけ編集すること
\title{\vspace{-5mm}\fontsize{14pt}{0pt}\selectfont ボットを活用するプロジェクトマネジメントツールの提案}
\author{\normalsize プロジェクトマネジメントコース 矢吹研究室 1342097 浜野太豪}
\date{}
\pagestyle{empty}
\begin{document}
\fontsize{10.5pt}{\baselineskip}\selectfont
\maketitle





%以下が本文
\section{背景}
クラウドの普及により様々なサービス(アプリケーション)が提供されるようになった.このようなクラウド上で提供されるサービスをSaaS(Software as a Service)と呼ぶ.利用者は自分のパソコンにソフトウェアをインストールすることなくインターネットを介して利用することができる\cite{saas}.近年では特定の機能に特化したSaaSの提供が増えている.システム運用の現場では監視ツールやコードレビューツールやインシデント管理ツールなどのSaaSを用いる.このようなSaaS同士を連携し,作業を自動化することで作業全体を効率化するという使い方がシステム運用の現場では一般化しつつある.しかしこのようなツールを連携するには多くのコストや手間がかかってしまう.

2013年にSlackというChatサービスが生まれた.Slackは他のサービスと連携が容易な構造になっているため\cite{slack}前述したシステム運用ツール(SaaS)の連携時の手間やコストを解決するサービスとして注目され,さまざまなSaaS連携の取り組みが実施されている.特にChatサービスとシステム運用に関するツールとの親和性が高いため,システム運用のツールとChatサービスを組み合わせてシステム運用の効率化を図る「ChatOps」が盛り上がりをみせている.

さらにChatOpsの盛り上がりに伴い,Chatボットの作成が簡単にできるOSSフレームワークFacebookのMessenger PlatformやMicrosoftの bot framework,Slackのbotkitなどが2016年に登場した.今後はChatサービスとSaaSとChatボットという組み合わせによって,更なる作業効率化が期待される.






\section{目的}
コミュニケーションツールSlackを利用してChatOpsを実現する.
システム開発における,複雑な開発環境をSlackに統合する.
統合することによって情報の共有,操作の可視化を行う.



\section{手法}
ボット開発フレームワークbotkitの導入とボットの実装を行う.
Slack APIを用いてJavaScriptで作成する.SlackAPIの例を以下に示す.

hears(コマンド名, メッセージの種類, コマンドの処理)

コマンド名ではユーザからどのようなメッセージに反応するか記述する.メッセージの種類では個人やチャンネルの種類を記述する.
コマンドの種類ではユーザからのメッセージに対しての処理を記述する.






\section{想定される成果物}
Slack上でGitHubなどのSaaSを操作することができるChatボットの開発

\section{進捗状況}
GitHubAPI,Node.jsライブラリの活用方法を理解した.そのため
Slack上の発言からGitHubのIssueを作成するボットの作成を行うことができた.




\section{今後の計画}
今後の作業計画を以下に示す.
\begin{enumerate}
\item チーム活動に必要になる要件をまとめる.
\item JavaScriptを用いて機能を実装する.
\item ボットを実際に利用してもらう.
\end{enumerate}



\bibliographystyle{junsrt}
\bibliography{biblio}%「biblio.bib」というファイルが必要.

\end{document}
