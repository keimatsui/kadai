%中間審査概要テンプレート ver. 3.0

\documentclass[uplatex,twocolumn,dvipdfmx]{jsarticle}
\usepackage[top=22mm,bottom=22mm,left=22mm,right=22mm]{geometry}
\setlength{\columnsep}{10mm}
\usepackage[T1]{fontenc}
\usepackage{txfonts}
\usepackage[expert,deluxe]{otf}
\usepackage[dvipdfmx,hiresbb]{graphicx}
\usepackage[dvipdfmx]{hyperref}
\usepackage{pxjahyper}
\usepackage{secdot}





%タイトルと学生番号,名前だけ編集すること
\title{\vspace{-5mm}\fontsize{14pt}{0pt}\selectfont ニコニコ動画のカテゴリ合算毎時総合ランキングの順位とTwiiterのツイート数の相関分析}
\author{\normalsize プロジェクトマネジメントコース 矢吹研究室 1342073 杉山喜彦}
\date{}
\pagestyle{empty}
\begin{document}
\fontsize{10.5pt}{\baselineskip}\selectfont
\maketitle





%以下が本文
\section{背景}


ニコニコ動画とは,株式会社ドワンゴが運営・提供している動画共有サービスである.ニコニコ動画の利用者の数は一般会員登録者数が約5000万人,有料会員は約250万人(2015年8月)である\cite{iii}.ニコニコ動画のランキングにカテゴリ合算毎時総合ランキングという項目がある.これはニコニコ動画に投稿された動画の再生数・コメント数・登録マイリスト数・ニコニコ広告宣伝ポイントを総合ポイントに変換し1時間毎にランキング順にしたものである.

Twitterは,「ツイート」と称される140文字以内の短文の投稿を共有するウェブ上の情報サービスである.

ニコニコ動画を利用している時に,毎回開いているページがカテゴリ合算毎時総合ランキングである.動画でランキングを上げるために投稿者が動画の説明欄に投稿者のTwiiterへ行くことができるリンクが貼られていた.そこで私はニコニコ動画のカテゴリ合算毎時総合ランキングとTwiiterのツイート数には相関性があると考えた.


\noindent



\section{目的}
ニコニコ動画のカテゴリ合算毎時総合ランキングの順位とTwiiterのツイート数との相関性があるかを調べる.

\section{手法}

以下の手法で研究を行う.

\begin{enumerate}

\item ニコニコ動画のカテゴリ合算毎時総合ランキングの1位から100位までの投稿動画の再生数を1時間毎に抜き出す.
\item 時間毎に増加していく再生数の累積のグラフを作成する.このグラフを①とする.
\item 再生数の増加を1時間毎に区切ったグラフを作成する.このグラフを②とする.
\item Twiiterでニコニコ動画のカテゴリ合算毎時総合ランキングの1位から100位までの動画の名前でツイートの検索し,1時間毎にツイート数を抜き出す.
\item 時間毎に増加していくツイート数の累積させたグラフを作成する.このグラフを③とする.
\item ツイート数の増加を1時間毎に区切ったグラフを作成する.このグラフを④とする.
\item ①と③,②と④の2通りの相関分析を行い,カテゴリ合算毎時総合ランキングとTwiiterのツイート数との相関性があるかを考察する.

\end{enumerate}

\section{想定される成果物}
①と③,②と④をそれぞれ100以上作成して相関分析を行う.この分析結果から相関性があるかないかを判断する.

\section{進捗状況}
ニコニコ動画のカテゴリ合算毎時総合ランキングから1時間毎の再生数を記録することができた.Twiterのツイート数を収集するAPIが使用できるかを確認し,データを収集する.

\section{今後の計画}
ニコニコ動画のカテゴリ合算毎時総合ランキングの101位以下のデータを取ることができるかを確認する.次にTwiiterのツイート数を収集するAPIの使用ができるかを確認する.次にデータの収集を行い①,②,③,④の4種類のグラフをそれぞれ100以上作成して相関分析を行う.最後にこの分析結果からニコニコ動画のカテゴリ合算毎時総合ランキングの順位とTwiiterのツイート数との間に,相関性があるかないかを判断する.


\bibliographystyle{junsrt}
\bibliography{biblio}%「biblio.bib」というファイルが必要.

\end{document}
