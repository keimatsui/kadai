%中間審査概要テンプレート ver. 3.0

\documentclass[uplatex,twocolumn,dvipdfmx]{jsarticle}
\usepackage[top=22mm,bottom=22mm,left=22mm,right=22mm]{geometry}
\setlength{\columnsep}{10mm}
\usepackage[T1]{fontenc}
\usepackage{txfonts}
\usepackage[expert,deluxe]{otf}
\usepackage[dvipdfmx,hiresbb]{graphicx}
\usepackage[dvipdfmx]{hyperref}
\usepackage{pxjahyper}
\usepackage{secdot}





%タイトルと学生番号,名前だけ編集すること
\title{\vspace{-5mm}\fontsize{14pt}{0pt}\selectfont ニコニコ動画のカテゴリ合算毎時総合ランキングとTwiiterのツイートと関連性があるのか}
\author{\normalsize プロジェクトマネジメントコース 矢吹研究室 1342073 杉山喜彦}
\date{}
\pagestyle{empty}
\begin{document}
\fontsize{10.5pt}{\baselineskip}\selectfont
\maketitle





%以下が本文
\section{背景}


ニコニコ動画とは,株式会社ドワンゴが運営・提供している動画共有サービスである.ニコニコ動画の利用者の数は一般会員登録者数が約5000万人,有料会員は約250万人(2015年8月)である\cite{iii}.ニコニコ動画のランキングにカテゴリ合算毎時総合ランキングという項目がある.これはニコニコ動画に投稿された動画の再生数・コメント数・登録マイリスト数・ニコニコ広告宣伝ポイントを総合ポイントに変換し1時間毎にランキング順にしたものである.

Twitterは,「ツイート」と称される140文字以内の短文の投稿を共有するウェブ上の情報サービスである.月当たりのアクティブユーザは全世界で3億2000万人,日本国内でのアクティブユーザは3500万人である.Twiiterでツイートされたものは不特定多数のユーザが観覧することができる.

ニコニコ動画を利用している時に,毎回開いているページがカテゴリ合算毎時総合ランキングである.動画でランキングを上げるために投稿者が動画の説明欄に投稿者のTwiiterへ行くことができるリンクが貼られていた.そこで私はニコニコ動画のカテゴリ合算毎時総合ランキングとTwiiterのツイート数には関連性があると考えた.


\noindent



\section{目的}
ニコニコ動画のカテゴリ合算毎時総合ランキングの順位とTwiiterのツイート数との関連性があるのかどうかを調べる.

\section{手法}

ニコニコ動画のカテゴリ合算毎時総合ランキングの1位から100位までの投稿動画の再生数を1時間毎に抜きだし,時間毎に増加していく再生数の累積のグラフと再生数の増加を1時間毎に区切ったグラフを作成する.
次に,Twiiterでニコニコ動画のカテゴリ合算毎時総合ランキングの1位から100位までの動画の名前でツイートの検索し,ツイート数を時間毎に累積させたグラフとツイート数の増加を1時間毎に区切ったグラフを作成する.
最後に,再生数の累積のグラフとツイート数の累積のグラフ,再生数の増加を1時間毎に区切ったグラフとツイート数の増加を1時間毎に区切ったグラフの2通りの相関分析を行い,カテゴリ合算毎時総合ランキングとTwiiterの関連性があるかを考察する.

\section{想定される成果物}
ニコニコ動画のカテゴリ合算毎時総合ランキングで作成した再生数の累積のグラフとTwiiterの検索結果から作成したツイート数の累積のグラフ,再生数の増加を1時間毎に区切ったグラフととツイート数の増加1時間毎に区切ったグラフをそれぞれ100以上作成して相関分析を行う.この分析結果から相関関係があるかないかを判断する.

\section{進捗状況}
ニコニコ動画のカテゴリ合算毎時総合ランキングから1時間毎の再生数を記録することはできた.Twiterのツイート数を収集するAPIが使用できるかを確認し,データを収集する.

\section{今後の計画}
ニコニコ動画のカテゴリ合算毎時総合ランキングの101位以下のデータを取ることができるのかの確認を行い,Twiiterのツイート数を収集してAPIの使用できるかを確認する.その後データの収集へ移り,収集期間を決めて再生数の累積のグラフと再生数の増加を1時間毎に区切ったグラフ,ツイート数の累積のグラフ,ツイート数の増加を1時間毎に区切ったグラフの4種類のグラフを作成して相関分析を行う.



\bibliographystyle{junsrt}
\bibliography{biblio}%「biblio.bib」というファイルが必要.

\end{document}
