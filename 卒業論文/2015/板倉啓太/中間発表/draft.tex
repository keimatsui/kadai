%中間審査概要テンプレート ver. 3.0

\documentclass[uplatex,twocolumn,dvipdfmx]{jsarticle}
\usepackage[top=22mm,bottom=22mm,left=22mm,right=22mm]{geometry}
\setlength{\columnsep}{10mm}
\usepackage[T1]{fontenc}
\usepackage{txfonts}
\usepackage[expert,deluxe]{otf}
\usepackage[dvipdfmx,hiresbb]{graphicx}
\usepackage[dvipdfmx]{hyperref}
\usepackage{pxjahyper}
\usepackage{secdot}






%タイトルと学生番号,名前だけ編集すること
\title{\vspace{-5mm}\fontsize{14pt}{0pt}\selectfont  データ分析教育へのアクティブラーニング手法の導入提案と実践}
\author{\normalsize プロジェクトマネジメントコース 矢吹研究室 1342015 板倉 啓太}
\date{}
\pagestyle{empty}
\begin{document}
\fontsize{10.5pt}{\baselineskip}\selectfont
\maketitle




%以下が本文
\section{背景}

大学で教育改革が進む中,多くの大学がアクティブ・ラーニングを導入している.アクティブ・ラーニングとは「能動的な学習」のことで,講師が一方的に学生に知識伝達をする講義形式ではなく,課題研究やPBL(プロジェクト・ベースド・ラーニング),ディスカッション,プレゼンテーションなど,学生の能動的な学習法の総称である.アクティブ・ラーニングが示す授業の形態や内容は非常に広く,その目的も大学や学部・学科によってさまざまである.1980年代までは,人材育成において中等教育の果たす役割が重視されていた.

しかし,80年代には情報化社会が到来し,90年代に入るとインターネットも登場して情報化が加速した\cite{a}.同時に,1990年代以降は,新しい知識,情報,技術が政治・経済・文化をはじめ社会のあらゆる領域での活動の基盤として飛躍的に重要性を増す「知識基盤社会」の時代を迎えた.これにより基礎的な知識に加え,多様性・創造性や他者と交渉する力などを備えた新しい社会を創出できる人材が求められるようになった.

こうした中で,より質の高い学習や教育を実現するために効果的な学習法として,アクティブ・ラーニングが注目を集めているのである.




\section{目的}

PM学科のPMコース・JABEEコースのデータマイニング入門を受講した学生を対象に,アクティブ・ラーニングをデータマイニング教育に取り入れ,受講者の能動的な学習への参加を取り入れた能力の育成を図る.受講者自身は,与えられたデータをマイニングするだけではなく,データをどうやって集めるか,データ収集法の設計から考え,学習することになる\cite{b}.



\section{手法}

以下の手法で研究する.

\begin{enumerate}

\item 受講者を4,5人で1グループに分ける.
\item 勉学を題材とした質問を各グループ,3つずつ考えてもらう.
\item 質問をGoogleフォームにまとめ,アンケートを作成して受講者に回答してもらう.
\item 解析手法を学んだ後,自分のグループの質問の結果と全ての質問の結果をデータマイニングしてもらい,その結果から考察を交えて発表してもらう.

\end{enumerate}


\section{想定される成果物}

受講者の能動的な学習への参加を取り入れた能力の育成.
グループ活動や課題解決型の手法を導入することにより,知識の定着が促進されたり,新しい発想が生まれたりする.


\section{進捗状況}

現在,データマイニング入門の指導教員である矢吹太朗准教授にアクティブ・ラーニング手法の導入の提案をし,手法と実践日ついて調整している.



\section{今後の計画}

今後のデータマイニング入門の講義における計画は以下の通りである.
\begin{table}[hbtp]
  \caption{今後のデータマイニング入門の講義における計画}
  \label{table:data_type}
  \centering
  \begin{tabular}{|l|l|}
    \hline
    日程 & 内容  \\ \hline \hline
    10/24 & 受講者を各グループ4,5人に分ける \\
    10/31 & グループで勉学に関する質問を決める \\
    11/7 & アンケートを作成し,受講者は回答する \\
    12/12 & 受講者は質問結果をマイニングし発表する \\
 \hline
  \end{tabular}
\end{table}

\bibliographystyle{junsrt}
\bibliography{biblio}%「biblio.bib」というファイルが必要.

\end{document}
