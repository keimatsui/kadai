%中間審査概要テンプレート ver. 3.0

\documentclass[uplatex,twocolumn,dvipdfmx]{jsarticle}
\usepackage[top=22mm,bottom=22mm,left=22mm,right=22mm]{geometry}
\setlength{\columnsep}{10mm}
\usepackage[T1]{fontenc}
\usepackage{txfonts}
\usepackage[expert,deluxe]{otf}
\usepackage[dvipdfmx,hiresbb]{graphicx}
\usepackage[dvipdfmx]{hyperref}
\usepackage{pxjahyper}
\usepackage{secdot}






%タイトルと学生番号,名前だけ編集すること
\title{\vspace{-5mm}\fontsize{14pt}{0pt}\selectfont  データ分析教育へのアクティブラーニング手法の導入提案と実践}
\author{\normalsize プロジェクトマネジメントコース 矢吹研究室 1342015 板倉 啓太}
\date{}
\pagestyle{empty}
\begin{document}
\fontsize{10.5pt}{\baselineskip}\selectfont
\maketitle




%以下が本文
\section{背景}

大学で教育改革が進む中,多くの大学がアクティブ・ラーニングを導入している.アクティブ・ラーニングとは「能動的な学習」のことで,講師が一方的に学生に知識伝達をする講義形式ではなく,課題研究やPBL(プロジェクト・ベースド・ラーニング),ディスカッション,プレゼンテーションなど,学生の能動的な学習法の総称である.アクティブ・ラーニングが示す授業の形態や内容は非常に広く,その目的も大学や学部・学科によってさまざまである.能動的な学習には,書く・話す・発表するなどの活動への関与と, そこで生じる認知プロセスの外化を伴う.1980年代までは,人材育成において中等教育の果たす役割が重視されていた\cite{a}.

しかし,1990年代以降は新しい知識,情報,技術が政治・経済・文化をはじめ社会のあらゆる領域での活動の基盤として飛躍的に重要性を増す「知識基盤社会」の時代を迎えた.これにより基礎的な知識に加え,多様性・創造性や他者と交渉する力などを備えた新しい社会を創出できる人材が求められるようになった.

こうした中で,より質の高い学習や教育を実現するために効果的な学習法として,アクティブ・ラーニングが注目を集めている.




\section{目的}

PM学科のPMコース・JABEEコースのデータマイニング入門を受講した学生を対象に,アクティブ・ラーニングをデータマイニング教育に取り入れ,受講者の能動的な学習への参加を取り入れた能力の育成を図る.受講者自身は,与えられたデータをマイニングするだけではなく,データをどうやって集めるか,データ収集方法の設計から考え,学習することになる\cite{b}.この研究では学習者自身を被験者とする.



\section{手法}

本研究では,千葉工業大学データマイニング入門を受講している学生137人で4週間に分けて行う.大学生にとっての勉強を題材とし,データを収集する.1グループ4,5人になるよう分け,全33グループで行う.受講者は何が知りたいかを考え,各グループで質問を3つずつ考えてもらう.それをGoogleフォームにまとめ,アンケートを作成してすべての受講者に回答してもらう.講義で解析手法を学んだ後,自分のグループの質問の結果と全ての質問結果の2つをデータマイニングしてもらう.その結果から考察を交え発表してもらう.



\section{結果}

データマイニング入門でアクティブラーニングを取り入れ,成果物発表を行った.発表から有意な結果はあまり得られなかった.しかしながら,学習者自身が大学生にとっての勉強と相関があるかを考えて質問を作ったことで,学習者の能動的な学習への参加を取り入れた能力の育成をすることができた.


\section{考察}

今回4週間でグループワークを行ったが,質問を考える段階からどのような結果が得られるのかを想定して計画していく必要があった.
また,各週の講義内でグループワークを割り当てるよう設計するとさらに効果的なアクティブラーニングを行うことができるだろう.





\section{結論}

今回の結果とその考察を活かすことによって,データマイニング教育に学習者の能動的な学習への参加を取り入れた能力の育成ができるだろう.


\bibliographystyle{junsrt}
\bibliography{biblio}%「biblio.bib」というファイルが必要.

\end{document}
