%卒論概要テンプレート ver. 4.0

\documentclass[uplatex,twocolumn,dvipdfmx]{jsarticle}
\usepackage[top=22mm,bottom=22mm,left=22mm,right=22mm]{geometry}
\setlength{\columnsep}{11mm}
\usepackage[T1]{fontenc}
\usepackage{txfonts}
\usepackage[expert,deluxe]{otf}
\usepackage[dvipdfmx,hiresbb]{graphicx}
\usepackage[dvipdfmx]{hyperref}
\usepackage{pxjahyper}
\usepackage{secdot}





%タイトルと学生番号,名前だけ編集すること
\title{\vspace{-5mm}\fontsize{14pt}{0pt}\selectfont ディープラーニングを用いたWebサイトデザインの年代解析}
\author{\normalsize プロジェクトマネジメントコース 矢吹研究室 1442104 増田準}
\date{}
\pagestyle{empty}
\begin{document}
\fontsize{10.5pt}{\baselineskip}\selectfont
\maketitle





%以下が本文
\section{序論}\label{序論}

Webサイトのデザインは,時代に合ったものが求められる\cite{bib001}.スマートフォンの爆発的な普及により,Webサイトは急速に発展を遂げた.Webサイトをデザインするということは,視覚的な良し悪しを求めるだけでなく使いやすさなど様々な要素を含む.その為Webサイトを閲覧するデバイスによってデザインを変える事もあり,現代におけるWebデザインの多様化は著しい.以上のことから,本研究では時代によって進化するWebデザインの解析を対象とする.

\section{目的}

この研究の目的は,年代ごとのWebデザインの変化を解析することである.デザインとは数値などで表すことができるものではなく,漠然としたものである場合が多い.その為,解析の際はページに映る要素を総合的に判断させることが重要だ.

\section{手法}

この研究は以下の手法を用いて行う. 

\subsection{画像解析}

機械学習による画像解析を利用する.この研究における画像解析とは,多数の教師画像を学習させ判別モデルを作成し,別の画像を判別させることで画像の特徴を解析する処理を指す.

\subsection{画像の収集}

Fortune Global 500\cite{bib002}にリストされた企業の,過去のホームページをInternet Archiveで閲覧し,そのスクリーンショットを撮る. 


\section{結果}

上記した手法に乗っ取り,14422枚の画像を取得した.その画像を使用し,数式処理ソフトMathematicaにて教師画像14322枚,テスト画像100枚で画像解析を行った.結果は次の図1の通りである.縦軸が算出された年代の値で,横軸が正解の値となっている.\vspace{0.2in} \\

\vspace{-1zh}
\begin{figure}[htb]
\centering
\includegraphics[width=6cm,clip]{graph.pdf}
\caption{Mathematicaによる解析結果}
\end{figure}
\vspace{-1zh}

また,ディープラーニング用のツールNeural Network Consoleにて,Webサイトの公開年を1996から2002,2003から2009,2010から2017という3世代に分類し画像解析した.結果は,正解率が40.75パーセント,予測と正解の相関係数は0.56となった.

\section{考察}

正解のばらつきが発生した原因は教師画像の不足していたことと,学習方法が最適でなかったことが考えられる.Mathematicaによる解析では,最終的に教師画像14322枚で学習させたが,枚数を増やすごとにばらつきは少なくなった.また,Neural Network Consoleでは学習モデルの構築を最適化する機能によって同じ教師画像で正解率を上げることもできた.

\section{結論}

年代ごとのWebデザインの変化を解析した.しかし,この研究における手法を用いての解析は,より多くの教師画像とより最適な学習モデルが必要である.

\bibliographystyle{junsrt}
\bibliography{biblio}%「biblio.bib」というファイルが必要.

\end{document}
