%中間審査概要テンプレート ver. 3.0

\documentclass[uplatex,twocolumn,dvipdfmx]{jsarticle}
\usepackage[top=22mm,bottom=22mm,left=22mm,right=22mm]{geometry}
\setlength{\columnsep}{10mm}
\usepackage[T1]{fontenc}
\usepackage{txfonts}
\usepackage[expert,deluxe]{otf}
\usepackage[dvipdfmx,hiresbb]{graphicx}
\usepackage[dvipdfmx]{hyperref}
\usepackage{pxjahyper}
\usepackage{secdot}





%タイトルと学生番号,名前だけ編集すること
\title{\vspace{-5mm}\fontsize{14pt}{0pt}\selectfont ディープラーニングを用いたWebサイトデザインの年代推定}
\author{\normalsize プロジェクトマネジメントコース 矢吹研究室 1442104 増田準}
\date{}
\pagestyle{empty}
\begin{document}
\fontsize{10.5pt}{\baselineskip}\selectfont
\maketitle





%以下が本文
\section{背景}

Webサイトのデザインは,時代に合ったものが求められる\cite{bib001}.スマートフォンの爆発的な普及により,Webサイトは急速に発展を遂げた.Webサイトをデザインするということは,視覚的な良し悪しを求めるだけでなく使いやすさなど様々な要素を含む.その為Webサイトを閲覧するデバイスによってデザインを変える事などもあり,現代におけるWebデザインの多様化は著しい.更に,人々の生活に密接に影響していることから,Webサイトに対する研究は学際的に取り組まれている\cite{bib002}.以上のことから,本研究では時代によって進化するWebデザインの解析を対象とする.

\section{目的}

この研究では,年代ごとのWebデザインの変化を解析することを目的とする.デザインとは数値などで表すことができるものではなく,漠然としたものである場合が多い.その為,解析の際はWebサイトに使用されている文字のフォントやその配置,色彩や画像など,ページに映る要素を総合的に判断させることが重要だ.

\section{手法}

この研究は以下の手法を用いて行う.

\subsection{機械学習}

機械学習による画像解析を利用する.この研究における画像解析とは,多数の教師画像を学習させ判別モデルを作成し,別の画像を判別させることで画像の特徴を解析する処理を指す.


\subsection{画像の収集}

Internet Archiveにて閲覧できるWebページを,Seleniumによるスクリーンショットを用いて多数保存する.保存の対象となるWebサイトは2017年度版のFortune Global 500\cite{bib003}にリストされた企業のホームページとする.

\subsection{画像解析}

保存する画像は,Webサイトの公開年を1996から2002,2003から2009,2010から2017という3世代に分類しタグ付けする.そしてこれらの画像を用いて,機械学習による画像解析を行いWebデザインのみで年代を判別可能にする.


\section{想定される成果物}

想定される成果物は以下のとおりである.

\begin{itemize}
\item Webサイトの年代解析結果及び予測モデル
\item Webサイトの年代が判別可能となる教師画像のデータセット
\item 機械学習による年代解析用のコード
\end{itemize}

\section{進捗状況}

上記した画像の収集にある手順で教師画像を集めた.スクリーンショットを実行するコードにおいてサイズの指定をしているが,キャプチャのサイズにばらつきが出ることや,Internet Archiveを読み込む際の処理落ちによりエラー表示のまま保存されているものが多かった.エラー表示のまま保存された画像を除き,7000枚強ほどの教師画像を収集した.

\section{今後の計画}

機械学習による画像解析用のコードの書き方を習得し,解析を実行する.解析の結果によって教師画像の追加や,リサイズを洗練させるなどして正解率の向上を図る.

\bibliographystyle{junsrt}
\bibliography{biblio}%「biblio.bib」というファイルが必要.

\end{document}
