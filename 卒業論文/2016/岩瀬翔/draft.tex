%中間審査概要テンプレート ver. 3.0

\documentclass[uplatex,twocolumn,dvipdfmx]{jsarticle}
\usepackage[top=22mm,bottom=22mm,left=22mm,right=22mm]{geometry}
\setlength{\columnsep}{10mm}
\usepackage[T1]{fontenc}
\usepackage{txfonts}
\usepackage[expert,deluxe]{otf}
\usepackage[dvipdfmx,hiresbb]{graphicx}
\usepackage[dvipdfmx]{hyperref}
\usepackage{pxjahyper}
\usepackage{secdot}





%タイトルと学生番号,名前だけ編集すること
\title{\vspace{-5mm}\fontsize{14pt}{0pt}\selectfont Web サービスの障害がプロジェクトに与える影響を Twitter を活用して調査する方法}
\author{\normalsize プロジェクトマネジメントコース 矢吹研究室 1442012 岩瀬翔}
\date{}
\pagestyle{empty}
\begin{document}
\fontsize{10.5pt}{\baselineskip}\selectfont
\maketitle





%以下が本文
\section{背景}
複数のメンバが同時に開発を行うソフトウェア開発プロジェクトでは,Webサービスを使うことがある.例えば,チーム内でファイルのバージョンを管理することのできる「GitHub」というサービスがある\cite{01}.そのGitHubのサーバーが2016年1月28日にダウンし,インターネット上で話題になった\cite{02}.

GitHubのサーバーがダウンしてしまった実例から,プロジェクトにも影響を及ぼすのではないかと考えた.したがって,GitHubで障害が発生した場合どのような影響が発生しているのか調査する.
\section{目的}
本研究の目的はソフトウェア開発プロジェクトで使用されるWebサービスに障害が発生した場合,どのような影響が発生し,その影響がどれほどの人に及ぶのか調査すること.また,調査結果から障害発生に対するリスク対策案を検討することである.
\section{手法}
調査方法はTwitterで投稿されている障害発生に関するツイートをデータとして収集する.TwitterではAPIが提供されているが,仕様によって過去のデータが取りきれないので以下の手順で行う\cite{03}.
\begin{enumerate}
 \item Twitterの検索結果を自動的に一番下までスクロールして,そのページ全体をHTMLで保存するプログラムとHTMLファイルからツイートの時間と本文のみを抽出するためのプログラムを作成する.
 \item GitHubに関連するすべてのサービスを継続的に状況監視している「GitHub Status」を参照し,2016年で主要なサービスが停止,復旧したと記録されている時間を調べる.その時間前後を作成したプログラムを用いてTwitterで検索し,ページ全体をHTMLで保存・データ抽出をする.これを各障害の発生時間ごとに繰り返す.
 \item 障害発生から復旧までのGitHubに対するツイート数と,どのくらいの時間で復旧が完了するのかを調べる.調査結果をもとに,リスク対策案を検討する.
\end{enumerate}
\section{想定される成果物}
GitHubにおける障害発生から復旧までに投稿されたツイート数のグラフ及び,障害の発生から復旧までの時間でグラフを作成する.これらのグラフから考察することにより,障害発生に対するリスク対策案を検討する.
\section{進捗状況}
GitHub Statusを参照に調べたところ,2016年のGitHubにおける障害発生回数は14回であった(ただし,10月21日の障害は同時にTwitterもダウンしていたため対象に加えない).それらの障害についてはデータの収集を完了し,障害の発生から復旧までに投稿されたツイート数のグラフ及び障害の発生から復旧までの時間でグラフを作成することができた.
\section{今後の計画}
今回取得したツイートが,Twitterにて販売されているデータと比べ,抜けや間違いがないか調査することを検討している.
\bibliographystyle{junsrt}
\bibliography{biblio}%「biblio.bib」というファイルが必要.

\end{document}
