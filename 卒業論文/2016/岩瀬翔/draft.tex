%中間審査概要テンプレート ver. 3.0

\documentclass[uplatex,twocolumn,dvipdfmx]{jsarticle}
\usepackage[top=22mm,bottom=22mm,left=22mm,right=22mm]{geometry}
\setlength{\columnsep}{10mm}
\usepackage[T1]{fontenc}
\usepackage{txfonts}
\usepackage[expert,deluxe]{otf}
\usepackage[dvipdfmx,hiresbb]{graphicx}
\usepackage[dvipdfmx]{hyperref}
\usepackage{pxjahyper}
\usepackage{secdot}





%タイトルと学生番号,名前だけ編集すること
\title{\vspace{-5mm}\fontsize{14pt}{0pt}\selectfont Twitter発言の分析によるWebサービス障害の影響調査}
\author{\normalsize プロジェクトマネジメントコース 矢吹研究室 1442012 岩瀬翔}
\date{}
\pagestyle{empty}
\begin{document}
\fontsize{10.5pt}{\baselineskip}\selectfont
\maketitle





%以下が本文
\section{背景}
複数のメンバが同時に開発を行うソフトウェア開発プロジェクトでは,Webサービスを使うことがある.例えば,チーム内でファイルのバージョンを管理することのできる「GitHub」というサービスがある\cite{01}.そのGitHubのサーバーが2016年1月28日にダウンし,インターネット上で話題になった\cite{02}.

GitHubのサーバーがダウンしてしまった実例から,Webサービスの障害がプロジェクトにも影響を及ぼすのではないかと考えた.したがって,GitHubで障害が発生した場合どのような影響が発生しているのか調査する.
\section{目的}
本研究の目的はソフトウェア開発プロジェクトで使用されるWebサービスに障害が発生した場合,どのような影響が発生し,その影響がどれほどの人に及ぶのか調査する.また,調査結果から障害発生に対するリスク対策案を検討する.
\section{手法}
調査方法はTwitterで投稿されているGitHubの障害発生に関するツイートをデータとして収集する.TwitterではAPIが提供されているが,仕様によって過去のデータが取りきれないので以下の手順で行う.
\begin{enumerate}
 \item GitHubに関連するすべてのサービスを継続的に状況監視している「GitHub Status」を参照し,2016年で主要なサービスが停止,復旧したと記録されている時間を調べる.その時間前後をTwitterで検索し,プログラムを使って時間と本文のみを抽出する.これを各障害の発生時間ごとに繰り返す\cite{03}.
 \item サービスの停止から復旧までのGitHubに対するツイート数と,どのくらいの時間で復旧が完了するのかを調べる.調査結果をもとに,リスク対策案を検討する.
\end{enumerate}
\section{想定される成果物}
GitHubにおけるサービスの停止から復旧までに投稿されたツイート数のグラフ及び,サービスの停止から復旧までの時間間隔でグラフを作成する.これらのグラフから考察することにより,障害発生に対するリスク対策案を検討する.
\section{進捗状況}
GitHub Statusを参照に調べたところ,2016年のGitHubにおける障害発生回数は14回であった(ただし,10月21日の障害は同時にTwitterも障害が発生していたため対象に加えない).それらの障害についてはデータの収集を完了し,サービスの停止から復旧までに投稿されたツイート数のグラフ及びサービス停止から復旧までの時間間隔でグラフを作成することができた.

この2つグラフによって,サービスの停止した時間帯や時間間隔が様々であることがわかった.そして,停止から復旧までの間隔が同じくらいでも時間帯によってツイート数に違いがあることがわかった.
\section{今後の計画}
現段階では2016年のデータしか集めていないため,更に過去のツイートや2017年のデータを集めることを検討する.
\bibliographystyle{junsrt}
\bibliography{biblio}%「biblio.bib」というファイルが必要.

\end{document}
