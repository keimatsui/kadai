%卒論概要テンプレート ver. 4.0

\documentclass[uplatex,twocolumn,dvipdfmx]{jsarticle}
\usepackage[top=22mm,bottom=22mm,left=22mm,right=22mm]{geometry}
\setlength{\columnsep}{11mm}
\usepackage[T1]{fontenc}
\usepackage{txfonts}
\usepackage[expert,deluxe]{otf}
\usepackage[dvipdfmx,hiresbb]{graphicx}
\usepackage[dvipdfmx]{hyperref}
\usepackage{pxjahyper}
\usepackage{secdot}





%タイトルと学生番号,名前だけ編集すること
\title{\vspace{-5mm}\fontsize{14pt}{0pt}\selectfont プロジェクトで発生するリスクのMBTI を用いた事前予測}
\author{\normalsize プロジェクトマネジメントコース 矢吹研究室 1442085 中村 真悟}
\date{}
\pagestyle{empty}
\begin{document}
\fontsize{10.5pt}{\baselineskip}\selectfont
\maketitle





%以下が本文
\section{序論}\label{序論}
MBTI(Myers-Briggs Type Indicator)という自己理解メソッドがある.MBTIとはカール・グスタフ・ユングの心理学的類型論の指標(内向:I-外向:E,感覚:S-直感:N,思考:T-感情:F)に判断的態度:J-知覚的態度:Pの指標を加えて,4指標16タイプとして性格を分類する.主に相談場面や教育現場,企業の組織編制,人事政策などに利用されている\cite{110001230195}.

\section{目的}

本研究の目的は,メンバのMBTIのタイプの相互作用がプロジェクトのリスクにどう影響を及ぼしているのかを調べ,メンバ間で発生しやすいリスクを予測することである.
\section{手法}

以下の手法で研究する.
\begin{enumerate}
\item グループワークで課題に取り組んでもらう.
\item グループワーク後に,性格検査と発生したリスクについてのアンケートを行う.
\item 集めた回答結果をトレーニング用とテスト用にデータを分ける.
\item トレーニング用データからメンバの性格とリスクのルールを見つける.
\item 抽出したルールとテストデータを参照し,調和平均を求める.
\item 調和平均の値が最も高くなるルールの確信度を求める.
\end{enumerate}

\section{結果}

調和平均の計算結果は表\ref{調和平均算出結果}の通りである.

39グループの性格検査とアンケートの結果をトレーニングデータ30件とテストデータ9件の2つに分けた.トレーニングデータをアソシエーション分析し,95件のルールを抽出した.

抽出したルールの正当性を確認するため,ルールとテストデータを参照し調和平均を求めた.調和平均は,確信度0.8を越えたルールだけにすると値が最も良くなった.
\renewcommand{\arraystretch}{0.75}
\begin{table}[htbp]
\centering
\caption{確信度0.8を越えたルールの調和平均}\label{調和平均算出結果}
\begin{tabular}{l|r}
\hline
精度 & 0.25\\
再現率 & 0.863636364\\
調和平均 & 0.387755102\\
\hline
\end{tabular}
\end{table}
\renewcommand{\arraystretch}{0.75}


\section{考察}
今回の結果から発生したリスクとメンバのMBTIのタイプには規則性があると考えられる.また,確信度0.8を越えたルールは0.864の確率で発生しており,リスクの予測にも利用することができると考えられる.

しかし,データが少なかったため,抽出されたルールも少ない.

データを取る対象を増やし,より多くのデータを集めれば今回の結果が変わる可能性がある.

\section{結論}
本研究では,グループワークからメンバのMBTI,発生したリスクをアンケートを用いて集め,相関関係を調べた.その結果,MBTIのタイプが相互作用し発生するリスクに規則性があることがわかった.

今後もデータを集めていけば,より多くのルールが見つかり,リスクが最も少ないグループの提案につながることが期待される.
\nocite{MBTI}
\nocite{110009915588}
\bibliographystyle{junsrt}
\bibliography{biblio}%「biblio.bib」というファイルが必要.

\end{document}
