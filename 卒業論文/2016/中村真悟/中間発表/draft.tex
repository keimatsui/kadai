%中間審査概要テンプレート ver. 3.0

\documentclass[uplatex,twocolumn,dvipdfmx]{jsarticle}
\usepackage[top=22mm,bottom=22mm,left=22mm,right=22mm]{geometry}
\setlength{\columnsep}{10mm}
\usepackage[T1]{fontenc}
\usepackage{txfonts}
\usepackage[expert,deluxe]{otf}
\usepackage[dvipdfmx,hiresbb]{graphicx}
\usepackage[dvipdfmx]{hyperref}
\usepackage{pxjahyper}
\usepackage{secdot}





%タイトルと学生番号,名前だけ編集すること
\title{\vspace{-5mm}\fontsize{14pt}{0pt}\selectfont プロジェクトで発生するリスクのMBTIを用いた事前予測}
\author{\normalsize プロジェクトマネジメントコース 矢吹研究室 1442085 中村 真悟}
\date{}
\pagestyle{empty}
\begin{document}
\fontsize{10.5pt}{\baselineskip}\selectfont
\maketitle





%以下が本文

\section{背景}
MBTI(Myers-Briggs Type Indicator)という自己理解メソッドがある.MBTIとはカール・グスタフ・ユングの心理学的類型論の指標(内向:I-外向:E,感覚:S-直感:N,思考:T-感情:F)に判断的態度:J-知覚的態度:Pの指標を加えて,4指標16タイプとして性格を分類する.主に相談場面や教育現場,企業の組織編制,人事政策などに利用されている\cite{110001230195}.

このMBTIを使い,プロジェクトの開始時点からメンバの性格を理解し,メンバの相互作用が原因となって起こる事象を予測したい.本研究ではMBTIを用いて,グループワークでの事象とメンバの性格との相関関係について研究する.

\section{目的}
本研究の目的は,グループメンバのMBTIの16タイプの相互作用がプロジェクトにどのような影響をもたらしているのかを調べ,MBTIのタイプからメンバ間で発生しやすいリスクを予測することである.

\section{手法}
以下の手法で研究する.
\begin{enumerate}
\item グループワークで課題に取り組んでもらう
\item グループワーク後に,性格検査\cite{MBTI}と失敗マンダラ\cite{110009915588}に基づいた事象についてのアンケートを行う
\item メンバの性格とアンケートの結果から仮説を立てる
\item 被験者にMBTIの性格検査を行う
\item 仮説に基づき数人のグループを作り,グループワークを行ってもらう
\item 課題提出時にアンケートを行う
\item タイプと事象についての仮説を実証する
\end{enumerate}
\section{想定される成果物}
想定される成果物はプロジェクト開始時から使用できるMBTIを用いたメンバ間のリスク予測リストである.

\section{進捗状況}
千葉工業大学で開講されている講義「プログラミング言語とプログラミング」と「データマイニング入門」,「PM実験・演習」のグループワークで,データを収集する.

現在,「データマイニング入門」の指導教員である矢吹太朗准教授に講義で行うグループワークの提案し,手法と課題について話し合っている.グループ分けは「プログラミング言語とプログラミング」で収集したデータから仮説を立て思索している.

\section{今後の計画}
「データマイニング入門」と「PM実験」のグループワークで,MBTIの性格検査とアンケートを実施する.

そのデータから課題研究などのデータから立てた仮説を実証する.


\nocite{110003745117}
\bibliographystyle{junsrt}
\bibliography{biblio}%「biblio.bib」というファイルが必要.

\end{document}