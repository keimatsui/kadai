%卒論概要テンプレート ver. 4.0

\documentclass[uplatex,twocolumn,dvipdfmx]{jsarticle}
\usepackage[top=22mm,bottom=22mm,left=22mm,right=22mm]{geometry}
\setlength{\columnsep}{11mm}
\usepackage[T1]{fontenc}
\usepackage{txfonts}
\usepackage[expert,deluxe]{otf}
\usepackage[dvipdfmx,hiresbb]{graphicx}
\usepackage[dvipdfmx]{hyperref}
\usepackage{pxjahyper}
\usepackage{secdot}





%タイトルと学生番号,名前だけ編集すること
\title{\vspace{-5mm}\fontsize{14pt}{0pt}\selectfont Word2vecを用いた文章構造の解析手法}
\author{\normalsize プロジェクトマネジメントコース 矢吹研究室 1442069 氏名 須山 武弘}
\date{}
\pagestyle{empty}
\begin{document}
\fontsize{10.5pt}{\baselineskip}\selectfont
\maketitle





%以下が本文
\section{序論}\label{序論}
レポートや論文を書く際には,読みやすく,論理的な文章を書くことが大切である.論理的文章を書くための書き方として,世界で標準的なパラグラフ・ライティング(Paragraph writing)がある\cite{02}.パラグラフ・ライティングは,英語文章の一般的スタイルであり,序論,本論,結論の3部構成となっている.序論でトピックとなる文が示され,本論は序論に続く支持文となり,最後に結論で文章をまとめる.冒頭にトピックとなる文章を示すと伝えたいことが明確になり,速読が可能となったり,内容の理解が深まるなど多数のメリットがある.

言語を定量的に表すツールとして,Word2vecがある.Word2vecは,単語をベクトルへ変換することができるため,文章の話題の方向性を解析し,文章作成の補助ができるのではないかと仮説を立て,本研究に取り組んだ\cite{01}.

\section{目的}
Word2vecを用いて文字列である文章をベクトルへ変換し,定量的に文章構造を解析することでパラグラフ・ライティングができているかを調査する.

\section{手法}

文章が論理的でパラグラフ・ライティングの原則に沿って書かれているか確かめ,実際にWord2vecによる文章解析ができるかを検証する.

矢吹研究室で過去に書かれた文章データや,新聞記事などの文章データを解析対象とし,以下の手順で研究を進めた.
\begin{enumerate}
 \item Mecabを使い,文章の形態素解析をした.
 \item 日本語 Wikipedia エンティティベクトルのコーパスを使用し,Word2vecによって文章をベクトルへ変換した.
 \item データ解析ツールを使用し,主成分分析を行った.
 \item 多数の文章で主成分分析を行い,その結果の比較,考察を行った.
\end{enumerate}

\section{結果}
私が3年次に課題研究の概要として書いた文書の二段落(6文章)を分析した結果が図\ref{分析結果}である.
\begin{figure}[h]
\centering
\includegraphics[width=5cm]{02.pdf}
\caption{word2vecによる分析結果}\label{分析結果}
\end{figure}

\section{考察}
図\ref{分析結果}では,一文章ごと結果がグラフにプロットされている.同段落内の文章は同じ話題でなければならないため,文章のベクトルも同じ方向性である必要がある.このことから,タグAの数値とタグBの数値がそれぞれ集中している事が必要であると仮定した.

本研究での分析結果は,同段落内の文章にもかかわらず,ベクトルは散らばって分布している事がわかることから,話題の方向性が違うと考えられる.解析対象の文章は人間の添削を通し,段落の方向性が同じだと考えられるので,必ずしも解析結果が正しいとは限らないと考えるこもできる.

\section{結論}

今回の結果から,Word2vecを用いて定量的に文章を見ていくことで,個人の主観だけでなく,定量的な文章の書き方を定義できることも期待される.

\bibliographystyle{junsrt}
\bibliography{biblio}%「biblio.bib」というファイルが必要.

\end{document}
