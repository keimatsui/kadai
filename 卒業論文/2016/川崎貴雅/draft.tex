%中間審査概要テンプレート ver. 3.0

\documentclass[uplatex,twocolumn,dvipdfmx]{jsarticle}
\usepackage[top=22mm,bottom=22mm,left=22mm,right=22mm]{geometry}
\setlength{\columnsep}{10mm}
\usepackage[T1]{fontenc}
\usepackage{txfonts}
\usepackage[expert,deluxe]{otf}
\usepackage[dvipdfmx,hiresbb]{graphicx}
\usepackage[dvipdfmx]{hyperref}
\usepackage{pxjahyper}
\usepackage{secdot}





%タイトルと学生番号,名前だけ編集すること
\title{\vspace{-5mm}\fontsize{14pt}{0pt}\selectfont 研究タイトル}
\author{\normalsize プロジェクトマネジメントコース 矢吹研究室 1234567 氏名}
\date{}
\pagestyle{empty}
\begin{document}
\fontsize{10.5pt}{\baselineskip}\selectfont
\maketitle





%以下が本文
\section{背景}

マインクラフトとは主に立体系のブロックで構成された世界でブロックを設置したり,破壊したりすることやアイテムやブロックを組み合わせて新しいアイテムやブロックを製作クラフトが出来る事や生態系や地形に昼夜等があるオープンワールドゲームである.
そんなマインクラフトが米国のNew York Times誌によると2011年以前より,40カ国,7000の学校のクラスでマインクラフトが教育利用されている事が発表されました.この世界的なマインクラフトブームを受け,独自でマインクラフトを教育に導入するケースが増え,教育ゲームとして認知されたことで,2011年より従来のマインクラフトを学校教育用にカスタムした「MinecraftEdu」が,40カ国以上の教育機関に提供されるようになりました.\cite{self}
Minecraftを開発したMojangは米国のMicrosoft社に買収されてからの,2016年の秋には教育特化型の「Minecraft Education Edition」を米国のMicrosoft社が販売しました.
「Minecraft Education Edition」の開発の背景にもなっているマインクラフト実際の教育活用例として3つほどあげると,スコットランドの小学生が都市計画やエンジニアリングを学ぶのに活用される事例が1つ\cite{self2},アメリカのシアトルにある小学校では,算数をマインクラフトで学習する方法が取り入れられているのが2つ目,最後に日本の渋谷区立猿楽小学校の6年生30名を対象にコミュニケーション能力や協同性を育む目的や,プログラミングを用いて建築を自動化することで,コンピュータサイエンスの基礎を学ぶ狙いで試験的に45分授業10コマ分が実施されたものもありました.
このことからプロジェクトマネジメントの学習にもマインクラフトを活用することが可能なのではないかと考えられる.




\section{目的}
本研究では,マインクラフトを教育に活用している事例を調べ,実際にプロジェクトマネジメントの学習に活用することが出来るグループワーク等に使用できる課題を作成し提案する.

\section{手法}
マインクラフトの教育活用例やその際の使用追加Mod(パソコン用の改造データのこと)の調査や具体的に課題に活用できそうなプロジェクトマネジメント的な事例や考え方,法則を挙げて再現できるか試す.試した中でバランスがとれているか等を検証し,修正をくわえてバランスを調整する.実際に学生に課題をやってもらった後でプロジェクトマネジメント的な事例の体験ができたかどうかをアンケートで確認を行い,定量的に評価する.

\section{想定される成果物}
マインクラフトを活用した課題の提案とその製作をすることとその際に出た評価をだす.

\section{進捗状況}
教育活用例のおおまかな調査し,課題に活用するプロジェクトマネジメント的な事例はトレードオフの関係を使うことに決め課題の製作,バランス調整を現在行っている.
またほかにマインクラフトで再現できそうなプロジェクトマネジメント的な事例がないか考案中である.


\section{今後の計画}
現在行っているトレードオフの関係で行っている課題製作,バランス調整の継続とほかのプロジェクトマネジメント的な事例による課題製作の考案,製作し確認をする.制作した課題を評価するのに必要なアンケートの内容の考案を行う.


\bibliographystyle{junsrt}
\bibliography{biblio}%「biblio.bib」というファイルが必要.

\end{document}
