%中間審査概要テンプレート ver. 3.0

\documentclass[uplatex,twocolumn,dvipdfmx]{jsarticle}
\usepackage[top=22mm,bottom=22mm,left=22mm,right=22mm]{geometry}
\setlength{\columnsep}{10mm}
\usepackage[T1]{fontenc}
\usepackage{txfonts}
\usepackage[expert,deluxe]{otf}
\usepackage[dvipdfmx,hiresbb]{graphicx}
\usepackage[dvipdfmx]{hyperref}
\usepackage{pxjahyper}
\usepackage{secdot}





%タイトルと学生番号,名前だけ編集すること
\title{\vspace{-5mm}\fontsize{14pt}{0pt}\selectfont Minecraftを利用するPMトレーニング法の提案と実践}
\author{\normalsize プロジェクトマネジメントコース 矢吹研究室 1442043 川崎貴雅}
\date{}
\pagestyle{empty}
\begin{document}
\fontsize{10.5pt}{\baselineskip}\selectfont
\maketitle





%以下が本文
\section{背景}

MinecraftとはMojangに開発されたオープンワールドゲームである.特徴としては立体系のブロックで構成された世界でブロックを設置したり,アイテムやブロックを組み合わせて新しいアイテムやブロックを製作するクラフトを行うことができる.


Minecraftは日本のプログラミング授業に使用されたことで,教育使用への関心が高まっている.現在ではMinecraftを教育特化型に変更した,Minecraft Education Editionが販売されている.またMinecraft Education Editionのホームページには,年齢や教科に合わせた課題の紹介も行っている\cite{self}.


Minecraftは教育特化型が出る前から教育活用例はある.例えば,獨協大学経済学部 2 年生のゼミにおいて全員が参加するプロジェクトとして,仮想の3D獨協大学が製作されている\cite{ 110009684401}.




\section{目的}
本研究では,Minecraftを教育に活用している事例を調べ,実際にPMのトレーニングに活用することができる課題を製作する.


製作した課題を提案・実際にPM学科の学生に課題を行ってもらい,製作した課題がPMとしての能力向上に活用できているのかを小テストの結果から定量的に判断する.

\section{手法}
Minecraftの教育活用例の調査し,PMの学習に使用できるものを参考に課題を製作する.

製作した課題をPM学科の4年生に体験してもらいコストに対する資材の量のバランスがとれているかをアンケートで確認する.
課題をやった被験者とやってない被験者に製作した課題に関連する10問ほどの小テストを行い,その結果を比較し判断する.

\section{想定される成果物}
PM的な学習をMinecraft内でできる課題の製作をする.実際に製作している課題は,倉庫を資金と時間と品質の3つの要素から一番利益が出る方法を考案してもらい,実行するというものである.
10問の小テストの結果より間違いの傾向などをまとめた一覧を製作する.


\section{進捗状況}
教育活用例を調査し,課題に活用するのはトレードオフの関係を使うことに決め,課題の製作を行っている.具体的には初めに考案した課題をゲームプランナーの新しい教科書やレベルアップのゲームデザイン,おもしろいのゲームデザインなどの文献を読み調整,ゲームデザインの改善を行っている.


ほかにMinecraftで再現できそうな法則等がないか調査をしている.



\section{今後の計画}
資金と時間と品質の3つによりできるトレードオフの関係を用いた課題の製作を終わらせ,出来上がった課題を体験できる状態にする.


製作した課題を評価するのに必要な小テストの内容の考案・製作を行い,課題の体験と評価をしてもらうために協力をお願いする.


課題のバランス調整のためのアンケートを製作・実施・集計を行い,課題の修正を図る.




\bibliographystyle{junsrt}
\bibliography{biblio}%「biblio.bib」というファイルが必要.

\end{document}
