%中間審査概要テンプレート ver. 3.0

\documentclass[uplatex,twocolumn,dvipdfmx]{jsarticle}
\usepackage[top=22mm,bottom=22mm,left=22mm,right=22mm]{geometry}
\setlength{\columnsep}{10mm}
\usepackage[T1]{fontenc}
\usepackage{txfonts}
\usepackage[expert,deluxe]{otf}
\usepackage[dvipdfmx,hiresbb]{graphicx}
\usepackage[dvipdfmx]{hyperref}
\usepackage{pxjahyper}
\usepackage{secdot}





%タイトルと学生番号,名前だけ編集すること
\title{\vspace{-5mm}\fontsize{14pt}{0pt}\selectfont Minecraftを利用するPMトレーニング法の提案と実践}
\author{\normalsize プロジェクトマネジメントコース 矢吹研究室 1442043 川崎貴雅}
\date{}
\pagestyle{empty}
\begin{document}
\fontsize{10.5pt}{\baselineskip}\selectfont
\maketitle





%以下が本文
\section{背景}

Minecraftとは主に立体系のブロックで構成された世界でブロックを設置したり,アイテムやブロックを組み合わせて新しいアイテムやブロックを製作するクラフトが出来るオープンワールドゲームである.
そんなMinecraftが日本でプログラミングの授業使用されたことで,日本でも教育使用への関心が高まっている.現在ではMinecraftを開発したMojangは教育特化型のMinecraft Education Editionを開発し,販売している.またこのMinecraft Education Editionでは公式が,年齢や教科に合わせた課題の紹介も行っている\cite{self}.


Minecraftは教育特化型が出る前からNew York Times誌で40カ国,7000の学校でマインクラフトが教育利用されている事が発表されている.具体的な例を挙げるならば,獨協大学経済学部 2 年生のゼミにおいて全員が参加するプロジェクトとして,仮想の3D獨協大学が制作されている\cite{ 110009684401}.




\section{目的}
本研究では,Minecraftを教育に活用している事例を調べ,実際にプロジェクトマネジメントのトレーニングに活用することができる課題を作成する.


作成した課題を提案・実際にプロジェクトマネジメント学科の生徒に課題を行ってもらい,製作した課題がPMとしての能力向上に活用できているのかを小テストの結果より定量的に判断する.

\section{手法}
Minecraftの教育活用例の調査や具体的に課題に活用できそうなプロジェクトマネジメント的な事例や考え方,法則を挙げて課題を作成する.

作成した課題をプレイしてもらいコストに対する資材の量のバランスがとれているかとプロジェクトマネジメント的トレーニングができてるかをトレーニングをやった生徒とやってない生徒でテストによる確認・修正を繰り返しする.

\section{想定される成果物}
プロジェクトマネジメント的な事例や考え方,法則をMinecraft内で体験することができる課題の作成をする.制作した課題を評価できるアンケートの作成・集計結果を一覧にする.


\section{進捗状況}
教育活用例のおおまかな調査をし,課題に活用するプロジェクトマネジメント的な事例はトレードオフの関係を使うことに決め,課題の製作を行っている.具体的には初めに考案した課題をゲームプランナーの新しい教科書やレベルアップのゲームデザイン,おもしろいのゲームデザインなどの著書を読み調整,ゲームデザインの改善を行っている.


ほかにマインクラフトで再現できそうなプロジェクトマネジメント的な事例がないか考案をしている.



\section{今後の計画}
現在着手している課題を終わらせ,プロジェクトマネジメント学科の生徒がプレイできる状態にする.


制作した課題を評価するのに必要なアンケートの内容の考案しアンケートを作成し課題のプレイと評価をしてもらうために協力をお願いする.


アンケート結果を集計し,課題の質の向上を図る.


\bibliographystyle{junsrt}
\bibliography{biblio}%「biblio.bib」というファイルが必要.

\end{document}
