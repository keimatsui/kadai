%卒論概要テンプレート ver. 4.0

\documentclass[uplatex,twocolumn,dvipdfmx]{jsarticle}
\usepackage[top=22mm,bottom=22mm,left=22mm,right=22mm]{geometry}
\setlength{\columnsep}{11mm}
\usepackage[T1]{fontenc}
\usepackage{txfonts}
\usepackage[expert,deluxe]{otf}
\usepackage[dvipdfmx,hiresbb]{graphicx}
\usepackage[dvipdfmx]{hyperref}
\usepackage{pxjahyper}
\usepackage{secdot}





%タイトルと学生番号,名前だけ編集すること
\title{\vspace{-5mm}\fontsize{14pt}{0pt}\selectfont Twitterのデマ情報拡散によるネットリテラシーの調査}
\author{\normalsize プロジェクトマネジメントコース 矢吹研究室 1442045 川辺明俊}
\date{}
\pagestyle{empty}
\begin{document}
\fontsize{10.5pt}{\baselineskip}\selectfont
\maketitle





%以下が本文
\section{序論}

ネットリテラシー とは,情報ネットワークを正しく利用することができる能力のことである.リテラシーとは,もともとは識字能力のことで,文字や言語に対する能力の意味である.それに「ネット」と付け加えることで,インターネットを使いこなす基本的な能力を指す言葉として「ネットリテラシー」が定着した\cite{a}.このネットリテラシーが不足していると,インターネットを使用する際に,不正サイトでクレジットカード番号を盗まれたり,コンピューターウイルスの感染により,個人や会社などの情報を流出されたり,ネット上にある嘘の情報に騙されてしまう.実際に,熊本地震でライオンが動物園から脱走したと,デマ情報を流しTwitter上で拡散した男性は,偽計業務妨害容疑で逮捕された.

Twitterとは,インターネット上で「ツイート」と呼ばれる140文字以内のメッセージや,画像,動画,URLを投稿できる情報サービスである.日本では,現在(2017年10月)利用者が4,500万人にもなるソーシャル・ネットワーキング・サービス(SNS)と見られている.

Twitterでは日々,大量の情報がツイートされる.もちろんデマ情報などの,ネットリテラシーを問われるような,情報も錯綜している.そこで私は,Twitterのデマ情報に騙され,ネットリテラシーが不足している人の情報を解析し,分析できるのではないかと考えた.

\section{目的}

Twitterのデマ情報を信じ,情報を拡散してしまうのに男女間で差は生まれるのかを調べる.

\section{手法}

研究方法は以下のとおりである.

\begin{enumerate}
\item 機械学習で男女の性別を,判定できるようにwikipediaのプロフィール画像をもとに,Neural Network Consoleを使用し,男女の顔画像を判別するための学習済みモデルを作成する.
\item デマ情報のツイートをリツイートした人のプロフィール画像を集める.
\item 集めたプロフィール画像を,始めに作成した学習済みモデルを使用し,機械学習で性別を判定させる.
\item 男女の数を集計し,どのぐらい差が生じるか調べ,考察する.
\end{enumerate}

\section{結果}

本研究の結果として,Neural Network Consoleを使用し,男性と女性の判別をした場合,80%程度の精度しか出なかった.デマ情報を拡散するTwitterのユーザーと,正確な情報を拡散するユーザーには関わらず,男性と女性では差あるのかは,分がからなかった.

\section{考察}

Twitterユーザーのプロフィール画像は,自分の顔画像を使用していることはとても少なく,データが不足してしまうことが分かった.そして,写真から男性と女性を学習させた学習済みモデルで,画像認識を使用した場合,機械学習の性能が低いため,性別を判別するのは,難しいと考えた.

\section{結論}

本研究ではネットリテラシーの不足している人には,どのような特徴があるのかを調べた.だが,Twitterでは,プロフィール画像で自分の写真を使用しているのは極わずかであり,機械学習を使用した画像認識では,性別を判別するのは難しいことだと分かった.
\bibliographystyle{junsrt}
\bibliography{biblio}%「biblio.bib」というファイルが必要.

\end{document}
