%卒論概要テンプレート ver. 3.0

\documentclass[uplatex,twocolumn,dvipdfmx]{jsarticle}
\usepackage[top=22mm,bottom=22mm,left=22mm,right=22mm]{geometry}
\setlength{\columnsep}{10mm}
\usepackage[T1]{fontenc}
\usepackage{txfonts}
\usepackage[expert,deluxe]{otf}
\usepackage[dvipdfmx,hiresbb]{graphicx}
\usepackage[dvipdfmx]{hyperref}
\usepackage{pxjahyper}
\usepackage{secdot}





%タイトルと学生番号,名前だけ編集すること
\title{\vspace{-5mm}\fontsize{14pt}{0pt}\selectfont SNSにおいてフェイクニュースを拡散するユーザの特徴抽出}
\author{\normalsize プロジェクトマネジメントコース 矢吹研究室 1442014 岩橋瑠伊}
\date{}
\pagestyle{empty}
\begin{document}
\fontsize{10.5pt}{\baselineskip}\selectfont
\maketitle





%以下が本文
\section{序論}
近年FacebookやTwitterを始めとしたマイクロブログがスマートフォンなどの普及と共に急激に普及している.例えばTwitterは2011年3月11日に発生した東日本大震災時に,携帯電話がつながらない状況下での有用な連絡手段として活躍した.しかし,その有用性はデマや誤情報も大量に拡散させる手助けとなりえる.例えば東日本大震災時には数十種類のデマや誤情報が情報として拡散されてしまい,日本中を混乱させた.震災時のように連絡手段が限られた状況はこれからも発生する可能性は十分にあり,対策が必要である\cite{dema1}.

本研究では,デマが拡散されることを防ぐためにデマツイートをリツイートしているユーザの特徴抽出を行う.

\section{目的}
デマが拡散されることを防ぐために,デマツイートをリツイートしているユーザの特徴抽出を行う.

\section{手法}
デマツイートをリツイートするユーザーとそれ以外のユーザーの違いを見つけ,その違いが偶然生じたものではないことを示すために以下の手法で研究する.
\begin{enumerate}
\item ユーザIDを乱数で指定し,日本人ユーザ50人をランダムサンプリングする.
\item TwitterAPIを用いてデマツイートをリツイートしたユーザ50人を取得する.
\item TwitterAPIを用いて集めた各ユーザの最新100ツイートに含まれるリツイートの数を調べる.
\item 日本人ユーザ50人とデマツイートをリツイートしたユーザ50人の直近100ツイートに含まれるリツイートの数の平均の差が,偶然的な誤差の範囲にあるものかどうかを判断する為に2標本t検定を行う.
\end{enumerate}

\section{結果}
ランダムサンプリングした日本人ユーザ50人の直近100ツイート中の平均リツイート数は20.04人,デマツイート1,2,3,4をリツイートしたユーザ50人の直近100ツイート中の平均リツイート数は,1が56.68人,2が62.64人,3が58.46人,4が57.92人となった.

デマツイート1,2,3,4と日本人ユーザのそれぞれ4組は対応がないデータなので,F検定を行い分散が等しいか等しくないかを確かめる.等分散の場合の2標本t検定と不等分散の場合のt検定をF検定の結果に沿って行った結果,全ての組み合わせで有意差が確認できた(有意水準は5%).

\section{考察}
デマ拡散ユーザに共通する特徴として,リツイート数に着目しランダムサンプリングしたユーザーとデマツイートをリツイートしたユーザーで直近100リツイート内のリツイート数を調査した結果,大きく数値が異なった.この結果からデマを拡散するようなユーザーはリツイート機能を多用する傾向にあり,ツイート内容の真偽を確かめる前にリツイートをし,デマ拡散者の一員となっていると考えられる.

\section{結論}
本研究の結果,デマツイートを拡散するユーザーの特徴として,ツイートに占めるリツイートの割合が高いことが確認できた.このような知識を活用することで,Twitterを閲覧する際に,デマツイートを真に受けて拡散してしまうリスクを下げられることが期待できる.

\bibliographystyle{junsrt}
\bibliography{biblio}%「biblio.bib」というファイルが必要.

\end{document}
