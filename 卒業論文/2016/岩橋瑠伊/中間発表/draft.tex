%中間審査概要テンプレート ver. 3.0

\documentclass[uplatex,twocolumn,dvipdfmx]{jsarticle}
\usepackage[top=22mm,bottom=22mm,left=22mm,right=22mm]{geometry}
\setlength{\columnsep}{10mm}
\usepackage[T1]{fontenc}
\usepackage{txfonts}
\usepackage[expert,deluxe]{otf}
\usepackage[dvipdfmx,hiresbb]{graphicx}
\usepackage[dvipdfmx]{hyperref}
\usepackage{pxjahyper}
\usepackage{secdot}





%タイトルと学生番号,名前だけ編集すること
\title{\vspace{-5mm}\fontsize{14pt}{0pt}\selectfont デマツイート拡散ユーザーのリツイート頻度分析}
\author{\normalsize プロジェクトマネジメントコース 矢吹研究室 1442014 岩橋瑠伊}
\date{}
\pagestyle{empty}
\begin{document}
\fontsize{10.5pt}{\baselineskip}\selectfont
\maketitle





%以下が本文
\section{背景}
Twitterは2006年7月15日に開設された「ツイート」と称される140文字以内の短文を投稿し共有するウェブ上の情報サービスである.国内月間アクティブユーザー数は4000万人である\cite{twitter}.

Twitterでは日常的なことからニュースまで,様々な発言が飛び交っている.その中に,デマツイートと呼ばれるものが含まれている.デマツイートの例として,熊本地震の発生直後に「熊本で動物園のライオンが逃げた」などというデマツイートがある.このデマツイートを流した神奈川県に住む20歳の会社員の男が,2016年7月20日業務妨害の容疑で逮捕された\cite{dema}.

私はデマツイートがリツイートされる原因として,デマをデマと見抜けないユーザー,面白半分でリツイートしているユーザーの2種類がいると考えた.この2種類のユーザーと,それ以外のユーザーにはリツイートする頻度の差があるのではないかと考えた.

\section{目的}
デマツイートをリツイートするユーザーと,それ以外のユーザーのリツイートの頻度を調べる.リツイートの頻度でデマツイートをリツイートするユーザーなのかを判別できるようにする.

\section{手法}
以下の手法で研究する.
\begin{itemize}
\item TwitterAPIを用いて日本人ユーザー50人をランダムサンプリングする.
\item TwitterAPIを用いてデマツイートをリツイートしたユーザー50人を取得する.
\item TwitterAPIを用いて集めたユーザーの最新100ツイートに含まれるリツイートの数を調べて,平均などを計算する.
\item 2標本T検定を行い,平均に有意差があるのかを確かめる.
\end{itemize}

\section{想定される成果物}
デマツイートをリツイートしたユーザー群の方が,最新100ツイートに含まれるリツイートの平均値が高くなる.2標本T検定の結果,日本人ユーザー群とデマツイートをリツイートしたユーザー群の最新100ツイートに含まれるリツイートの平均値に有意差が確認できる.

\section{進捗状況}
TwitterAPIを用いて日本人ユーザー50人をランダムサンプリングした.また,3つのデマツイートからそれぞれ50人のリツイートユーザーを取得した.日本人ユーザー50人の平均リツイート数は20.04人となり,デマツイート1の50人の平均リツイート数は56.68人,デマツイート2の50人の平均リツイート数は62.64人,デマツイート3の50人の平均リツイート数は58.46人となった.デマツイート1と日本人ユーザーの2標本T検定を行った結果,有意差が確認できた.デマツイート2と日本人ユーザーの2標本T検定を行った結果,有意差が確認できた.デマツイート3と日本人ユーザーの2標本T検定を行った結果,有意差が確認できた.

\section{今後の計画}
データ量が少ないのでデータ量をさらに増やすために新しいデマツイートを発見する.

\bibliographystyle{junsrt}
\bibliography{biblio}%「biblio.bib」というファイルが必要.

\end{document}
