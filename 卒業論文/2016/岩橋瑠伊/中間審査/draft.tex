%中間審査概要テンプレート ver. 3.0

\documentclass[uplatex,twocolumn,dvipdfmx]{jsarticle}
\usepackage[top=22mm,bottom=22mm,left=22mm,right=22mm]{geometry}
\setlength{\columnsep}{10mm}
\usepackage[T1]{fontenc}
\usepackage{txfonts}
\usepackage[expert,deluxe]{otf}
\usepackage[dvipdfmx,hiresbb]{graphicx}
\usepackage[dvipdfmx]{hyperref}
\usepackage{pxjahyper}
\usepackage{secdot}





%タイトルと学生番号,名前だけ編集すること
\title{\vspace{-5mm}\fontsize{14pt}{0pt}\selectfont SNSにおいてデマを拡散するユーザの特徴抽出}
\author{\normalsize プロジェクトマネジメントコース 矢吹研究室 1442014 岩橋瑠伊}
\date{}
\pagestyle{empty}
\begin{document}
\fontsize{10.5pt}{\baselineskip}\selectfont
\maketitle





%以下が本文
\section{背景}
近年FacebookやTwitterを始めとしたマイクロブログがスマートフォンなどの普及と共に急激に普及している.特にTwitterは手軽にリアルタイムな情報を多くのユーザに伝搬出来るため社会に大きな影響を与えている.例えばTwitterは2011年3月11日に発生した東日本大震災時に,携帯電話がつながらない状況下での有用な連絡手段として活躍した.しかし,その有用性はデマや誤情報も大量に拡散させる手助けとなりえる.例えば東日本大震災時には数十種類のデマや誤情報が情報として拡散されてしまい,日本中を混乱させた.震災時のように連絡手段が限られた状況はこれからも発生する可能性は十分にあり,対策が必要である\cite{dema}.

本研究では,デマが拡散されることを防ぐためにデマツイートをリツイートしているユーザーの特徴抽出を行う.デマツイートがリツイートされる原因として,デマをデマと見抜けないユーザー,面白半分でリツイートしているユーザーの2種類がいると考えた.この2種類のユーザーと,それ以外のユーザーにはTwitterの使い方に違いがあるのではないかと考えた.

\section{目的}
デマツイートをリツイートするユーザーと,それ以外のユーザーのTwitterの使い方に違いを見つける.違いからデマツイートをリツイートするユーザーなのかを判別できるようにする.

\section{手法}
デマツイートをリツイートするユーザーとそれ以外のユーザーの違いを見つけ,その違いが偶然生じたものではないことを示すために以下の手法で研究する.
\begin{enumerate}
\item TwitterAPIを用いて日本人ユーザー50人をランダムサンプリングする.
\item TwitterAPIを用いてデマツイートをリツイートしたユーザー50人を取得する.
\item TwitterAPIを用いて集めた各ユーザーの最新100ツイートに含まれるリツイートの数を調べて,平均を計算する.
\item 日本人ユーザー50人とデマツイートをリツイートしたユーザー50人の最新100ツイートに含まれるリツイートの数の平均の差が,偶然的な誤差の範囲にあるものかどうかを判断する為に2標本T検定を行う.
\end{enumerate}

\section{想定される成果物}
デマツイートをリツイートするユーザーと,それ以外のユーザーのTwitterの使い方に違いが見つかる.例として2標本T検定の結果,日本人ユーザー群とデマツイートをリツイートしたユーザー群の最新100ツイートに含まれるリツイートの平均値に有意差が確認できる.

\section{進捗状況}
TwitterAPIを用いて日本人ユーザー50人をランダムサンプリングした.3つのデマツイートからそれぞれ50人のリツイートユーザーを取得した.日本人ユーザー50人の平均リツイート数は20.04人,デマツイート1は56.68人,デマツイート2は62.64人,デマツイート3は58.46人となった.デマツイート1と日本人ユーザー,デマツイート2と日本人ユーザー,デマツイート3と日本人ユーザーのそれぞれ3組で2標本T検定を行った結果,全て有意差が確認できた.

\section{今後の計画}
データ量が少ないのでデータ量をさらに増やすために新しいデマツイートを発見し分析を行う.

\bibliographystyle{junsrt}
\bibliography{biblio}%「biblio.bib」というファイルが必要.

\end{document}
