%卒論概要テンプレート ver. 4.0

\documentclass[uplatex,twocolumn,dvipdfmx]{jsarticle}
\usepackage[top=22mm,bottom=22mm,left=22mm,right=22mm]{geometry}
\setlength{\columnsep}{11mm}
\usepackage[T1]{fontenc}
\usepackage{txfonts}
\usepackage[expert,deluxe]{otf}
\usepackage[dvipdfmx,hiresbb]{graphicx}
\usepackage[dvipdfmx]{hyperref}
\usepackage{pxjahyper}
\usepackage{secdot}





%タイトルと学生番号,名前だけ編集すること
\title{\vspace{-5mm}\fontsize{14pt}{0pt}\selectfont 文書自動添削システムによる学生の文書改善履歴の調査}
\author{\normalsize プロジェクトマネジメントコース 矢吹研究室 1442031   小山隆太郎}
\date{}
\pagestyle{empty}
\begin{document}
\fontsize{10.5pt}{\baselineskip}\selectfont
\maketitle





%以下が本文
\section{序論}
学生が行う研究では,研究だけではなく文書を作成する時間が長い.卒業論文は文量が多く,執筆形式も指摘される.大量の文書を人の目で添削を行うことには限界があり,かかる労力は大きい.

また,文書を自分以外が読んでもわかりやすく書く必要があり,文が長いほど理解が難しくなってしまう場合や,口語が混じり,文書の質が落ちてしまうことがある.

そこで,継続的インテグレーション\cite{a}を用いることで,文書添削を自動化できないか考えた.継続的インテグレーションとは,プログラム全体を常に統合し,動作する状態を指している.

文書自動添削ツールで活用されているRedPenを執筆環境に導入することで,文書の質が向上すると考えた.継続的インテグレーションとRedPenを組み合わせ,文書添削を自動化するツールを構築する.

\section{目的}
RedPenが提供する添削機能は,利用する組織のルールに対応できるように設定が柔軟に行える仕様になっている.
RedPenの文書添削機能を確立し,学生が書く文書の質の向上と,作成時間の短縮を図ることを目的とする.

\section{手法}
本研究の手法について以下に記述する.
\begin{enumerate}
\item 文書自動添削ツールの添削機能を作成する.
\item GitHubにアップロードした文書の添削を自動化する.
\item 作成した添削機能を用いて,文中のミス数の推移を記録する.
\end{enumerate}

\section{結果}
矢吹研究室に所属する3年生が書いた課題研究の概要文の添削を行った際の,エラー数の推移は図\ref{conf}のとおりである.各折れ線が文章1つのミス数の推移を表している.ミス数が減った文書の修正は以下のように行われた.

\begin{enumerate}
\item 「の」,「が」等の接続詞の多用や,同一単語の複数回利用を抑えたことで,文長を短くした.「丁度」,「ちょうど」といった同じ言葉や,数値,アルファベットの表記を統一し,文書を修正した.
\item 「これ」,「あれら」等の指示語の利用を抑えた.「感じる」,「思う」といった感嘆符を使用している文書は,断定系に修正された.
\end{enumerate}

\begin{figure}[htb]
\centering
\includegraphics[width=5cm,clip]{redpen.pdf}
\caption{添削ツールを使用した文書の添削項数の推移}\label{conf}
\end{figure}

\section{考察}
文書自動添削ツールの添削機能を作成し,執筆に使用したところ,専門用語を用いて解説する文書を多く見ることができた.「感じる」,「考えられる」等の感嘆符の利用を避け,「考える」と「である調」を使用したことで,研究内容を詳細に解説することに役立った.

\section{結論}
文書添削ツールを使用し,文書添削をしたことで,文中のミスを削減できた.文書添削ツールを利用することで,文書作成の効率化が実現できるか調査することが今後の課題である.




\bibliographystyle{junsrt}
\bibliography{biblio}%「biblio.bib」というファイルが必要.

\end{document}
