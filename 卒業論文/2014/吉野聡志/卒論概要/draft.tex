%卒業論文概要テンプレート ver. 1.0

\documentclass[uplatex,twocolumn,dvipdfmx]{jsarticle}
\usepackage[top=22mm,bottom=22mm,left=20mm,right=20mm]{geometry}
\setlength{\columnsep}{15mm}
\usepackage[T1]{fontenc}
\usepackage{txfonts}
\usepackage{wrapfig}
\usepackage[expert,deluxe]{otf}
\usepackage[dvipdfmx,hiresbb]{graphicx}
\usepackage[dvipdfmx]{hyperref}
\usepackage{pxjahyper}
\usepackage{secdot}

\makeatletter
\renewcommand{\section}{%
  \@startsection{section}{1}{\z@}%
  {0.6\Cvs}{0.4\Cvs}%
  {\normalfont\normalsize\raggedright}}
\renewcommand{\subsection}{\@startsection{subsection}{2}{\z@}%
  {\z@}{\z@}%
  {\normalfont\normalsize}}
\renewcommand{\subsubsection}{\@startsection{subsubsection}{3}{\z@}%
  {\z@}{\z@}%
  {\normalfont\normalsize}}
\makeatother
%ここから上を編集する必要はない.





%タイトルと学生番号,名前だけ編集すること
\title{\vspace{-5mm}\fontsize{14pt}{0pt}\selectfont SNS経由で入手される情報のユーザ間差異の可視化}
\author{\normalsize プロジェクトマネジメントコース・ソフトウェア開発管理グループ 矢吹研究室 1242131 吉野聡志}
\date{}
\pagestyle{empty}
\begin{document}
\fontsize{10.5pt}{\baselineskip}\selectfont
\maketitle





%以下が本文
\section{序論}
世界的に人気のあるSNS(Social Networking Service)のひとつとしてTwitterが存在する.2015年6月30日現在,月間アクティブユーザは3億1600万人である\cite{twitterinc}.

Twitterのユーザは「つぶやき」と呼ばれる140字以内の短い記事を書き込むことが可能で,また,他の不特定多数のユーザがそれを閲覧することができる.さらに,つぶやきに返信をすることでコミュニケーションが生まれる.他のユーザのつぶやきを追跡することを「フォローする」という.タイムラインと呼ばれる画面には,自分とフォローしたユーザのつぶやきが同一列上にリアルタイムで,時系列に沿って表示される.自分が閲覧していない間もタイムラインは常に流れていき,フォロー数が多いとつぶやきを見逃す可能性が出てくるため,ユーザ同士の密接でない,ゆるいつながりがTwitter上で生まれるとされている\cite{ascii}.

SNSの中でもアクティブユーザ数が非常に多く,利用スタイルも多数あるTwitterに対し,ユーザである人々が持っているニーズが何であるかが分かれば,他のSNSとの差別化を図りやすくなる.これにより効率的なマーケティングの手法をTwitter社や,Twitter上に広告を打ち出す企業に提案できるのではないかと考えられる.

\section{目的}
数名のTwitterタイムラインを取得し,各人のタイムライン上に並ぶ単語や,単語同士の結びつきの強さを可視化する.その結果からつぶやきの性質を分析し,各人の嗜好や関心事項と合致するものを見つけ出し,Twitterへのニーズを読み取ることが目的である.

\section{手法}
自らが利用するものと,矢吹研究室に所属する3年生5人のTwitterアカウントのタイムラインを,TwitterのStreaming APIを用いて1日分取得する.

次に,User Localテキストマイニング(http://textmining.userlocal.jp/)を用いて単語の出現頻度をカウントしたり,単語同士の結びつきの強さを可視化する「共起ネットワーク」を表示させ,各人のTwitterタイムライン上にあるつぶやきの性質を可視化する.

本研究を行う過程でUser Localテキストマイニングでは不都合な場面が生じたため,形態素解析システムのMeCabを用いてUser Localテキストマイニングの分析で行われている処理の再現も試みた.

\section{結果}
User Localテキストマイニングによる分析の結果,各人の趣味趣向に関連する単語が多く出現した.それに対し,MeCabを用いて分析したところ,どの人のタイムラインにおいてもより一般的な単語が多く出現し,個人間での違いはあまり大きく現れなかった.

\section{考察}
両者における結果の違いは,User Localテキストマイニングを運営する株式会社ユーザーローカルはビッグデータ分析を主とする会社であるのに対し,デフォルトの状態におけるMeCabは新語や流行語,専門的な単語に弱い傾向があるために現れたものと思われる.

\section{結論}
MeCabを用いてTwitterタイムラインを分析し,ニーズを読み取るには,まず新語等の辞書をMeCabに追加し,可能な限りトレンドに追いついていく必要があるといえる.

\bibliographystyle{junsrt}
\bibliography{biblio}%「biblio.bib」というファイルが必要.

\end{document}