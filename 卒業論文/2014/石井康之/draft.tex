%卒論中間審査用研究概要テンプレート ver. 1.0

\documentclass[uplatex,twocolumn]{jsarticle}
\usepackage[top=22mm,bottom=22mm,left=20mm,right=20mm]{geometry}
\setlength{\columnsep}{15mm}
\usepackage[T1]{fontenc}
\usepackage{txfonts}
\usepackage{wrapfig}
\usepackage[expert,deluxe]{otf}
\usepackage[dvipdfmx,hiresbb]{graphicx}
\usepackage[dvipdfmx]{hyperref}
\usepackage{pxjahyper}
\usepackage{secdot}

\makeatletter
\renewcommand{\section}{\@startsection{section}{1}{\z@}{0pt}{0.4\Cvs}{\normalfont\raggedright}}
\renewcommand{\subsection}{\@startsection{subsection}{2}{\z@}{\z@}{\z@}{\normalfont}}
\renewcommand{\subsubsection}{\@startsection{subsubsection}{3}{\z@}{\z@}{\z@}{\normalfont}}
\makeatother
%ここから上を編集する必要はない.





%タイトルと学生番号,名前だけ編集すること
\title{\vspace{-5mm}\fontsize{14pt}{0pt}\selectfont Wikipediaによるオープンなプロジェクトの品質管理のあり方について}
\author{\normalsize プロジェクトマネジメントコース・ソフトウェア開発管理グループ 矢吹研究室 1242005 石井康之}
\date{}
\pagestyle{empty}
\begin{document}
\fontsize{10.5pt}{\baselineskip}\selectfont
\maketitle





%以下が本文
\section{研究の背景}
\makeatletter\renewcommand{\section}{\@startsection{section}{1}{\z@}{0.6\Cvs}{0.4\Cvs}{\normalfont\raggedright}}\makeatother%余白の調整(気にしなくていい)


 Wikipediaは多くのボランティアにより,始まってから10年足らずの間に,大きな成長を見せたオンライン百科事典プロジェクトである.Wikipediaの規模は膨大だ.文字列は10億語を超え,ブリタニカとエンカルタの合計の何倍にもなる.英語版のWikipediaは250万以上の記事があるが,全体の3分の1にも満たない.Wikipediaはさまざまな言語が参加するグローバルなプロジェクトでもある\cite{wikirevo}.
このオープンなプロジェクトの百科事典は制限なく、誰でも自由に使用でき編集することもできる.
情報の「フリー」化には,思わぬ利点がある.フリーだからこそ,ボランティアの人々は気軽に参加でき,特定の企業や個人の金儲けに力を貸していると感じることなく,時間と労力を注ぐことができるのである.
だが、編集者のすべてが善意を持っているとは限らず、中には悪意のある編集をするものもいる.悪意のある行為をする人とわかっていてもWikipediaでは規制などをしたりはしない.それにもかかわらず,我々がWikipediaを使用している際はそのような記事は見かけず,信頼のおける品質が保たれている.記事は完成・確定されることはないので,新しい情報にいつでも改変することができる.
Wikipediaは,顔や素性が分からない人たちと信頼し合い,共同作業で作り出される史上最大のプロジェクトである.
本研究では,Wikipediaの全編集データをマイニングすることによって,Wikipediaの品質が保たれている成功理由を見つけ出す.




\section{目的}
 Wikipediaを一つのプロジェクトとみなし,このオンライン百科事典で品質管理がどのように行われているか調査する.この調査により,オープンな共同作業プロジェクトにおける,品質管理マネジメントの在り方についての知見を得たい.




\section{研究方法}
 Wikipedia日本語版の編集履歴まで含んだ巨大なファイルをダウンロードし,ローカルでデータマイニングを行い,どのような品質管理が行われているか調査する.また,オープンなプロジェクトにおける品質管理マネジメントの在り方を提案する.



\section{成果物のイメージ}

 Wikipediaで行われている差し戻しの編集回数を調査し,その傾向をグラフにまとめる.さらに,オープンなプロジェクトでの品質マネジメントの知見を得る.

\section{進捗状況}

 Googleが提供しているBigQueryというビッグデータを扱うことができるサイトで,Wikipediaのデータを提供されているので,差し戻しデータを抽出し,ランキング化することができた.
 BigQueryが提供しているデータは、英語版のみであり,多言語版を解析するには別の解析方法をとる必要がある\cite{bigquery}.


\section{今後の計画}

 Wikipediaが提供しているWikimediaというサイトから日本語版の全履歴データをダウンロードする.ローカルで解析するためにパソコンの環境も整える必要がある.Wikipediaの全履歴データを解析し,オープンなプロジェクトをする際の品質管理のあり方について調査し提案する.また,既に世に出ているWikipediaマイニングについて調査を行う.



\bibliographystyle{junsrt}
\bibliography{biblio}%「biblio.bib」というファイルが必要.

\end{document}