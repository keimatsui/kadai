%卒業論文概要テンプレート ver. 1.0

\documentclass[uplatex,twocolumn,dvipdfmx]{jsarticle}
\usepackage[top=22mm,bottom=22mm,left=20mm,right=20mm]{geometry}
\setlength{\columnsep}{15mm}
\usepackage[T1]{fontenc}
\usepackage{txfonts}
\usepackage{wrapfig}
\usepackage[expert,deluxe]{otf}
\usepackage[dvipdfmx,hiresbb]{graphicx}
\usepackage[dvipdfmx]{hyperref}
\usepackage{pxjahyper}
\usepackage{secdot}

\makeatletter
\renewcommand{\section}{%
  \@startsection{section}{1}{\z@}%
  {0.6\Cvs}{0.4\Cvs}%
  {\normalfont\normalsize\raggedright}}
\renewcommand{\subsection}{\@startsection{subsection}{2}{\z@}%
  {\z@}{\z@}%
  {\normalfont\normalsize}}
\renewcommand{\subsubsection}{\@startsection{subsubsection}{3}{\z@}%
  {\z@}{\z@}%
  {\normalfont\normalsize}}
\makeatother
%ここから上を編集する必要はない.





%タイトルと学生番号,名前だけ編集すること
\title{\vspace{-5mm}\fontsize{14pt}{0pt}\selectfont Twitterにおけるユーザープロフィールと拡散力の関係分析}
\author{\normalsize プロジェクトマネジメントコース・ソフトウェア開発管理グループ 矢吹研究室 1142016 井上 乃祐}
\date{}
\pagestyle{empty}
\begin{document}
\fontsize{10.5pt}{\baselineskip}\selectfont
\maketitle





%以下が本文
\section{序論}
Twitterは2006年に開始したサービスで,コミュニケーションツールのひとつとして利用されているソーシャル・ネットワーキング・サービスである.国内ユーザーは2014年6月では1980万人,2015年5月では2390万人であり,月間アクティブ率は70.2%である\cite{twitter}.

Twitterはツイートと呼ばれる140字以内の短い文字列を投稿するサービスである.自分以外のユーザーのツイートを読むためには,そのユーザーのページにアクセスする方法以外に,そのユーザーをフォローすることでツイートを読むことができる.
その他の機能にリツイート呼ばれるものがあり,それは他の人のツイートを再びツイートするというものである.自分の画面上に流れてきたツイートをリツイートすると,自分のフォロワーの画面にも流れる.同じように,自分がフォローしているユーザーがリツイートすれば,自分のタイムライン上にリツイートが流れてくる.リツイートされるツイートには,ツイート内容という情報以外にアイコンや,ユーザーのIDなどの本質以外の情報も含まれる.

Twitterを利用していると,フォローしているユーザーからリツイートが流れてくることがある.その流れてきたリツイートを見てみると,似たような内容でもリツイートされた回数に違いがあることに気づいた.そこで私はプロフィールのアイコンがリツイート数に影響があるのではないかと考えた.







\section{目的}

Twitterのアイコンが拡散力に影響があるかを調べ,情報の本質でないアイコン部分が本質に与える影響を調べる.

\section{手法}





APIを使用し,リツイートされたツイートをランダムで集める.次に1度だけリツイートされたアイコンと,2000回以上リツイートされたアイコンのデータをそれぞれ100個ずつ,計200個集める.次にアイコンを24種類のタグ(若い男女,中年の男女,年配の男女,ロゴ・マーク等)で分類し,見比べる.




%\bigskip


\section{結果}

1度だけリツイートされたアイコンは女性のアイコンが多く,その中でも若い女性のタグをつけたものは17個含まれていた.2000回以上リツイートされたアイコンでは男性のアイコンが多く100個のデータの中の39個のアイコンが男性のタグをつけたものだった.その中でも若い男性のタグをつけたものが21個と半数以上を占めていた.




\section{考察}

若い女性のタグをつけたアイコンがリツイート数が少ない理由として,日記のようなツイートが多く,他人があまり興味を持ちにくいツイートをするからではないかと考えた.次に若い男性のタグをつけたアイコンがリツイート数が多い理由として,若い男性は流行,トレンドに早く追いつき話題になるようなツイートが多く,他人が興味を持ちやすいツイートをするからではないかと考えた.

\section{結論}
本研究で,男性アイコンのほうがリツイート数が高くなるという結果が出た.また本研究では,アイコンを分類する上でタグ付けをする作業を,自分の目で見て手作業による分類方法を使用していたため,アイコンの分類に多少の主観が入ってしまっている.画像認識サービスによる分類使用することによって客観的にアイコンを分類し調査することが今後の課題と言える.


\bibliographystyle{junsrt}
\bibliography{biblio}%「biblio.bib」というファイルが必要.

\end{document}
