%卒業論文概要テンプレート ver. 1.0

\documentclass[uplatex,twocolumn,dvipdfmx]{jsarticle}
\usepackage[top=22mm,bottom=22mm,left=20mm,right=20mm]{geometry}
\setlength{\columnsep}{15mm}
\usepackage[T1]{fontenc}
\usepackage{txfonts}
\usepackage{wrapfig}
\usepackage[expert,deluxe]{otf}
\usepackage[dvipdfmx,hiresbb]{graphicx}
\usepackage[dvipdfmx]{hyperref}
\usepackage{pxjahyper}
\usepackage{secdot}

\makeatletter
\renewcommand{\section}{%
  \@startsection{section}{1}{\z@}%
  {0.6\Cvs}{0.4\Cvs}%
  {\normalfont\normalsize\raggedright}}
\renewcommand{\subsection}{\@startsection{subsection}{2}{\z@}%
  {\z@}{\z@}%
  {\normalfont\normalsize}}
\renewcommand{\subsubsection}{\@startsection{subsubsection}{3}{\z@}%
  {\z@}{\z@}%
  {\normalfont\normalsize}}
\makeatother
%ここから上を編集する必要はない.





%タイトルと学生番号,名前だけ編集すること
\title{\vspace{-5mm}\fontsize{14pt}{0pt}\selectfont クラウドファンディングにおける成功の判別分析}
\author{\normalsize プロジェクトマネジメントコース・ソフトウェア開発管理グループ 矢吹研究室 1242105 三浦泰介}
\date{}
\pagestyle{empty}
\begin{document}
\fontsize{10.5pt}{\baselineskip}\selectfont
\maketitle





%以下が本文
\section{序論}
クラウドファンディングが世界中で利用されている\cite{kaihatu}.クラウドファンディングとは,プロジェクトの活動資金を,インターネットを利用し,不特定多数の支援者から募集する資金調達の手法である.

クラウドファンディングは一般的に資金提供者に対するリターンの形態によって,金銭的リターンのない「寄付型」と金銭的リターンのある「投資型」,権利や物品を購入することで支援する「購入型」に分けられる.

日本におけるクラウドファンディングは,2014年に金融商品取引法が改正されるまで\cite{kisei}は,見返りを得ない寄付型,購入型に限られていた.購入型はリターンがあるためリターンを目当てに多くの出資者が集まる傾向があり,購入型は日本で一番市場が大きいクラウドファンディングの形態である.


\section{目的}
プロジェクトにおいて資金集めは重要で,資金集めができなければプロジェクトは破綻してしまう.
クラウドファンディングの成否に関わる要因を見いだすことでプロジェクトの資金集めでの成功率を上げることを目標とする.

\section{手法}
クラウドファンディングの成功要因を以下の手順で分析する.

まず,日本の大手クラウドファンディングサービスであるMakuake とREADYFOR を定期的(1日1回)チェックし,そこで実施されているクラウドファンディングについての情報を収集する.収集する情報は,目標金額と支援コース数,支援最低金額,支援最高金額,資金提供者が得る物,動画の有無である.最終的に資金調達に成功したかどうかも記録する.

次に,資金調達の成否の決定木を作成する.決定木の目的変数は資金調達の成否,説明変数は上述の収集情報(目標金額等)である.

最後に,作成された決定木をもとに,クラウドファンディングの成功要因を考察する.


\section{結果}
100 個のプロジェクトのデータを収集し,決定木分析を行った.使用した要因は「手法」で述べたとおりである.作成した決定木は本文に載せる.この決定木は,全体の83\% を再現している.

\section{考察}
決定木から,支援最低金額が3,744 円から35,990円で目標金額が375,000 円以上のプロジェクトの成功率が最も高くなることがわかった.これは支援最低金額でリターンがもらえるプロジェクトが成功しやすいためだと思われる.

目標金額が375,000 円以上の場合の成功率が高くなることは,この目標金額が今回調査したプロジェクトの平均目標金額(160,6724 円)よりかな
り低く,そのことが資金調達のしやすさに影響しているためだと思われる.最低金額と目標金額が決定木に頻出することからも,この2 つの設定額が重要であることがわかる.

\section{結論}

クラウドファンディングにおいて,プロジェクトの内容以外の成功要因を,決定木分析によって調査した.決定木の予測が実際の結果とよく合っていることから,そのような要因が存在することが示唆される.


\bibliographystyle{junsrt}
\bibliography{biblio}%「biblio.bib」というファイルが必要.

\end{document}
