%卒論中間審査用研究概要テンプレート ver. 1.0

\documentclass[uplatex,twocolumn]{jsarticle}
\usepackage[top=22mm,bottom=22mm,left=20mm,right=20mm]{geometry}
\setlength{\columnsep}{15mm}
\usepackage[T1]{fontenc}
\usepackage{txfonts}
\usepackage{wrapfig}
\usepackage[expert,deluxe]{otf}
\usepackage[dvipdfmx,hiresbb]{graphicx}
\usepackage[dvipdfmx]{hyperref}
\usepackage{pxjahyper}
\usepackage{secdot}

\makeatletter
\renewcommand{\section}{%
\@startsection{section}{1}{\z@}%
{0.6\Cvs}{0.4\Cvs}%
{\normalfont\normalsize\raggedright}}
\renewcommand{\subsection}{\@startsection{subsection}{2}{\z@}%
{\z@}{\z@}%
{\normalfont\normalsize}}
\renewcommand{\subsubsection}{\@startsection{subsubsection}{3}{\z@}%
{\z@}{\z@}%
{\normalfont\normalsize}} 
\makeatother
%ここから上を編集する必要はない.





%タイトルと学生番号,名前だけ編集すること
\title{\vspace{-5mm}\fontsize{14pt}{0pt}\selectfont 大学入試センター試験数学1・Aを用いた数式処理システムの性能評価}
\author{\normalsize プロジェクトマネジメントコース・ソフトウェア開発管理グループ 矢吹研究室 1242116 森谷慧士}
\date{}
\pagestyle{empty}
\begin{document}
\fontsize{10.5pt}{\baselineskip}\selectfont
\maketitle





%以下が本文
\section{序論}

東京大学の入試問題を全自動で解くプロジェクト(東ロボプロジェクト)が進められている\cite{arai2014}.その処理システムは,大学入試センター試験の模擬試験(5教科8科目)で950満点中511点(偏差値は57.8)を獲得,数学と世界史の偏差値は特に高く,数学Ⅰ・A は偏差値64,数学Ⅱ・Bは偏差値65.8,世界史は偏差値66.5であったという\cite{tourobo}.
人工知能がこのように発達すると,その影響は数学教育にも及ぶだろう.今日の数学教育は,すべてを紙と鉛筆で行うことを前提に行われているが,その一部はコンピュータで置き換えることができるはずである.人間が行うこととコンピュータが行うことをうまく識別する能力の育成が求められるようになるだろう.


\section{目的}
本研究では,センター試験の数学I・Aを題材にして,数学教育にコンピュータを導入することの可能性を調査する.その結果として,高校程度の数学能力を問う問題にコンピュータを活用して解く際に必要となる数学の知識とコンピュータの知識を明らかにすることを目指す.

\section{手法}
本研究では,センター試験の数学I・Aの問題をコンピュータを用いて解き,問題を解くために必要な数学の知識とコンピュータの知識を確認する.
2009年から2015年までの7年分のセンター試験の数学I・Aの問題を解く.
本研究では,数式処理システムの中で最も普及しているものの一つであるMathematicaを採用する.
本研究では,無料で利用できるクラウドサービス版,Wolfram Programming Lab\footnote{\url{https://lab.open.wolframcloud.com/objects/wpl/GetStarted.nb}}を用いる.
問題を解いたら,その際に用いたMathematicaの組み込みシンボルの種類を数える. 
センター試験の数学I・Aの問題を解くのに使える組み込みシンボルが出尽くせば,シンボルの種類の増加は止まるはずである.


\section{結果}
2009年から2015年までのセンター試験の数学Ⅰ・Aの問題を解き,利用したMathematicaの組み込みシンボルの累積数を記録した.
利用した組み込みシンボルは全部で23個となる. 



\section{考察}
組み込みシンボルの累積数の増加が図{累積グラフ}のように6年で止まったため,センター試験の数学I・Aの問題を解くのに必要なMathematicaの機能は,「結果」に記載した23種類で十分であることがわかる.Mathematicaのような数式処理システムには膨大な機能が備えられているが,数学I・Aのために必要なのはこのように比較的少数の機能があり,教育の現場に導入するのも容易だと思われる.

\section{結論}
本研究では,数式処理システムMathematicaを用いてセンター試験の数学Ⅰ・Aの問題を解いた.その際に利用したMathematicaの組み込みシンボルを集計することによって,センター試験の数学Ⅰ・Aを解くのに必要な数式処理システムについての知識を明らかにすることができた.本研究のような事例を増やすことによって,コンピュータのサポートのもとで数学の問題を解くのに必要な数学の知識を明らかにすることが今後の課題であろう.


\bibliographystyle{junsrt}
\bibliography{biblio}%「biblio.bib」というファイルが必要.

\end{document}