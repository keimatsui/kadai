%卒論中間審査用研究概要テンプレート ver. 1.0

\documentclass[uplatex,twocolumn]{jsarticle}
\usepackage[top=22mm,bottom=22mm,left=20mm,right=20mm]{geometry}
\setlength{\columnsep}{15mm}
\usepackage[T1]{fontenc}
\usepackage{txfonts}
\usepackage{wrapfig}
\usepackage[expert,deluxe]{otf}
\usepackage[dvipdfmx,hiresbb]{graphicx}
\usepackage[dvipdfmx]{hyperref}
\usepackage{pxjahyper}
\usepackage{secdot}

\makeatletter
\renewcommand{\section}{\@startsection{section}{1}{\z@}{0pt}{0.4\Cvs}{\normalfont\raggedright}}
\renewcommand{\subsection}{\@startsection{subsection}{2}{\z@}{\z@}{\z@}{\normalfont}}
\renewcommand{\subsubsection}{\@startsection{subsubsection}{3}{\z@}{\z@}{\z@}{\normalfont}}
\makeatother
%ここから上を編集する必要はない.





%タイトルと学生番号,名前だけ編集すること
\title{\vspace{-5mm}\fontsize{14pt}{0pt}\selectfont 大学入試試験における数式処理システムの性能評価}
\author{\normalsize プロジェクトマネジメントコース・ソフトウェア開発管理グループ 矢吹研究室 1242116 森谷慧士}
\date{}
\pagestyle{empty}
\begin{document}
\fontsize{10.5pt}{\baselineskip}\selectfont
\maketitle





%以下が本文
\section{研究の背景}
\makeatletter\renewcommand{\section}{\@startsection{section}{1}{\z@}{0.6\Cvs}{0.4\Cvs}{\normalfont\raggedright}}\makeatother%余白の調整(気にしなくていい)
2014年11月にはみずほ銀行がコールセンターにIBMの人工知能であるWatsonを導入したことで話題となった\cite{mizuho2014}.人工知能を利用することで,膨大な解答例データの中から最適な回答案を優先的に表示させ,コールセンターの対応時間の短縮につなげることができる.このように,ビジネス内での様々なシステムに人工知能が導入され始めている.
 
2014年11月に「ロボットは東大に入れるか」という研究が取り上げられて話題となった\cite{tourobo2014}.これは,人工知能を利用して東大入試を突破できる計算機プログラムを開発することにより,「思考するプロセス」を研究しようというものである.この研究により,人工知能が施行して学習するというプロセスを得ることになり,SFに登場するような思考し自己学習をする人工知能を搭載したシステムやロボットが登場してくると推測される.

政府が進めているプロジェクトもある.「第4の産業革命」というロボットや,人工知能を活用した革新的なものづくりを目指す取り組みが始まった.この取り組みは,ドイツで「インダストリー4.0」と呼ばれる動きから始まり,日本政府も経済産業省を中心に取り組まれ始めている\cite{sangyou2014}.

\section{研究目的}
人工知能を使用する際に,我々がプロジェクトマネージャとして必要となる知識が存在する.そこで本研究では,人間が問題を数学的表現に処理する際に必要となる知識を調査する.本研究では,Mathematicaを使用する.Mathematicaとは数式処理を行うツールである.今回は,数式処理を実行させるためにMathematicaに与える命令は何かを研究する.

\section{研究方法}
本研究では,課題研究に引き続いて数学の問題を解く過程を二つにする.一つ目は,数学の問題を理解して,計算式などの数学的表現に処理する過程である.二つ目は,数学的表現に処理した式を数的処理して,値を求める過程である.今回は後者を人工知能に処理させ,前者を人間が処理するように分ける.その際に,人間がいかに簡潔に問題文を処理できるかを研究する.
今回は大学入試センター試験の数学をMathematicaに処理させる.本研究では,使用したコードの数と,利用した数学的知識を集計する.

\section{成果物のイメージ}
試験の年数を重ねることで,新しく使用するコードの種類が減少し,新しくコードを増やすことがなくなると考える.
また,成果物をグラフとし,横軸に実施年数をとり,縦軸にコードや使用した数学的知識をとる.

\section{進捗状況}
現在までに,2011~2015年までのセンター試験の数学1・A をMathematicaに処理させた.現在,処理したデータを集計しており,新しく使用するコードもほぼゼロに近くなっている.

\section{今後の計画}
今後は,引く続き大学入試センター試験の数学1・AをMathematicaに処理させ,新しく使用するコードがゼロになるまで実験を続ける..


\bibliographystyle{junsrt}
\bibliography{biblio}%「biblio.bib」というファイルが必要.

\end{document}