
%卒論中間審査用研究概要テンプレート ver. 1.0

\documentclass[uplatex,twocolumn]{jsarticle}
\usepackage[top=22mm,bottom=22mm,left=20mm,right=20mm]{geometry}
\setlength{\columnsep}{15mm}
\usepackage[T1]{fontenc}
\usepackage{txfonts}
\usepackage{wrapfig}
\usepackage[expert,deluxe]{otf}
\usepackage[dvipdfmx,hiresbb]{graphicx}
\usepackage[dvipdfmx]{hyperref}
\usepackage{pxjahyper}
\usepackage{secdot}

\makeatletter
\renewcommand{\section}{\@startsection{section}{1}{\z@}{0pt}{0.4\Cvs}{\normalfont\raggedright}}
\renewcommand{\subsection}{\@startsection{subsection}{2}{\z@}{\z@}{\z@}{\normalfont}}
\renewcommand{\subsubsection}{\@startsection{subsubsection}{3}{\z@}{\z@}{\z@}{\normalfont}}
\makeatother
%ここから上を編集する必要はない.





%タイトルと学生番号,名前だけ編集すること
\title{\vspace{-5mm}\fontsize{14pt}{0pt}\selectfont Twitterにおけるユーザープロフィールと拡散力の関係分析}
\author{\normalsize プロジェクトマネジメントコース・ソフトウェア開発管理グループ 矢吹研究室 1142016 井上 乃佑}
\date{}
\pagestyle{empty}
\begin{document}
\fontsize{10.5pt}{\baselineskip}\selectfont
\maketitle





%以下が本文
\section{研究の背景}
\makeatletter\renewcommand{\section}{\@startsection{section}{1}{\z@}{0.6\Cvs}{0.4\Cvs}{\normalfont\raggedright}}\makeatother%余白の調整(気にしなくていい)

Twitterは2006年に開始したサービスで,コミュニケーションツールのひとつとして利用されていて,月当たりのアクティブユーザは全世界で13億5000万人,投稿数は一日あたり約5億ツイートされていると言われ,多くの人に使われているソーシャル・ネットワーキング・サービスである\cite{self}.ユーザーはつぶやきと呼ばれる140字の短文を投稿し,それを共有するウェブ上の情報サービスである.そのつぶやきは基本的に全世界の不特定多数のユーザーが閲覧できる,他のユーザーのつぶやきを自分の画面上に共有する仕組みをフォローと言う,また自分がフォローされている人をフォロワーという。さらにTwitterにはリツイート呼ばれるものがあり,それは他の人のツイートを再びツイートするというものである.自分の画面上に流れてきたツイートをリツイートすると,自分のフォロワーの画面にも流れます.同じように,自分がフォローしているユーザーがリツイートすれば,自分の画面上にリツイートが流れてくる.リツイートされるツイートには,ツイート内容という情報以外にアイコンや,ユーザーのIDなどの本質以外の情報も含まれる.
私は現在ツイッターを利用していて,フォローしているユーザーからリツイートが流れてくることがある.その流れてきたリツイートを見てみると,似たような内容でもリツイートされた回数に違いがあることに気づいた.そこで私は自分のプロフィールのアイコンが拡散率に影響があるのではないかと考えた.





\noindent


\noindent


\noindent


\section{目的}
Twitterのアイコンが拡散率に影響があるかを調べ,情報の本質でないアイコン部分が本質に与える影響を調べる.
\section{研究方法}
TwitterのAPIを用いて,タイムラインに流れてきたリツイートのアイコンと,リツイート数,お気に入り数,フォロワー数,を自動で保存しデータを集める.その後リツイートされたアイコンを,自分で決めたいくつかの要素でタグ付けする.その要素の量が細かいほどデータの信憑性は増す.さらに説明変数をタグ付けしたデータ,目的変数をリツイート数/フォロワー数で重回帰分析をする.
他にはロジスティック分析をしてどの要素が一番リツイート率に関係するかを調べ、クラスター分析をしてリツイート率の高い要素を順位付けする。

\section{成果物のイメージ}
昨年の課題研究ではリツイートされたアイコンを,若い男性,中年の男性,年配の男性,子供の男の子,男複数,若い女性,中年の女性,年配の女性,子供の女の子,女複数,男女複数,初期アイコン,男アニメ,男アニメ複数,女アニメ,女アニメ複数,アニメ・マスコット,マスコット・キャラクター,無機物,自作の絵,動物・ペット,ロゴ・マーク,景色・風景,文字,食べ物の25要素で分析した.その結果,要素の数が少なく正確な分析とは言えなかった.分析方法は重回帰分析だけでなく,他の分析方法を使いより正確な結果を目指す.






\section{進捗状況}
昨年の課題研究の結果から,リツイートされたアイコンをタグ付けする上での要素が少なく感じた.そのため去年の要素のほかに,いつくかの要素を考えた.
\section{今後の計画}
アイコンをタグ付けする上での要素をさらに増やし,タイムラインに流れてくるリツイートを集める.また,課題研究では重回帰分析のみの分析方法であったが,他の分析方法を試してより正確な分析を目指す.



\bibliographystyle{junsrt}
\bibliography{biblio}%「biblio.bib」というファイルが必要.

\end{document}

