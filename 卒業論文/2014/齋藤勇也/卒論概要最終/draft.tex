%卒業論文概要テンプレート ver. 1.0

\documentclass[uplatex,twocolumn,dvipdfmx]{jsarticle}
\usepackage[top=22mm,bottom=22mm,left=20mm,right=20mm]{geometry}
\setlength{\columnsep}{15mm}
\usepackage[T1]{fontenc}
\usepackage{txfonts}
\usepackage{wrapfig}
\usepackage[expert,deluxe]{otf}
\usepackage[dvipdfmx,hiresbb]{graphicx}
\usepackage[dvipdfmx]{hyperref}
\usepackage{pxjahyper}
\usepackage{secdot}

\makeatletter
\renewcommand{\section}{%
  \@startsection{section}{1}{\z@}%
  {0.6\Cvs}{0.4\Cvs}%
  {\normalfont\normalsize\raggedright}}
\renewcommand{\subsection}{\@startsection{subsection}{2}{\z@}%
  {\z@}{\z@}%
  {\normalfont\normalsize}}
\renewcommand{\subsubsection}{\@startsection{subsubsection}{3}{\z@}%
  {\z@}{\z@}%
  {\normalfont\normalsize}}
\makeatother
%ここから上を編集する必要はない.





%タイトルと学生番号,名前だけ編集すること
\title{\vspace{-5mm}\fontsize{14pt}{0pt}\selectfont オンラインショッピングサイト利用者による商品に対するレビューの動向調査}
\author{\normalsize プロジェクトマネジメントコース・ソフトウェア開発管理グループ 矢吹研究室 1242042 齋藤 勇也}
\date{}
\pagestyle{empty}
\begin{document}
\fontsize{10.5pt}{\baselineskip}\selectfont
\maketitle





%以下が本文
\section{序論}




インターネットを利用した電子商取引は1994年に米国のピザハットが行ったのが最初であるといわれている\cite{sugasaka2003}.
1994年以前,商品の購入方法は商品の下に足を運ぶ必要があった.つまり商品を購入した人物は直接顔を合わせた相手にのみにしかレビューを語ることが出来ない状態である.

書籍によると全ての商取引における電子取引の割合が2014年時点で3.7%となり,2008年の1.8%比べ 倍近く上昇し,仮想空間でも商品の売買が行いやすい環境であることが分かる\cite{keizai2014}.
仮想空間での商品の売買が可能となった1994年以降は,電子商取引であるオンラインショッピングのレビューが重要視されている.



レビューが実装されている有名なオンラインショッピングサイトでは,利用者は商品についてのレビューを記入することや,商品に得点を付けることが可能である.例えば,Amazonのレビューではおすすめ度と称して平均値しか表示していない.商品とは無関係のレビューや商品の特徴を理解せず記述したレビューがあり,平均値などの統計に加えるべきでないものが存在する.

そこで,レビューの表示方法が平均評価では商品の判断材料としての指標が少ないと判断し,平均値よりも信頼できる方法を探す.\cite{hattori2011} \cite{yamazawa2006}.






\section{目的}

Amazonはレビューを平均値で表示している.
他の指標を加え判断材料を増やすことで,平均値のみの表示よりも信頼できるレビューを作る.

\section{手法}

各レビューがどれだけ信用できるかの指標として,参考になったと答えた人物の比率,購入者のレビューのみでの参考になったと答えた人物の比率を計測する.


\section{結果}

平均との差異を図るために計86件のレビューデータを収集し,散布図を作成した.
平均評価と重み付き評価の間で相関があり,相関係数は0.97という数値が算出された.
また,Amazon内で購入した,していない人物を分類分けした.
9件のレビューデータを収集し,平均評価と購入した人物のみのレビューを使用した重み付き評価の散布図を作成し,相関係数は0.5という数値が算出された.

\section{考察}

重み付き評価値は相関係数が0.97であり,購入者のみの重み付き評価値は相関係数0.5で購入者のみの相関係数のほうが相関が低いと言える.


\section{結論}

レビューデータ全体から分析する方法に比べ,相関が低い結果が出た購入者に絞ったレビューデータから参考になったと答えた比率を分析すれば平均値より正確な商品の評価がわかる可能性が高いと言える.


\bibliographystyle{junsrt}
\bibliography{biblio}%「biblio.bib」というファイルが必要.

\end{document}
