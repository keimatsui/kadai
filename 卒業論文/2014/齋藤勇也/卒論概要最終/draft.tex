%卒業論文概要テンプレート ver. 1.0

\documentclass[uplatex,twocolumn,dvipdfmx]{jsarticle}
\usepackage[top=22mm,bottom=22mm,left=20mm,right=20mm]{geometry}
\setlength{\columnsep}{15mm}
\usepackage[T1]{fontenc}
\usepackage{txfonts}
\usepackage{wrapfig}
\usepackage[expert,deluxe]{otf}
\usepackage[dvipdfmx,hiresbb]{graphicx}
\usepackage[dvipdfmx]{hyperref}
\usepackage{pxjahyper}
\usepackage{secdot}

\makeatletter
\renewcommand{\section}{%
  \@startsection{section}{1}{\z@}%
  {0.6\Cvs}{0.4\Cvs}%
  {\normalfont\normalsize\raggedright}}
\renewcommand{\subsection}{\@startsection{subsection}{2}{\z@}%
  {\z@}{\z@}%
  {\normalfont\normalsize}}
\renewcommand{\subsubsection}{\@startsection{subsubsection}{3}{\z@}%
  {\z@}{\z@}%
  {\normalfont\normalsize}}
\makeatother
%ここから上を編集する必要はない.





%タイトルと学生番号,名前だけ編集すること
\title{\vspace{-5mm}\fontsize{14pt}{0pt}\selectfont オンラインショッピングサイト利用者による商品に対するレビューの動向調査}
\author{\normalsize プロジェクトマネジメントコース・ソフトウェア開発管理グループ 矢吹研究室 1242042 齋藤 勇也}
\date{}
\pagestyle{empty}
\begin{document}
\fontsize{10.5pt}{\baselineskip}\selectfont
\maketitle





%以下が本文
\section{序論}



オンラインショッピングサイトでは商品についてのレビューを記載することができる.このレビューについて下記に記載する.

インターネットを利用した電子商取引は1994年に米国のピザハットが行ったのが最初であるといわれている\cite{sugasaka2003}.
このことから,1994年以前は商品の購入方法は販売する場所に足を運ぶ必要があり,商品を購入した人物は直接顔を合わせた相手にのみにしかレビューを語ることが出来ない.

%全ての商取引における電子取引の割合が2014年時点で3.7%となり,2008年の1.8%比べ 倍近く上昇し,仮想空間でも商品の売買が行いやすいことが分かる\cite{keizai2014}.
%仮想空間での商品の売買が可能となった1994年以降は,電子商取引であるオンラインショッピングのレビューが重要視されている.

オンラインショッピングサイトでは,利用者は商品についてのレビューを記入することだけでなく,商品に得点を付けることが可能である.Amazonのレビューではおすすめ度として平均値しか表示していない.しかし,商品とは無関係のレビューや商品の特徴を理解せず記述したレビューがあり,平均値の統計に加えるべきでないものも存在する\cite{yamazawa2006}.
上記のような統計に加えるべきでないレビューも入ってしまうため,平均評価よりも信頼できる方法を探すこととした.


%\cite{hattori2011}




\section{目的}

従来のレビューでは平均値のみが判断材料になっていたが,他の指標を加え判断材料をさらに増やすことで,平均値のみの表示よりも信頼できるレビューを作る.

\section{手法}


参考になったと判断した人物の比率を掛けることと購入者の絞込みを行うことの二点の判断材料を増やすことにした.
また,参考になったと判断した人物の比率を掛けた平均評価を重み付き平均評価と呼称する.

レビューデータはAmazonの2003年のアニメ映画から収集することにした.
Amazonのレビューが平均値と比較してどれだけ信用できるかの指標として,「平均評価」「重み付き平均評価」「購入者に絞った平均評価」「購入者に絞った重み付き平均評価」3点を計測する.


\section{結果}

21件の商品データから927件のレビューを収集し,平均評価を基準とした散布図を3つ作成した.

\begin{enumerate}
%\setlength{\parskip}{3mm}


 \item	平均評価と購入者に絞った平均評価の比較を行った.
線形近似曲線は$y = 0.6687x + 1.611$であり相関係数は$R^2 = 0.586$となった.

 \item	平均評価と重み付き平均評価の比較を行った.
線形近似曲線は$y = 1.3418x - 2.6459$であり相関係数は$R^2 = 0.5669$となった.

 \item	平均評価と購入者に絞った重み付き平均評価の比較を行った.
線形近似曲線は$y = 0.9676x - 1.0229$であり相関係数は$R^2 = 0.2295$となった.

\end{enumerate}

標準偏差は購入者のみで0.64,購入者を含めた全体で0,81であり,約0.2ほど差がある.

\section{考察}

2003年のアニメ映画という条件では「購入者のみの重み付き平均評価」が約0.2と低い数値が求められた.今後,参考になったと判断した人物の比率を掛けることと購入者の絞込みを行うことの二点の判断材料を設けた場合このような結果になることを考慮し進められる.

\section{結論}

購入者と全体の標準偏差が0.2しか差がないことから購入者の配点の特徴が見受けられなかった.
しかし,2003年のアニメ映画という条件でしか適応されないため今後,年度やジャンルを増やし幅を広げていく必要がある.


\bibliographystyle{junsrt}
\bibliography{biblio}%「biblio.bib」というファイルが必要.

\end{document}
