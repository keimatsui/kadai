%卒論中間審査用研究概要テンプレート ver. 1.0

\documentclass[uplatex,twocolumn]{jsarticle}
\usepackage[top=22mm,bottom=22mm,left=20mm,right=20mm]{geometry}
\setlength{\columnsep}{15mm}
\usepackage[T1]{fontenc}
\usepackage{txfonts}
\usepackage{wrapfig}
\usepackage[expert,deluxe]{otf}
\usepackage[dvipdfmx,hiresbb]{graphicx}
\usepackage[dvipdfmx]{hyperref}
\usepackage{pxjahyper}
\usepackage{secdot}

\makeatletter
\renewcommand{\section}{\@startsection{section}{1}{\z@}{0pt}{0.4\Cvs}{\normalfont\raggedright}}
\renewcommand{\subsection}{\@startsection{subsection}{2}{\z@}{\z@}{\z@}{\normalfont}}
\renewcommand{\subsubsection}{\@startsection{subsubsection}{3}{\z@}{\z@}{\z@}{\normalfont}}
\makeatother
%ここから上を編集する必要はない.





%タイトルと学生番号,名前だけ編集すること
\title{\vspace{-5mm}\fontsize{14pt}{0pt}\selectfont 農業におけるWikiを活用する知の構造化}
\author{\normalsize プロジェクトマネジメントコース・ソフトウェア開発管理グループ 矢吹研究室 1242034 氏名 小池 克人}
\date{}
\pagestyle{empty}
\begin{document}
\fontsize{10.5pt}{\baselineskip}\selectfont
\maketitle





%以下が本文
\section{研究の背景}
\makeatletter\renewcommand{\section}{\@startsection{section}{1}{\z@}{0.6\Cvs}{0.4\Cvs}{\normalfont\raggedright}}\makeatother%余白の調整(気にしなくていい)

平成27年3月10日に内閣官房情報通信技術(IT)総合戦略室が発表した農業情報の標準化に関する個別ガイドライン等に関する説明会では,農業の情報の相互運用性を確保するインターオペラビティーとポータビリティー確保の標準化が必要とされている\cite{naikaku2014}.

しかし,異なる企業や団体の意思を一つにまとめる作業には困難が伴う.標準化して,データをやり取りするプロトコルやデータ形式は各企業の利害がぶつかるため,困難である\cite{kizi2015}.そのため複数のシステムの間でマスターを統一しようとする共通語彙には,目的と実現する考え方のアーキテクチャーが異なり,マスターは構造も用語も異なるため現実感が無いため,共通的な方法論ができないかを考える.

作物名称は,研究,行政,流通,農家自身のための作物名称などによって名称が変る.研究は研究目的に応じた分類のため,行政は行政の目的に応じた分類のため,流通は流通に都合の良いネーミングのため,農家自身は営農の都合の良い分類のための用途により視点が異なるため用語も異なる.

これを目的に応じたよう用語の変換(翻訳)を実現する共通な仕組みができないかに着目する.

Wikipediaの記事構造からの上位下位関係抽出の論文からMediaWikiを利用する\cite{nogyo2015}.MediaWikiは,オープンソース(GPL)で配布されているため,MediaWikiを用いれば,自分専用のWikipediaを設置・運用することができるため,MediaWikiを利用し,目的に応じて最適な語彙の翻訳を可能とする検索システムの仕組みの開発を研究する.



\section{目的}
農業情報の用途により視点,用語が異なる語彙を目的に応じた最適な語彙への翻訳ができるような仕組みを作ることが目的である.
\section{研究方法}
以下の方法で研究する.
\begin{enumerate}
\item MediaWikiのサーバーを立ち上げ.
\item MediaWikiに知識を登録する.
\item Wikiに登録した情報から,単語間の関連情報を抽出する.
\item 抽出した情報を使って,用語の翻訳を試みる.
\end{enumerate}


\section{成果物のイメージ}
類似名称のある作物や意味が同じ作物固有の名称がある作物など目的に応じて最適な語彙の翻訳を可能とする検索をすることで,用語と世界をつなぐことができる.
\section{進捗状況}
現在の進捗状況は,検索システムのプロトタイプの開発のための準備をしている.
\section{今後の計画}
今後は以下のように進める.
\begin{enumerate}
\item MediaWikiの立ち上げ.
\item データの入力を行う.
\item 入力したデータを抽出できるかを実験.
\end{enumerate}




\bibliographystyle{junsrt}
\bibliography{biblio}%「biblio.bib」というファイルが必要.

\end{document}

