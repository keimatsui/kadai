%卒論中間審査用研究概要テンプレート ver. 1.0

\documentclass[uplatex,twocolumn]{jsarticle}
\usepackage[top=22mm,bottom=22mm,left=20mm,right=20mm]{geometry}
\setlength{\columnsep}{15mm}
\usepackage[T1]{fontenc}
\usepackage{txfonts}
\usepackage{wrapfig}
\usepackage[expert,deluxe]{otf}
\usepackage[dvipdfmx,hiresbb]{graphicx}
\usepackage[dvipdfmx]{hyperref}
\usepackage{pxjahyper}
\usepackage{secdot}

\makeatletter
\renewcommand{\section}{\@startsection{section}{1}{\z@}{0pt}{0.4\Cvs}{\normalfont\raggedright}}
\renewcommand{\subsection}{\@startsection{subsection}{2}{\z@}{\z@}{\z@}{\normalfont}}
\renewcommand{\subsubsection}{\@startsection{subsubsection}{3}{\z@}{\z@}{\z@}{\normalfont}}
\makeatother
%ここから上を編集する必要はない.





%タイトルと学生番号,名前だけ編集すること
\title{\vspace{-5mm}\fontsize{14pt}{0pt}\selectfont 研究タイトル}
\author{\normalsize プロジェクトマネジメントコース・ソフトウェア開発管理グループ 矢吹研究室 1242034 氏名 小池 克人}
\date{}
\pagestyle{empty}
\begin{document}
\fontsize{10.5pt}{\baselineskip}\selectfont
\maketitle





%以下が本文
\section{研究の背景}
\makeatletter\renewcommand{\section}{\@startsection{section}{1}{\z@}{0.6\Cvs}{0.4\Cvs}{\normalfont\raggedright}}\makeatother%余白の調整(気にしなくていい)

平成27年3月10日に内閣官房情報通信技術(IT)総合戦略室が発表した農業情報の標準化に関する個別ガイドライン等に関する説明会では,農業の情報の相互運用性を確保するインターオペラビティーとポータビリティー確保の標準化が必要となっている\cite{naikaku2014}.

しかし,異なる企業や団体の意思を一つにまとめる作業には困難が伴う標準化して,データをやり取りするプロトコルやデータ形式は各企業の利害がぶつかるため,決して容易ではない\cite{kizi2015}.そして,多様なシステムの間でマスターを統一しようとする共通語彙には,目的と実現する考え方のアーキテクチャーが異なり,マスターは構造も用語も異なるため現実感が無いため,共通的な方法論ができないかを考える.

作物名は,研究のための作物名称,行政のための作物名称,流通のための作物名称,農家自身のための作物名称などによって名称が変る.研究のための作物名称は研究目的に応じた分類のため,行政のための作物名称は行政の目的に応じた分類のため,流通のための作物名称は流通に都合の良いネーミングのため,農家自身のための作物名称は営農の都合の良い分類のための用途により視点が異なるため用語も異なる.

これを目的に応じたよう用語の変換(翻訳)を実現する共通な仕組みができないかに着目して,目的に応じて最適な語彙の翻訳を可能とする検索システムの仕組みの開発を研究する\cite{nogyo2015}.





\section{目的}
農業情報の用途により視点,用語が異なる語彙を目的に応じた最適な語彙への翻訳を目的とする.
\section{研究方法}
\begin{enumerate}
\item Media Wikipediaのサーバーを立ち上げ.
\item Wikipediaに知識を入れる.
\item その知識をを抽出できるようにしたもののプロトタイプが動くことを検証する.
\end{enumerate}


\section{成果物のイメージ}
類似名称のある作物や意味が同じ作物固有の名称がある作物など目的に応じて最適な語彙の翻訳を可能とする検索をすることで,用語と世界をつなぐことができる.
\section{進捗状況}
現在の進捗状況は,検索システムのプロトタイプの開発のための準備をしている.
\section{今後の計画}

今後は以下のように進める.

\begin{enumerate}
\item Wikipediaの立ち上げ.
\item データの入力を行う.
\item 2で行ったものの実験.
\end{enumerate}




\bibliographystyle{junsrt}
\bibliography{biblio}%「biblio.bib」というファイルが必要.

\end{document}

