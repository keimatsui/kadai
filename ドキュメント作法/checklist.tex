\documentclass[uplatex,9pt,a5paper]{jsarticle}
\usepackage[dvipdfm,paperwidth=148truemm,paperheight=210truemm,top=10truemm,bottom=10truemm,left=10truemm,right=10truemm,includehead,includefoot]{geometry}
\usepackage[dvipdfmx]{hyperref}
\pagestyle{empty}

\begin{document}
\begin{center}
ドキュメント作法 チェックリスト
\end{center}

\noindent チェックした文書:

\section*{重要なこと}
\begin{description}
\item[□ 他人の言葉を盗用してはいない] 文章は自分の言葉で書くこと。他人の言葉を使う場合は,それが引用であることが明確にわかるようにすること。\TeX の場合,長い引用にはquotation環境やquote環境を使う。短い引用は「」の中に書く。
\item[□ 参考文献の形式は正しい] \url{https://github.com/taroyabuki/yabukilab/wiki/参考文献リストの書き方}を参照し,参考文献を正確に書くこと。
\item[□ すべての文の主語と述語があっている] 主語と述語の関係を確認するためには、文章を「・・・が・・・する」とか「・・・は・・・だ」というように、短い形に変換して違和感がないことを確かめればいい。
\item[□ 一つの段落で二つ以上の話題を扱っていない]
\item[□ 段落の最初の文が,その段落で最も重要な文になっている] 各段落の最初の文だけを抜き出して並べたとき,その文章のだいたいの意味が通じればよい。
\item[□ 長すぎる文はない]
\end{description}

\section*{細かいこと}
\begin{description}
\item[□ 参考文献番号が句読点の前に付けいている] 「.[1]」ではなく「[1].」
\item[□ 句読点が「,.」(いずれも全角)になっているか]
\item[□ 本文とキャプションの両方で図について説明している]
\item[□ 図がベクター形式で埋め込まれている] Acrobatで最大まで拡大したときに,ぎざぎざになってはいけない。PowerPointやExcelで作成した図は,PDF形式で出力してTeX原稿に取り込むこと。
\item[□ 箇条書きで始まる段落はない]
\end{description}

\section*{コメント(任意)}

\vspace{13truemm}\noindent 責任を持ってサインすること

\noindent 日付:\hspace{40truemm}氏名:\hspace{40truemm}
\end{document}
