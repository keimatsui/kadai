%課題研究レジュメテンプレート ver. 1.0

\documentclass[uplatex]{jsarticle}
\usepackage[top=20mm,bottom=20mm,left=20mm,right=20mm]{geometry}
\usepackage[T1]{fontenc}
\usepackage{txfonts}
\usepackage{wrapfig}
\usepackage[expert,deluxe]{otf}
\usepackage[dvipdfmx,hiresbb]{graphicx}

\makeatletter
  \renewcommand{\section}{%
    \if@slide\clearpage\fi
    \@startsection{section}{1}{\z@}%
    {\Cvs \@plus.5\Cdp \@minus.2\Cdp}% 前アキ
    {.5\Cvs \@plus.3\Cdp}% 後アキ
    %{\normalfont\Large\headfont\raggedright}}
    {\normalfont\raggedright}}

  \renewcommand{\subsection}{\@startsection{subsection}{2}{\z@}%
    {\Cvs \@plus.5\Cdp \@minus.2\Cdp}% 前アキ
    {.5\Cvs \@plus.3\Cdp}% 後アキ
    %{\normalfont\large\headfont}}
    {\normalfont}}

  \renewcommand{\subsubsection}{\@startsection{subsubsection}{3}{\z@}%
    {\Cvs \@plus.5\Cdp \@minus.2\Cdp}%
    {\z@}%
    %{\normalfont\normalsize\headfont}}
    {\normalfont}}
\makeatother
%ここから上を編集する必要はない.





\title{\vspace{-14mm}千葉工業大学入試試験における数式処理システムの性能評価}
\author{PMコース 矢吹研究室 1242116 森谷慧士}
\date{}%日付を入れる必要はない.
\pagestyle{empty}%ページ番号は振らない.
\begin{document}
\maketitle






\section{研究の背景}

2014年11月にはみずほ銀行がコールセンターにIBMの人工知能であるWatsonを導入したことで話題となった\cite{mizuho2014}.なぜなら人工知能を利用することで,膨大な解答例データの中から最適な回答案を優先的に表示させ,コールセンターの対応時間の短縮につなげることができるからである.このように,ビジネス内での様々なシステムに人工知能が導入され始めているので,プロジェクトマネージャも人工知能を理解し,活用していくべきだと考える.
 
人工知能の更なる進化のための研究も進められている.2014年11月に「ロボットは東大に入れるか」という研究が取り上げられて話題となった\cite{tourobo2014}.これは,人工知能を利用して東大入試を突破できる計算機プログラムを開発することにより,「思考するプロセス」を研究しようというものである.この研究により,人工知能が施行して学習するというプロセスを得ることになり,SFに登場するような人工知能を搭載したシステムやロボットが登場してくると推測される.そこで,人工知能を制御する人間をまとめ的確に指示を出す役割としてプロジェクトマネージャが活躍していくと考える.

「第4の産業革命」というロボットや,人工知能を活用した革新的なものづくりを目指す取り組みが始まった.この取り組みは,ドイツで「インダストリー4.0」と呼ばれる動きから始まり,日本政府も経済産業省を中心に取り組まれ始めている\cite{sangyou2014}.第4の産業革命では,人工知能が工場の一つ一つの機械に対して最適な動きを指示するので,大量生産から完全受注の個別生産まで様々な生産体制に対応することができる.そして,工場の機械を管理し人工知能に的確な指示を出す役割を,プロジェクトマネージャが果たすべきだと考える.




\section{研究の目的}

本研究では,数学の問題を解く過程を二つにする.
一つ目は,数学の問題を理解して,計算式などの数学的表現に処理する過程である.
二つ目は,数学的表現に処理した式を数的処理して,値を求める過程である.今回は後者を人工知能に処理させ,前者を人間が処理するように分ける.その際に,人間がいかに簡潔に問題文を処理できるかを研究する.






\section{プロジェクトマネジメントとの関連}

人工知能を使用する際に,我々がプロジェクトマネージャとして必要となる知識が存在する.
そこで本研究では,人間が問題を数学的表現に処理する際に必要となる知識を調査する.






\section{研究の方法}

本研究では,Mathematicaを使用する.Mathematicaとは数的処理を行うツールである.本研究で使用するツールは,ウェブ上でMathematicaを使用できるWolfram Mathematica Onlineである.


\begin{enumerate}
\item 問題文を理解しMathematicaで数的処理できるように式に変換する\cite{nagaoka2014}.
\item この際に使用した数学の知識をまとめ,統計を取る.
\item Mathematicaを利用して,式を数的処理する\cite{wolfram2014}\cite{nihon2011}\cite{sakakibara2010}.
\item この際にMathematicaで使用した関数をまとめ,統計を取る.
\end{enumerate}







\section{現在の進捗状況}

千葉工業大学2014年A日程の数学の入試問題全問を,Mathematicaに処理させた.

問題文を数学的表現に処理する際に,必要とした数学的知識をまとめた.千葉工業大学の数学の入試問題は,約3割の問題で人間が数学的表現に処理する必要がなく,そのまま数的処理に移行できた.さらに,その他の問題でも問題文で与えられている式や値を公式に当てはめるだけで,数的処理が可能な問題が多くあることが分かった.

Mathematicaで処理する際に使用した関数は何かをまとめ,利用した頻度にグラフ化した.すると,最も多く使用した関数は,方程式あるいは不等式でxなどの変数の値を処理するSolveであった.
以下にSolveを使用した式の処理例を記載する.

\[
  \frac{x+5}{x^2 +x-2} = \frac{a}{x-1} +\frac{b}{x+2}
\]
この式が,xについての恒等式である時にaとbについての値を求める.xについての恒等式より恒等式を求める関数であるSolveAlwaysを使用する.処理式は
\[SolveAlways[(x+5)/(x^2 +x-2) ==a/(x-1) +b/(x+2) ,x] \]
である.これをMathematicaに打ち込むと{{a→2,b→-1}}と答えが返ってくる.

二番目に多く用いられた関数は,式を簡略化させるSimplifyであった.
以下にSimplifyを使用した式の処理例を記載する.

\[
(5^{\frac{1}{3}} -5^{\frac{-1}{3}} )(5^{\frac{2}{3}} +1+5^{\frac{-2}{3}} ) 
\]
を簡略した値を求める.簡略した式を処理する関数であるSimplifyを使用する.処理式は
\[Simplify[(5^{\frac{1}{3}} -5^{\frac{-1}{3}} )(5^{\frac{2}{3}} +1+5^{\frac{-2}{3}} ) ] \]
である.これをMathematicaに処理させると$\frac{24}{5}$と答えが返ってくる.

以上より,千葉工業大学の数学の入試問題は,方程式に当てはめて解く問題や,与えられた式を簡略化することで,値を求める問題が多いことが分かった.




\section{今後の計画}

千葉工業大学の入試問題の中で,まだ研究していない他の日程の問題を,引き続きMathematicaで処理させ統計を取る.
また,千葉大学などの他の大学の入試問題をMathematicaで処理できるか検証する.

さらに, IBMの人工知能のツールであるWatsonを利用して,プロジェクトの効率化が図れるかどうかを研究する.

\bibliographystyle{junsrt}
\bibliography{biblio}%「biblio.bib」というファイルが必要.

\end{document}
