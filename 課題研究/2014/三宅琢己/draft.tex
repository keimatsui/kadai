%課題研究レジュメテンプレート ver. 1.0

\documentclass[uplatex]{jsarticle}
\usepackage[top=20mm,bottom=20mm,left=20mm,right=20mm]{geometry}
\usepackage[T1]{fontenc}
\usepackage{txfonts}
\usepackage{wrapfig}
\usepackage[expert,deluxe]{otf}
\usepackage[dvipdfmx,hiresbb]{graphicx}

\makeatletter
  \renewcommand{\section}{%
    \if@slide\clearpage\fi
    \@startsection{section}{1}{\z@}%
    {\Cvs \@plus.5\Cdp \@minus.2\Cdp}% 前アキ
    {.5\Cvs \@plus.3\Cdp}% 後アキ
    %{\normalfont\Large\headfont\raggedright}}
    {\normalfont\raggedright}}

  \renewcommand{\subsection}{\@startsection{subsection}{2}{\z@}%
    {\Cvs \@plus.5\Cdp \@minus.2\Cdp}% 前アキ
    {.5\Cvs \@plus.3\Cdp}% 後アキ
    %{\normalfont\large\headfont}}
    {\normalfont}}

  \renewcommand{\subsubsection}{\@startsection{subsubsection}{3}{\z@}%
    {\Cvs \@plus.5\Cdp \@minus.2\Cdp}%
    {\z@}%
    %{\normalfont\normalsize\headfont}}
    {\normalfont}}
\makeatother
%ここから上を編集する必要はない.





\title{\vspace{-14mm}集合知の成功事例としての株価変動についての調査}
\author{PMコース 矢吹研究室 1242109 三宅琢己}
\date{}%日付を入れる必要はない.
\pagestyle{empty}%ページ番号は振らない.
\begin{document}
\maketitle





\section{研究の背景}


集合知というものに興味があったので,この研究を行うことにした.

本を読んでいた時,たまたま書いてあったことがあった.
ある事故が起きたとき,原因もわかっていないのにとある株式会社の株価が下がった.
その数か月後にその株価の下がった株式会社が原因でその事故が起きたと公開された.

これを見た時,偶然起きたのかそれとも株式市場は賢く,先にわかっていたためそこの株を大勢が売って株価が下がったのかと思った.
これは集合知と関係しているので研究を行うことにした.





\section{研究の目的}

株式市場というのは本当に賢く,
事故原因というのを誰よりも早く知っているのか.
というの調査する.





\section{プロジェクトマネジメントとの関連}


集合知というのはナレッジマネジメントの分野ではもともとの意味は複数人の智恵の集合と言う意味である。.
開発や課題解決に取り組む時,天才的な超優秀な一人より、それなりに優秀な何人かの集合が、複数の視点を上手に使って取り組む方が高い成果を生むことができる場合もある.
ということから何かプロジェクトを行うときは集合知を使えば高い成果をあげられる.
このように関連している.





\section{研究の方法}

\subsection{\LaTeX 原稿の書き方}

事故を調べる.
事故の条件
複数の企業がかかわっている
・すぐに原因企業が見つからず、あとから判明する
・原因の企業が株式会社である

この条件に当てはまる事故を
インターネットで検索をかけて調べる.

そしてその原因であった株式会社の株価が事故直後に下がっているものは
トレンドを時系列分析をして,トレンドの変化を調べる.





\subsubsection{参考文献}


\begin{itemize}
\item Wikipedia\cite{Wikipedia}
\item ヤフーファイナンス\cite{yahoo japan}
\end{itemize}

細かい注意:\LaTeX の命令の先頭は「\UTF{005C}\hspace{-0.5zw}」だが,Webなどの資料ではそれが「\UTF{00A5}\hspace{-0.5zw}」になっていることがある.テキストエディタでは,VLゴシックのような「\UTF{005C}\hspace{-0.5zw}」と「\UTF{00A5}\hspace{-0.5zw}」を区別できるフォントを使うといい.






\section{現在の進捗状況}

事故の条件に当てはまる事故は,Wikipediaで事故を調べて一つ一つ見ていた結果4つ出てきた.
その4つの事故のうち,原因企業の株価をヤフーファイナンスやgoogleで調べたところ3社の株価しか
出てこなかった.
株価の下がった企業は3社のうち2社.
現状,データが少ないのと各事故との関連企業を調べられていないので,
結果ははっきりと言えない状況である.



\section{今後の計画}

事故が4つあるので,各事故の日時の株価の変動を見ていき,株価の下がっている企業を
見ていきその事故と関連性のある企業か調べていく.

そしてそれぞれ関連性のある企業の時系列分析を行い,トレンドの変化を見ていき,
原因企業の株価の動きが他の企業と異なっていれば株式市場は
原因判明前にわかっていたと判断する.




\bibliographystyle{junsrt}
\bibliography{biblio}%「biblio.bib」というファイルが必要.

\end{document}
