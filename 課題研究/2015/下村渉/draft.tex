%課題研究レジュメテンプレート ver. 1.2

\documentclass[uplatex]{jsarticle}
\usepackage[top=20mm,bottom=20mm,left=20mm,right=20mm]{geometry}
\usepackage[T1]{fontenc}
\usepackage{txfonts}
\usepackage{wrapfig}
\usepackage[expert,deluxe]{otf}
\usepackage[dvipdfmx,hiresbb]{graphicx}
\usepackage[dvipdfmx]{hyperref}
\usepackage{pxjahyper}
\usepackage{secdot}

\makeatletter
  \renewcommand{\section}{%
    \if@slide\clearpage\fi
    \@startsection{section}{1}{\z@}%
    {\Cvs \@plus.5\Cdp \@minus.2\Cdp}% 前アキ
    {.5\Cvs \@plus.3\Cdp}% 後アキ
    %{\normalfont\Large\headfont\raggedright}}
    {\normalfont\raggedright}}

  \renewcommand{\subsection}{\@startsection{subsection}{2}{\z@}%
    {\Cvs \@plus.5\Cdp \@minus.2\Cdp}% 前アキ
    {.5\Cvs \@plus.3\Cdp}% 後アキ
    %{\normalfont\large\headfont}}
    {\normalfont}}

  \renewcommand{\subsubsection}{\@startsection{subsubsection}{3}{\z@}%
    {\Cvs \@plus.5\Cdp \@minus.2\Cdp}%
    {\z@}%
    %{\normalfont\normalsize\headfont}}
    {\normalfont}}
\makeatother
%ここから上を編集する必要はない.





\title{\vspace{-14mm}Wikiを活用する概念関係構築支援システム}
\author{PMコース 矢吹研究室 1342069 下村 渉}
\date{}%日付を入れる必要はない.
\pagestyle{empty}%ページ番号は振らない.
\begin{document}
\maketitle
\section{研究の背景}
プロジェクトの進行に密接に関わってくる環境的・技術的などの要因は多々あるが,最も重要な事は関係者間のコミュニケーションである.コミュニケーションを重要視して適切に取っていく事は関係者間での認識のズレ,使用の抜け・漏れ,誤解が生まれるのを防止するという効果があるので,プロジェクトの円滑な進行を期待できる.発注者と開発者との間の「不適切なコミュニケーション」の例としては次のようなものが考えられる.
\begin{itemize}
\item 開発者が専門的な用語で説明をしてしまうと,発注者は十分な理解を得られない可能性がある
\item 発注者が「この機能」は実装して当然だ」と想定していても,開発者は「発注者から何も言われていない」という理由で機能をしないかもしれない
\end{itemize}
仕様書や設計書,計画書などのドキュメントを作る目的の一つとしては「関係者間のコミュニケーションツールとして利用するため」という事である.これらのドキュメントは「関係者間のコミュニケーションツールとして利用する」という事を意識して,「不適切なコミュニケーション」が発生しないように作成する事で,プロジェクトの円滑な進行に役立つツールとなる\cite{a}.

本研究では発注者と開発者がそれぞれ使う専門的な用語を目的に応じた用語に変換を実現するため共通な仕組みができないか着目し,用語の変換を行うために用語をタグで結び,関連用語を抽出することを考える.私はこれと同じような事例を探し,参考に研究し,さらに,2008年のWikipediaの記事構造から上位下位関係抽出の論文を参考にしてMediaWikiを利用する\cite{b}.上位下位関係の例えとして,Aが上位語でBが下位語だったとき,上位下位関係は「A/B」と示すとして,「ラーメン」と「食品」は「食品/ラーメン」,「ファイナルファンタジー」と「ゲーム」は「ゲーム/ファイナルファンタジー」と示す.
MediaWikiは現在,オープンソース(GPL)で配布されているため,誰でも無償で利用できる\cite{c}.
デフォルトの設定では誰もがどのページをも編集することができるが,設定をすれば自分用として運用ができる\cite{d}.
以上のことから本研究はMediaWikiを利用し,目的に応じて語彙の変換を可能にするシステムを研究する.





\section{研究の目的}

本研究の目的は,MediaWiki内にある単語の上位下位関係を抽出し,関係者間での視点や用語が異なる語彙を目的に応じた最適な語彙への変換ができるようなシステムを作ること,また卒業論文で応用するため知識を深めることである.





\section{プロジェクトマネジメントとの関連}

本研究は,発注者が発注する際に作る仕様書,設計書などのドキュメントを開発者が十分に理解するために使用する.これは発注者と開発者の不適切なコミュニケーションを避けることが出来るのでコミュニケーションマネジメントに関連づくと考えられる.\\




\section{研究の方法}
本研究では以下のように進めていく.
\begin{itemize}
\item MediaWikiのサーバーを立ち上げる
\item Ruby1.8,MeCab,Pecco,Hyponumyをインストールする
\item MediaWikiで専門用語のページを作る
\item 上位下位関係抽出ツールを使い作ったページから用語間の関連情報を抽出し,抽出した情報から用語の翻訳する
\end{itemize}










\section{現在の進捗状況}

\begin{wraptable}[11]{r}[1cm]{8cm}
\vspace*{-\intextsep}
\caption{メインキャラクター}\label{サンプル表}
\begin{tabular}{ccc}\hline
上位語 & 下位語 & SVM \\
\hline
{メインキャラクター} & イオ・ユークレース & 1.229342\\
{メインキャラクター} & オイゲン & 1.229342\\
{メインキャラクター} & カタリナ・アリゼ & 1.229342\\
{メインキャラクター} & ビィ & 1.229342\\
{メインキャラクター} & ラカム & 1.229342\\
{メインキャラクター} & ルリア & 1.229342\\
{メインキャラクター} & ロゼッタ & 1.229342\\ \hline
\end{tabular}
\end{wraptable}

現在の進捗状況は上位下位関係抽出ツールを使うためMediaWiki,Ruby1.8,MeCab,Pecco,Hyponumyををインストールした.テストデータとしてMediaWikiでいくつかページを作り,作ったMediaWikiのページのデータをXMLにダンプし上位下位関係抽出ツールで解析し抽出した.ここではMediaWikiで作ったページから抽出したテストデータの一部を紹介する.\\



表1の結果は「グランブルーファンタジー」というページの「メインキャラクター」の解析結果で,左から上位語,下位語,SVMを示している.
SVMは抽出した上位下位関係の信頼度の指標となり値が大きいほど高信頼度である.
次に同じページの「登場キャラクター」の解析結果の一部を表2に記載する.

\begin{wraptable}[10]{r}[1cm]{8cm}
\vspace*{-\intextsep}
\caption{登場キャラクター}\label{サンプル表}
\begin{tabular}{ccc}\hline
上位語 & 下位語 & SVM \\
\hline
{登場キャラクター} & イオ・ユークレース & 1.349616\\
{登場キャラクター} & ぴにゃこら太 & 0.996346\\
{登場キャラクター} &オイゲン & 1.349616\\
{登場キャラクター} & カッタクリ & 1.068125\\
{登場キャラクター} & ラカム & 1.372444\\
{登場キャラクター} & ピエール &1.505221\\
{登場キャラクター} & ロゼッタ & 1.372444\\
{登場キャラクター} & 帝国兵& 0.874956\\\hline
\end{tabular}
\end{wraptable}

記載した解析結果は「メインキャラクター」とそれ以外のキャラクターの結果を交互に並べたものである.この結果より表1の「メインキャラクター」で抽出されたキャラはSVMが1.3を超えている.そうでないキャラクターは基本下回っていることが分かるが,たまに上回っているものもある.\\
\\
\\
\\
\\
\\


\section{今後の計画}


今後の計画は専門用語を検索したときに適切な用語に転換できるよう検索システムの開発を予定している.


\bibliographystyle{junsrt}
\bibliography{biblio}%「biblio.bib」というファイルが必要.

\end{document}
