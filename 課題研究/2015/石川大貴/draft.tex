%課題研究レジュメテンプレート ver. 1.2

\documentclass[uplatex]{jsarticle}
\usepackage[top=20mm,bottom=20mm,left=20mm,right=20mm]{geometry}
\usepackage[T1]{fontenc}
\usepackage{txfonts}
\usepackage{wrapfig}
\usepackage[expert,deluxe]{otf}
\usepackage[dvipdfmx,hiresbb]{graphicx}
\usepackage[dvipdfmx]{hyperref}
\usepackage{pxjahyper}
\usepackage{secdot}

\makeatletter
  \renewcommand{\section}{%
    \if@slide\clearpage\fi
    \@startsection{section}{1}{\z@}%
    {\Cvs \@plus.5\Cdp \@minus.2\Cdp}% 前アキ
    {.5\Cvs \@plus.3\Cdp}% 後アキ
    %{\normalfont\Large\headfont\raggedright}}
    {\normalfont\raggedright}}

  \renewcommand{\subsection}{\@startsection{subsection}{2}{\z@}%
    {\Cvs \@plus.5\Cdp \@minus.2\Cdp}% 前アキ
    {.5\Cvs \@plus.3\Cdp}% 後アキ
    %{\normalfont\large\headfont}}
    {\normalfont}}

  \renewcommand{\subsubsection}{\@startsection{subsubsection}{3}{\z@}%
    {\Cvs \@plus.5\Cdp \@minus.2\Cdp}%
    {\z@}%
    %{\normalfont\normalsize\headfont}}
    {\normalfont}}
\makeatother
%ここから上を編集する必要はない.





\title{\vspace{-14mm}開発現場のツールと技法の調査}
\author{PMコース 矢吹研究室 1342011 石川大貴}
\date{}%日付を入れる必要はない.
\pagestyle{empty}%ページ番号は振らない.
\begin{document}
\maketitle





\section{研究の背景}

複数の人たちによるチーム開発を進めていくと様々な問題に直面する.具体的には,課題をチーム間で適切に共有できず,進捗が見えにくくなることや,複数人の人たちで1つの製品を開発するため,開発内容が競合したりする.特にソフトウェア開発の現場では,コードの質を均一化することや,製品コードの全容を把握することも難しくなる.また,一度リリースしたらそれで開発が終わりということではなく,長い年月をかけてアップデートをしながら運用し続けることが多くなってきている.

このような問題を改善するために様々なツールや技法が開発されている.チーム開発では,どんな状況にも対応できる万能なツールや技法があるというわけではない.企業の規模や製品の規模,チームの規模など状況に応じて適切なものは変わる.自分の行うプロジェクトにはどんなものが適切か見極めて選ぶ必要がある.\cite{ikeda2014}

ある企業の製品開発現場を見る機会があった,その企業も様々な問題に直面していた.ベンチャー企業ということもあり,ノウハウや知識が十分にあるわけではない.開発を進める上で必要なツールや技法の選定を試行錯誤しながら行っていた.



\section{研究の目的}

背景で挙げた企業のプロジェクトでの問題点を発見し改善できるようなツールや技法の提案をし,業務の効率を上げられるようにする.



\section{プロジェクトマネジメントとの関連}
プロジェクトマネジメントにおける10個の知識エリアのうち,タイムとコストマネジメントに関連づけられる.計画の段階でどのようなツールや技法が自分たちのプロジェクトに適切かを判断し,より時間とコストを削減することを考えるためである.


\section{研究の方法}

\begin{enumerate}
\item 実際の企業の開発現場の様子を調査して現状を把握する
\item 現状を見た上で企業のプロジェクトを進める上での問題点を発見する
\item 現在のプロジェクトにおいて使われているツールや技法について調査する
\item その企業に適した問題解決のためのツールや技法を提案する
\item 提案したツールや技法で改善できるか観察する
\end{enumerate}



\section{現在の進捗状況}

ある実際の企業の開発現場を調査したのでそこで把握したことをまとめる.10人程度で新製品開発を行うベンチャー企業である.作成する機能等によって個人または複数人でチームを組んで活動している.3つのプロジェクトを進めており全員がすべてのプロジェクトに同時にアサインしている.

現在主に使用しているツールや技法は以下である.
スクラムというアジャイルソフトウェア開発手法の一つを使用している.\cite{nagase2012}スクラムでは開発を機能単位などで短期間にわけ,くり返し行うことによってスピーディーに,またあとからの変更に柔軟に対応できるようにしている.
コミュニケーションツールにはslackを使用している.slackの導入によりコミュニケーションコストは大幅に減少した.slackは使いやすいUIでどんなデバイスでも使うことができる.他のサービスとの連携が豊富なことや,簡単にログをたどれることからチーム内のコミュニケーションは活性化したようだ.また勤退管理としてもslackを使っている.コミュニケーションツールにslackを使用している.
バージョン管理システムにはGitHubを使用している.ソースコードの変更内容を記録してくれる.全員で同時に開発することができ,その際間違って他人の変更を上書きしないで済む仕組みがある.
また,任意の時点まで巻き戻すことができる等様々なメリットがある.

発見された問題点を報告する.
タスクの管理がうまくできていないことが問題点として挙げられる.予定や進行中のタスク状況はslackやMTGのやりとりで把握している.タスクの量が多くなってくる,複数のプロジェクトに同時にアサインしているといった状況から,今自分はどのプロジェクトをやるべきなのか,どのタスクが最優先なのか,残り期限はいつまでかといったことが分かりにくくなってしまっていた.

発見された問題点への解決策を提案する.
提案するものはTrelloというツールである.
Trelloは1つのタスクを1つのカードに書き起こしてボードに張り付けていくイメージで管理するツールである.
ToDo,Doing,Doneなどといったエリアに分けて,タスクが今どの状況にあるのかを把握することができる.
また,急なタスクの発生時に容易に追加することができる.
そのためスケジュールの変更が多いアジャイル開発に適しているといえる.
メンバー同士の共有ができお互いの進捗状況を把握できる点も有用である.
さらに,Elegantt for Trelloという拡張機能を使えばTrelloで管理しているタスクをガントチャートに変換してくれる.
プロジェクトマネジメントでタスクの進捗管理をしようというときにガントチャートを使うことが多い.
全体の計画を見える化することができ全体像をつかみやすいというメリットがあり,今回挙げた問題を解決できるといえる.
しかし,今回調査した企業でも以前に一度ガントチャートを使用していたことがあるがやめていた.
それはアジャイルソフトウェア開発での開発であるため,スケジュールの変更が多く,ガントチャートの修正に時間をかけてしまうからだ.
Elegantt for Trelloでは,Trelloでタスクを追加するだけで自動的にガントチャートが更新される.
そのため急な変更にも柔軟に対応できると考える.
Trelloは比較的簡単に導入できると考える.
無料で使用することができ,とてもシンプルで使いやすいUIになっているため,導入時のオペレーションも少なくて済むだろう.
こういったことから,あまりコストをかけたくないというこの企業にはよいツールとなると考える.

もう一つの提案をする.
チケット管理システムの導入である.
主な機能として,タスクの定義から担当者のアサイン,期限の管理やタスクのステータスの確認といった機能を持ったシステムのことである.
Trelloよりも一覧性,検索性が高い点や,過去の知見,またはプログラムの仕様などプロジェクトの重要情報が1か所で管理,共有できる点は優れている.
また,チケット管理システムを使って,チケットを中心として開発フローを組み立てた方法論をチケット開発駆動といい,このような方法論を取り入れることも有用であろう.\cite{ikeda2014}
さらに,バージョン管理システムと連携できるものも多く,効果的に利用することでソースコードの変更ををいつ,だれが,どのように行ったのかが関連づいて見えるようになる.
それによって,問題が生じた場合でも原因の究明を早くすることができるようになる.
Trelloよりも導入コストはかかってしまうが,チケット管理システムの導入とバージョン管理システムの連携ができれば,現状より効率よくタスク管理ができるであろう.

\section{今後の計画}

提案の実践後の変化等を数週間ごとに調査し,よりよい改善を行う.
開発現場の現状と使えるツールと技法をさらに調査し,新たな提案を考える.



\bibliographystyle{junsrt}
\bibliography{biblio}%「biblio.bib」というファイルが必要.

\end{document}