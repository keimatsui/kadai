%課題研究レジュメテンプレート ver. 1.2

\documentclass[uplatex]{jsarticle}
\usepackage[top=20mm,bottom=20mm,left=20mm,right=20mm]{geometry}
\usepackage[T1]{fontenc}
\usepackage{txfonts}
\usepackage{wrapfig}
\usepackage[expert,deluxe]{otf}
\usepackage[dvipdfmx,hiresbb]{graphicx}
\usepackage[dvipdfmx]{hyperref}
\usepackage{pxjahyper}
\usepackage{secdot}

\makeatletter
  \renewcommand{\section}{%
    \if@slide\clearpage\fi
    \@startsection{section}{1}{\z@}%
    {\Cvs \@plus.5\Cdp \@minus.2\Cdp}% 前アキ
    {.5\Cvs \@plus.3\Cdp}% 後アキ
    %{\normalfont\Large\headfont\raggedright}}
    {\normalfont\raggedright}}

  \renewcommand{\subsection}{\@startsection{subsection}{2}{\z@}%
    {\Cvs \@plus.5\Cdp \@minus.2\Cdp}% 前アキ
    {.5\Cvs \@plus.3\Cdp}% 後アキ
    %{\normalfont\large\headfont}}
    {\normalfont}}

  \renewcommand{\subsubsection}{\@startsection{subsubsection}{3}{\z@}%
    {\Cvs \@plus.5\Cdp \@minus.2\Cdp}%
    {\z@}%
    %{\normalfont\normalsize\headfont}}
    {\normalfont}}
\makeatother
%ここから上を編集する必要はない.





\title{\vspace{-14mm}小規模ベンチャー企業でのPM技術導入実験}
\author{PMコース 矢吹研究室 1342011 石川大貴}
\date{}%日付を入れる必要はない.
\pagestyle{empty}%ページ番号は振らない.
\begin{document}
\maketitle





\section{研究の背景}

複数の人たちによるチームでの開発を進めていくと様々な問題に直面する.
具体的には,課題をチーム間で適切に共有できず,進捗が見えにくくなることや,複数人の人たちで1つの製品を開発するため,開発内容が競合したりする.
特にソフトウェア開発の現場では,コードの質を均一化することや,製品コードの全容を把握することも難しくなる.
また,一度リリースしたらそれで開発が終わりということではなく,長い年月をかけてアップデートをしながら運用し続けることが多くなってきているため,過去の記録をたどらなければいけない場面がある.

このような問題を改善するために様々なツールや技法が開発されている.
チームでの開発では,どんな状況にも対応できる万能なツールや技法があるというわけではない.
企業の規模や製品の規模,チームの規模など状況に応じて適切なものは変わる.
自分の行うプロジェクトにはどんなものが適切か見極めて選ぶ必要がある\cite{ikeda2014}.

私はabaという製品開発を行う企業の現場を見る機会がある.
abaもチームでの開発をする上での様々な問題に直面している.
abaは2011年に設立した,ヘルスケア業界向けのロボティクス技術の研究開発やサービスの提供をしている企業である\cite{abalab}.
新しい企業であるためノウハウや知識が十分にあるわけではない.
そのため,開発を進める上で必要なツールや技法の選定を試行錯誤しながら行っている.



\section{研究の目的}

abaのプロジェクトでの問題点を発見し改善できるようなツールや技法の提案をすることで,業務の効率を上げられるようにする.



\section{プロジェクトマネジメントとの関連}

本研究はプロジェクトマネジメントにおける10個の知識エリアのうち,タイムとコストマネジメントに関連づけられる.
計画の段階でどのようなツールや技法が自分たちのプロジェクトに適切かを判断し,より時間とコストを削減することを考えるためである.



\section{研究の方法}

以下の手順で研究を進める.
\begin{enumerate}
\item abaの開発現場の様子を調査して現状を把握する.
\item 多くのプロジェクトにおいて使われているツールや技法について調査する.
\item abaの現状を見た上で,プロジェクトでの問題点を発見する.
\item abaに適した問題解決のためのツールや技法を提案する.
\item 提案したツールや技法で改善できるか観察する.
\end{enumerate}



\section{現在の進捗状況}

abaの開発現場を調査したので,そこで把握したことをまとめる.
\begin{itemize}
\item 10人程度で新製品の研究開発を行うベンチャー企業である.
\item 作成する機能等によって個人または複数人でチームを組んで活動している.
\item 3つのプロジェクトを進めており全員がすべてのプロジェクトに同時にアサインしている.
\item スクラムというアジャイルソフトウェア開発手法の一つを使用している\cite{nagase2012}.
スクラムにより開発を機能単位などで短期間にわけ,くり返し行うことによってスピーディーに,またあとからの変更に柔軟に対応できるようにしている.
\item コミュニケーションツールにはslackを使用している.
slackの導入によりコミュニケーションコストは大幅に減少した.
slackは使いやすいUIでどんなデバイスでも使うことができる.
他のサービスとの連携が豊富なことや,簡単にログをたどれることからチーム内のコミュニケーションは活性化した.
また勤退管理としてもslackを使っている.
\item バージョン管理システムにはGitHubを使用している.
\end{itemize}

タスクの管理がうまくできていないことが問題点として挙げられた.
abaのプロジェクトのガントチャートを作成した結果で,優先順位等のタスク間の関係が分かりにくいということや,1人にタスクが偏ってしまっていることが分かった.
スタッフの話でもタスクの量が増えてきている,複数のプロジェクトに同時にアサインしているといった状況から,今自分はどのプロジェクトをやるべきなのか,残り期限はいつまでかが分かりにくいとのことだ.
abaでは予定や進行中のタスク状況は主にslackやミーティングのやりとりで把握しているだけであり,このような点からタスクの管理がうまくできていないことが問題点として挙げられる.

発見された問題点への解決策を2つ提案する.

1つ目の提案はTrelloというツールの使用である.
Trelloは1つのタスクを1つのカードに書き起こしてボードに張り付けていくイメージで管理できるタスク管理ツールである.
タスクが今どの状況にあるのかを把握しやすくなる.
また,急なタスクの発生時に容易に追加することができるため,スケジュールの変更が多いアジャイル開発に適しているといえる.
さらに,Elegantt for Trelloという拡張機能を使えばTrelloで管理しているタスクをガントチャートに変換してくれる.
全体の計画を見える化することができ,全体像をつかみやすいというメリットがありる.
Trelloは比較的簡単に導入でき,あまりコストをかけたくないというabaにはよいツールとなると考える.
無料で使用することができ,とてもシンプルで使いやすいUIになっているため,導入時のオペレーションも少なくて済むだろう.

2つ目の提案はチケット管理システムの導入である\cite{ikeda2014}.
チケット管理システムは,タスクの定義から担当者のアサイン,期限の管理やタスクのステータスの確認といった機能を持ったシステムのことである.
Trelloよりも一覧性,検索性が高い点や,過去の知見,またはプログラムの仕様などプロジェクトの重要情報が1か所で管理,共有できる点は優れている.
さらに,バージョン管理システムと連携できるものも多く,効果的に利用することでソースコードの変更ををいつ,だれが,どのように行ったのかが関連づいて見えるようになる.
それによって,問題が生じた場合でも原因の究明を早くすることができるようになる.
Trelloよりも導入コストはかかってしまうが,チケット管理システムの導入とバージョン管理システムの連携ができれば,現状より効率よくタスク管理ができると考える.



\section{今後の計画}

提案後のプロジェクトの変化を調査し,よりよい改善を行う.

開発現場の現状やツールと技法をさらに調査し,新たな提案を考える.



\bibliographystyle{junsrt}
\bibliography{biblio}%「biblio.bib」というファイルが必要.

\end{document}