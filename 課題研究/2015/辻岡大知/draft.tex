%課題研究レジュメテンプレート ver. 1.1

\documentclass[uplatex]{jsarticle}
\usepackage[top=20mm,bottom=20mm,left=20mm,right=20mm]{geometry}
\usepackage[T1]{fontenc}
\usepackage{txfonts}
\usepackage{wrapfig}
\usepackage[expert,deluxe]{otf}
\usepackage[dvipdfmx,hiresbb]{graphicx}
\usepackage[dvipdfmx]{hyperref}
\usepackage{pxjahyper}

\makeatletter
  \renewcommand{\section}{%
    \if@slide\clearpage\fi
    \@startsection{section}{1}{\z@}%
    {\Cvs \@plus.5\Cdp \@minus.2\Cdp}% 前アキ
    {.5\Cvs \@plus.3\Cdp}% 後アキ
    %{\normalfont\Large\headfont\raggedright}}
    {\normalfont\raggedright}}

  \renewcommand{\subsection}{\@startsection{subsection}{2}{\z@}%
    {\Cvs \@plus.5\Cdp \@minus.2\Cdp}% 前アキ
    {.5\Cvs \@plus.3\Cdp}% 後アキ
    %{\normalfont\large\headfont}}
    {\normalfont}}

  \renewcommand{\subsubsection}{\@startsection{subsubsection}{3}{\z@}%
    {\Cvs \@plus.5\Cdp \@minus.2\Cdp}%
    {\z@}%
    %{\normalfont\normalsize\headfont}}
    {\normalfont}}
\makeatother
%ここから上を編集する必要はない.





\title{\vspace{-14mm}ソーシャルメディアにおける炎上案件に関わるユーザの動向調査}
\author{PMコース 矢吹研究室 1342081 辻岡 大知}
\date{}%日付を入れる必要はない.
\pagestyle{empty}%ページ番号は振らない.
\begin{document}
\maketitle





\section{研究の背景}

 SNS(ソーシャル・ネットワーキング・サービス)やブログが一般に認知され始めてから炎上が発生している.炎上とはSNSやブログの投稿に対し非難・批判・誹謗・中傷などのコメントやトラックバックが発生することである.炎上は2010年から顕著に増加している\cite{a}.

 短文の投稿を共有するウェブ上の情報サービスであるtwitterでは日常的に悪質なツイートがされる\cite{b}.例えばファッションセンターしまむらの店員に対し土下座を強要し,その光景を写真に収めツイートした件やコンビニエンスストアであるローソンの従業員がアイスケースの中に入った写真をツイートした件,ゲームセンターのアーケードゲームを壊した写真をツイートし,自慢した件などがある.

 日常的に悪質なツイートがされることに対し,悪ふざけや犯罪を自慢するツイート,情報モラル,情報リテラシーが低いツイートを見過ごさず,通報やリツイートをする正義感溢れる人達がいる.彼らはそれ相応の罰を受ける必要があるという気持ちや何度も同じ過ちを繰り返してしまわないようにという正義感から通報やリツイートをする.リツイート数が伸びると便乗してリツイートするユーザが増え,結果事態が大きくなってしまい炎上してしまう場合がある.
 
 炎上してしまった場合,次のようなトラブルを起こしまったケースもある.例えばスーパー「カスミ」で客がアイスケースに入って撮影しツイートした結果,アイスをすべて撤去することになり,身元を特定され書類送検された.他にも経済評論家である勝間和代さんが仲の良い友人4人と食事会で東京・日本橋のレストランに訪れた際,来店した勝間和代を飲食店店員がtwitterで中傷した件では,勝間さん本人からお客を貶める感想をツイートするのは問題である旨を返信され,最終的にアカウントを削除して逃亡するという結末を迎えた\cite{t}.

 以上のような事態に陥らないためには,炎上しないための対策が必要である. 炎上しないための対策はいくつかある.例えば,発言は誰でも見ることができることを意識する,プライベートなアカウントでも用心する,過去に犯罪を起こしたことがあるかのような発言は絶対にしない,発言の前に友人や知人に確認するなどの方法がある\cite{enjo}.それに加えて私は多くの炎上をリツイートしているユーザをブロックするという方法も炎上しないための対策の1つだと考えた.そのため,多くの炎上をリツイートしているユーザの特定をすることは炎上するリスク対策につながると考えた.





\section{研究の目的}

本研究の目的は
\begin{itemize}
 \item 悪ふざけの投稿や犯罪自慢投稿,情報モラル,リテラシーの低いツイートをより多くリツイートしているユーザの特定をする.
 \item 炎上のリスク対策ができるような指標を作成する.
\end{itemize}
以上2点の目標を達成することを目指す.


\section{プロジェクトマネジメントとの関連}

本研究はプロジェクトマネジメントにおける10個の知識エリアのうち,リスクマネジメントに関連付けられる\cite{d}.犯罪を自慢するツイートに対し,通報やリツイートすることにより,さらなる罪を犯す可能性を抑止している為である.また,多くの炎上をリツイートしているユーザをブロックすることにより炎上の対策になる為である.







\section{研究の方法}

本研究はtwitterを使用しデータの集計をする.集めるデータは炎上しているツイートのidとそのツイートをリツイートしているユーザの内部idである.内部idとはtwitter内部でユーザの管理用にユーザ毎に振られているシリアルナンバーのことである\cite{id}.データを集めるためtwitterAPIを使用する\cite{api}.本研究は以下のように進める.

\begin{enumerate}
 \item twitterAPIを使用することのできるプログラムの作成をする.
 \item 炎上案件を集め,ツイートのidを取得する.
 \item 作成したプログラムを使用しリツイートしたユーザの内部idを取得する.
 \item リツイートしたユーザの内部idの最頻値を取る. 
\end{enumerate}









\section{現在の進捗状況}

\subsection{進捗}
\begin{enumerate}
 \item twitterAPIを使用することができるプログラムの作成をした.
 \item 作成したプログラムを使用し5つの炎上を集め,統計を取った.
 \item データの量が少なすぎるためユーザの特定をすることができなかったと考察した.
\end{enumerate}

\subsection{考察}
現在多くの炎上をリツイートしているユーザの特定はできていない.集めることのできた炎上案件の数が少ないことがユーザの特定できなかった原因として挙げられる.データの量を増やすためには炎上案件を自動的に取得することができるプログラムの作成をする必要があると考察する.




\section{今後の計画}

データを増やすことにより,多くの炎上ツイートをリツイートしているユーザの特定をすることができると仮定し,以下のように研究を進める計画である.
\begin{enumerate}
 \item 自動で炎上ツイートを取得できるようにする.
 \item 集めたデータの分析をする.
 \item 多くの炎上ツイートをリツイートしているユーザを特定する.
 \item 論文の執筆を行う.
\end{enumerate}


\bibliographystyle{junsrt}
\bibliography{biblio}%「biblio.bib」というファイルが必要.

\end{document}

