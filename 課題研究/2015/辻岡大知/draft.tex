%課題研究レジュメテンプレート ver. 1.1

\documentclass[uplatex]{jsarticle}
\usepackage[top=20mm,bottom=20mm,left=20mm,right=20mm]{geometry}
\usepackage[T1]{fontenc}
\usepackage{txfonts}
\usepackage{wrapfig}
\usepackage[expert,deluxe]{otf}
\usepackage[dvipdfmx,hiresbb]{graphicx}
\usepackage[dvipdfmx]{hyperref}
\usepackage{pxjahyper}

\makeatletter
  \renewcommand{\section}{%
    \if@slide\clearpage\fi
    \@startsection{section}{1}{\z@}%
    {\Cvs \@plus.5\Cdp \@minus.2\Cdp}% 前アキ
    {.5\Cvs \@plus.3\Cdp}% 後アキ
    %{\normalfont\Large\headfont\raggedright}}
    {\normalfont\raggedright}}

  \renewcommand{\subsection}{\@startsection{subsection}{2}{\z@}%
    {\Cvs \@plus.5\Cdp \@minus.2\Cdp}% 前アキ
    {.5\Cvs \@plus.3\Cdp}% 後アキ
    %{\normalfont\large\headfont}}
    {\normalfont}}

  \renewcommand{\subsubsection}{\@startsection{subsubsection}{3}{\z@}%
    {\Cvs \@plus.5\Cdp \@minus.2\Cdp}%
    {\z@}%
    %{\normalfont\normalsize\headfont}}
    {\normalfont}}
\makeatother
%ここから上を編集する必要はない.





\title{\vspace{-14mm}twitterにおける炎上をリツイートしているユーザーの特定}
\author{PMコース 矢吹研究室 1342081 辻岡 大知}
\date{}%日付を入れる必要はない.
\pagestyle{empty}%ページ番号は振らない.
\begin{document}
\maketitle





\section{研究の背景}

 SNSやブログが一般に認知され始めてから炎上が発生している.炎上とはSNSやブログの投稿に対し非難・批判・誹謗・中傷などのコメントやトラックバックが発生することである.炎上は2010年から顕著に増加している\cite{a}.

 短文の投稿を共有するウェブ上の情報サービスであるtwitterでは日常的に悪質な投稿がされる.例えばファッションセンターしまむらの店員に対し土下座を強要し,その光景を写真に収めツイートした件やコンビニエンスストアであるローソンの従業員がアイスケースの中に入った写真をツイートした件,ゲームセンターのアーケードゲームを壊した写真をツイートし,自慢した件などがある.

 それに対し,悪ふざけや犯罪を自慢するツイート,情報モラル,情報リテラシーが低いツイートを見過ごさず,通報やリツイートをする正義感溢れる人達がいる.彼らはそれ相応の罰を受ける必要があるという気持ちや何度も同じ過ちを繰り返してしまわないようにという正義感から通報やリツイートをする.リツイート数が伸びると便乗してリツイートするユーザーが増え,結果事態が大きくなってしまい炎上してしまう場合がある.炎上してしまい身元を特定されてしまったケースもある.

 炎上しないためには対策が必要である. 炎上しないための対策として多くの炎上をリツイートしているユーザーをブロックするという手段がある.そのため,多くの炎上をリツイートしているユーザーの特定をすることは炎上するリスク対策につながると考えた.





\section{研究の目的}

研究の目的は,悪ふざけの投稿や犯罪投稿,情報モラル,リテラシーの低い投稿をより多く拡散しているユーザーの特定をすることである.




\section{プロジェクトマネジメントとの関連}

本研究はプロジェクトマネジメントにおける10個の知識エリアのうち,リスクマネジメントに関連付けられる.犯罪を自慢するツイートに対し,通報やリツイートすることにより,さらなる罪を犯す可能性を抑止している為である.また,多くの炎上をリツイートしているユーザーをブロックすることにより炎上の対策になる為である.







\section{研究の方法}

本研究はtwitterを使用しデータの集計をする.集めるデータは炎上しているツイートのidとそのツイートをリツイートしているユーザーの内部idである.内部idとはtwitter内部でユーザーの管理用にユーザー毎に振られているシリアルナンバーのことである.データを集めるためtwitterAPIを使用する.本研究は2段階に分かれる.

\begin{enumerate}
 \item twitterAPIを用いて炎上のデータを集める.
 \item 集めたデータをExcelに表示し,分析する.
\end{enumerate}









\section{現在の進捗状況}

本研究は以下のように進んでいる.
\begin{enumerate}
 \item twitterAPIを使用することができるプログラムの作成をした.
 \item 作成したプログラムを使用し5つの炎上を集め,統計を取った.
 \item データの量が少なすぎるためユーザーの特定をすることができなかったと考察した.
\end{enumerate}





\section{今後の計画}

データを増やすことにより,多くの炎上ツイートをリツイートしているユーザーの特定をすることができると仮定し,以下のように研究を進める計画である.
\begin{enumerate}
 \item 自動で炎上ツイートを取得できるようにする.
 \item 集めたデータの分析をする.
 \item 多くの炎上ツイートをリツイートしているユーザーを特定する.
 \item 論文の執筆を行う.
\end{enumerate}



\bibliographystyle{junsrt}
\bibliography{biblio}%「biblio.bib」というファイルが必要.

\end{document}

