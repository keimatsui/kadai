%課題研究レジュメテンプレート ver. 1.1

\documentclass[uplatex]{jsarticle}
\usepackage[top=20mm,bottom=20mm,left=20mm,right=20mm]{geometry}
\usepackage[T1]{fontenc}
\usepackage{txfonts}
\usepackage{wrapfig}
\usepackage[expert,deluxe]{otf}
\usepackage[dvipdfmx,hiresbb]{graphicx}
\usepackage[dvipdfmx]{hyperref}
\usepackage{pxjahyper}

\makeatletter
  \renewcommand{\section}{%
    \if@slide\clearpage\fi
    \@startsection{section}{1}{\z@}%
    {\Cvs \@plus.5\Cdp \@minus.2\Cdp}% 前アキ
    {.5\Cvs \@plus.3\Cdp}% 後アキ
    %{\normalfont\Large\headfont\raggedright}}
    {\normalfont\raggedright}}

  \renewcommand{\subsection}{\@startsection{subsection}{2}{\z@}%
    {\Cvs \@plus.5\Cdp \@minus.2\Cdp}% 前アキ
    {.5\Cvs \@plus.3\Cdp}% 後アキ
    %{\normalfont\large\headfont}}
    {\normalfont}}

  \renewcommand{\subsubsection}{\@startsection{subsubsection}{3}{\z@}%
    {\Cvs \@plus.5\Cdp \@minus.2\Cdp}%
    {\z@}%
    %{\normalfont\normalsize\headfont}}
    {\normalfont}}
\makeatother
%ここから上を編集する必要はない.





\title{\vspace{-14mm}twitterにおける炎上をリツイートしているユーザーの特定}
\author{PMコース 矢吹研究室 1342081 辻岡 大知}
\date{}%日付を入れる必要はない.
\pagestyle{empty}%ページ番号は振らない.
\begin{document}
\maketitle





\section{研究の背景}

 急速な情報化が進んだ今日においては,情報の流れが我々を取り巻く生活環境を一変させ,教育や産業のあり方までを大きく変容させつつある.情報社会を健全かつ柔軟に生き抜くためには,高度なICT技能(Information and Communication Technology)のみならず,情報リテラシーや情報モラルに関する広い見識と判断能力,そして確かな倫理観が必要となる.

 しかし,twitterなどのSNSやブログでは日常的に悪ふざけの投稿や恐喝まがいの投稿,犯罪をしたことを自慢するかのような投稿がされる.これらの投稿をするのは「内輪ルール優位性」によるものだと考える.内輪ルール優位性とは,本人たちは悪いとわかっていてもみんなやっているから自分もするべきだという感覚に陥ることである.そして,問題を起こした後,武勇伝的な感覚で投稿をする.さらに,何かのきっかけで悪ふざけの度合いが増すと,その暴走は止まらず、罪の意識が薄れていく\cite{a}.

 それに対し,悪ふざけのツイートや情報モラル,情報リテラシーが低いツイートを見過ごさず,通報やリツイートにより同じ過ちを繰り返さないようにする正義感溢れる人達がいる.

 私の研究ではそのような多くの炎上ツイートをリツイートしているユーザーの特定,共通点の発見をする研究である.






\section{研究の目的}

悪ふざけのツイートや犯罪ツイート,情報モラル,リテラシーの低いツイートに対し,多くのリツイートをしているユーザーの特定をする.





\section{プロジェクトマネジメントとの関連}

本研究はプロジェクトマネジメントの10個の知識エリアのうち、リスクマネジメントに関連付けられる.
犯罪を自慢するツイートに対し,通報やリツイートをすることにより,更なる罪を犯す可能性を抑止している為である.







\section{研究の方法}

twitterを使用しデータの集計をする.集めるデータは炎上しているツイートのidとそのツイートをリツイートしているユーザーの内部idである.データを集めるためubuntuとtwitterAPIを使用する.研究は2段階に分かれる.

\begin{enumerate}
 \item ubuntuでtwitterAPIを用いて炎上ツイートのデータを集める
 \item 集めたデータをExcleに表示し,分析する
\end{enumerate}


twitterで炎上ツイートを見つけ,炎上ツイートのidとツイートしたユーザーのidをデータベースに格納し,集めたデータをExcleに表示した後,Excleのcount機能を使用し多くの炎上ツイートをリツイートしているユーザーを特定する.








\section{現在の進捗状況}

ubuntuでtwitterAPIを使用することができるプログラムを作成した。作成したプログラムを使用し5つの炎上ツイートのデータを集めて統計を行った.

1つの炎上ツイートからリツイートしたユーザーのidは最大で100件までしか取得することができない.そのため,多くの炎上ツイートをリツイートしているユーザーの特定はすることができなかった.多くの炎上ツイートをリツイートしているユーザーを特定するためには,大量の炎上ツイートを用意する必要があると考える.そのためには炎上ツイートの定義をさらに細かく設定し,炎上ツイートを自動で取得できるプログラムの開発を行う必要があると考える.

多くの炎上ツイートの情報を集める際、データ項目数を増やし多くの炎上ツイートをリツイートしているユーザーの共通点を見つける.




\section{今後の計画}

炎上ツイートの定義をより細かく正確なものにし、自動で炎上ツイートを取得できるようにする.

炎上ツイートしているツイートの多くをリツイートしているユーザーを特定し,集めるデータ項目を増やしそのユーザーの共通点を見つける研究を行う.


\bibliographystyle{junsrt}
\bibliography{biblio}%「biblio.bib」というファイルが必要.

\end{document}

