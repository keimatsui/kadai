%課題研究レジュメテンプレート ver. 1.1

\documentclass[uplatex]{jsarticle}
\usepackage[top=20mm,bottom=20mm,left=20mm,right=20mm]{geometry}
\usepackage[T1]{fontenc}
\usepackage{txfonts}
\usepackage{wrapfig}
\usepackage[expert,deluxe]{otf}
\usepackage[dvipdfmx,hiresbb]{graphicx}
\usepackage[dvipdfmx]{hyperref}
\usepackage{pxjahyper}

\makeatletter
  \renewcommand{\section}{%
    \if@slide\clearpage\fi
    \@startsection{section}{1}{\z@}%
    {\Cvs \@plus.5\Cdp \@minus.2\Cdp}% 前アキ
    {.5\Cvs \@plus.3\Cdp}% 後アキ
    %{\normalfont\Large\headfont\raggedright}}
    {\normalfont\raggedright}}

  \renewcommand{\subsection}{\@startsection{subsection}{2}{\z@}%
    {\Cvs \@plus.5\Cdp \@minus.2\Cdp}% 前アキ
    {.5\Cvs \@plus.3\Cdp}% 後アキ
    %{\normalfont\large\headfont}}
    {\normalfont}}

  \renewcommand{\subsubsection}{\@startsection{subsubsection}{3}{\z@}%
    {\Cvs \@plus.5\Cdp \@minus.2\Cdp}%
    {\z@}%
    %{\normalfont\normalsize\headfont}}
    {\normalfont}}
\makeatother
%ここから上を編集する必要はない.





\title{\vspace{-14mm}twitterにおける炎上をリツイートしているユーザーの特定}
\author{PMコース 矢吹研究室 1342081 辻岡 大知}
\date{}%日付を入れる必要はない.
\pagestyle{empty}%ページ番号は振らない.
\begin{document}
\maketitle





\section{研究の背景}

急速な情報化が進んだ今日においては,情報の流れが我々を取り巻く生活環境を一変させ,教育や産業のあり方までを大きく変容させつつある.情報社会を健全かつ柔軟に生き抜くためには,高度なICT技能のみならず,情報リテラシーや情報モラルに関する広い見識と判断能力,そして確かな倫理観が必要となる.

 しかし,twitterなどのSNSやブログでは,悪ふざけの投稿や恐喝まがいの投稿,犯罪をしたことを自慢するかのような投稿をするユーザーがいる.私はこれらの投稿をする原因を「内輪ルール優位性」によるものだと考える.内輪ルール優位性とは,本人たちは悪いとわかっていてもみんなやっているから自分もするべきだという感覚に陥ることである.そして,そのユーザーが問題を起こした後,武勇伝的な感覚でSNSやブログに投稿をする.さらに,何かのきっかけで悪ふざけの度合いが増すと,その暴走は止まらず,罪の意識が薄れていく\cite{a}.

 それに対し,悪ふざけや犯罪を自慢する投稿,情報モラル,情報リテラシーが低い投稿を見過ごさず,通報やリツイートにより同じ過ちを繰り返さないようにする正義感溢れる人達がいる.そこでは私はそのような正義感溢れる人たちはどのような人間なのかを本研究で解明したいと考えた.





\section{研究の目的}

研究の目的は,悪ふざけのツイートや犯罪ツイート,情報モラル,リテラシーの低いツイートに対し,多くのリツイートをしているユーザーの特定をすることである,




\section{プロジェクトマネジメントとの関連}

本研究はプロジェクトマネジメントにおける10個の知識エリアのうち,リスクマネジメントに関連付けられる.犯罪を自慢するツイートに対し,通報やリツイートすることにより,さらなる罪を犯す可能性を抑止している為である.







\section{研究の方法}

本研究はtwitterを使用しデータの集計をする.集めるデータは炎上しているツイートのidとそのツイートをリツイートしているユーザーの内部idである.データを集めるためubuntuとtwitterAPIを使用する.段階は2段階に分かれる.

\begin{enumerate}
 \item ubuntuでtwitterAPIを用いて炎上のデータを集める
 \item 集めたデータをExcleに表示し,分析する
\end{enumerate}


twtterで炎上を見つけ,炎上のidとツイートしたユーザーのidをデータベースに格納し,集めたデータをExcleに表示する.Excleにデータを格納した後,Excleのcount機能を使用し多くの炎上をリツイートしているユーザーを特定する.








\section{現在の進捗状況}

ubuntuでtwitterAPIを使用することができるプログラムを作成した.さらに作成したプログラムを使用し5つの炎上のデータを集めて統計を行った.

twitterAPIは1つの炎上につき,炎上をリツイートしたユーザーのidを最大で100件までしか取得することができない.そのため,本研究では多くの炎上をリツイートしているユーザーの特定はすることができなかった.多くの炎上をリツイートしているユーザーを特定するためには,大量の炎上を用意する必要がある.多くの炎上をリツイートしているユーザーを特定するため,本研究では炎上の定義をさらに細かく設定し,炎上を自動で取得できるプログラムの開発を行う必要がある.

多くの炎上ツイートの情報を集める際,データ項目数を増やし多くの炎上ツイートをリツイートしているユーザーの共通点を見つける.




\section{今後の計画}

現在のプログラムでは全自動でデータを集めることはできていない.そのためデータの量が少なく,多くの炎上ツイートをリツイートしているユーザーの特定に至ることはできなかった.データを増やすことにより,多くの炎上ツイートをリツイートしているユーザーの特定をすることができると仮定し,以下のように研究を進める計画である.
\begin{enumerate}
 \item 炎上ツイートの定義をより細かく正確なものにし、自動で炎上ツイートを取得できるようにする.
 \item 炎上ツイートしているツイートの多くをリツイートしているユーザーを特定し,集めるデータ項目を増やしそのユーザーの共通点を見つける研究を行う.
 \item 論文を執筆を行う.
\end{enumerate}



\bibliographystyle{junsrt}
\bibliography{biblio}%「biblio.bib」というファイルが必要.

\end{document}

