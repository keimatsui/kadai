%課題研究レジュメテンプレート ver. 1.2

\documentclass[uplatex]{jsarticle}
\usepackage[top=20mm,bottom=20mm,left=20mm,right=20mm]{geometry}
\usepackage[T1]{fontenc}
\usepackage{txfonts}
\usepackage{wrapfig}
\usepackage[expert,deluxe]{otf}
\usepackage[dvipdfmx,hiresbb]{graphicx}
\usepackage[dvipdfmx]{hyperref}
\usepackage{pxjahyper}
\usepackage{secdot}

\makeatletter
  \renewcommand{\section}{%
    \if@slide\clearpage\fi
    \@startsection{section}{1}{\z@}%
    {\Cvs \@plus.5\Cdp \@minus.2\Cdp}% 前アキ
    {.5\Cvs \@plus.3\Cdp}% 後アキ
    %{\normalfont\Large\headfont\raggedright}}
    {\normalfont\raggedright}}

  \renewcommand{\subsection}{\@startsection{subsection}{2}{\z@}%
    {\Cvs \@plus.5\Cdp \@minus.2\Cdp}% 前アキ
    {.5\Cvs \@plus.3\Cdp}% 後アキ
    %{\normalfont\large\headfont}}
    {\normalfont}}

  \renewcommand{\subsubsection}{\@startsection{subsubsection}{3}{\z@}%
    {\Cvs \@plus.5\Cdp \@minus.2\Cdp}%
    {\z@}%
    %{\normalfont\normalsize\headfont}}
    {\normalfont}}
\makeatother
%ここから上を編集する必要はない.





\title{\vspace{-14mm}ディープラーニングを用いたウェブサイトデザインの年代推定}
\author{PMコース 矢吹研究室 1442104 増田準}
\date{}%日付を入れる必要はない.
\pagestyle{empty}%ページ番号は振らない.
\begin{document}
\maketitle





\section{研究の背景}

Webサイトにおいてデザインは時代とともに変化を繰り返すものであり,インターネットが身近な存在となった現代では特に重要視すべきだ.こもりまさあき氏による「Webデザインの新しい教科書\cite{bib002}」において,「言語や配信の仕組みに変わりはないが,広い意味での『Webデザイン』という行為は,その時々に合わせて変化し続けている.」とあるように,その時代のユーザーの嗜好や傾向に沿ったWebサイトが求められている.
また,西垣通氏監修による「ユーザーが作る知の形\cite{bib001}」において,「ネット界は多並行分散型のネットワークになっているので,より多様化を進める方向でウェブという市場は推移する.」とあるように,流行の変化に適応することがウェブ運営にとっても重要だと考えることができる.






\section{研究の目的}

この研究では,ディープラーニングによってWebサイトのデザインからその年代を推定したい.
Webデザインには年代ごとのトレンドが存在する.2016年のトレンドパターンは11個だと言われている.例を挙げると,情報整理がしやすくデバイスを問わず動作が可能な「カード型のデザイン」や,ユーザーが直感的に移動させることができ,スクロール,クリック,時間経過にも対応した「フルスクリーンスライド」,更には,ヘッダーに映画のような高解像度の動画を用い,Webデザインは映画製作のようになるだろうとも言われている「ヒーロービデオヘッダー」等がある\cite{bib004}.
この様に一口にトレンドと言っても,その様々なパターンがある.しかし利用者は,アクセスしたページから,新しい・古いや,良い・悪いなどを漠然と感じることができる.この感覚が流行を生むと考え,人間の神経経路を基にしたアルゴリズムであるディープラーニングを用いれば,それを機械的に数値で表すことができると考えた.





\section{プロジェクトマネジメントとの関連}

この研究によってWebデザインのトレンドを証明することができれば,Webサイト運営においてデザインがトレンドに合致しているかどうかを数値的に判断することもできる.
松本耕太氏による課題研究「玩具開発プロジェクトのためのデータマイニング手法\cite{bib003}」において,「データマイニングを利用してヒットの要因を把握する技術は,プロジェクトの新規性を見出す方法のひとつとなる.」とあるように,この研究にはPMとの関係性もあるといえる.





%\begin{wrapfigure}[行数]{r}{幅}%行数はオプションだが,調整しないとうまくいかない.

%\includegraphics[width=図の幅,clip]{ファイル名}\label{参照用ラベル}


\section{研究の方法}

\subsection{ライブラリについて}

この研究ではディープラーニングを用いた画像解析用ライブラリであるCaffeを利用する。多くのディープラーニング用ライブラリの中でCaffeを選んだ理由は,ディープラーニングの歴史の中では老練なライブラリであり,開発コミュニティの動きも活発でサンプルコード等も多く初心者向きであると判断したためである.



\subsubsection{手法}

画像解析にはトレーニング用とテスト用の2種類の画像を用意する必要がある.トレーニング用の画像としてインターネット・アーカイブに保存されたページをキャプチャする.その際,AlexaというWebサイト分析サービスに基づき,世界的にアクセス数の多いサイトであるものを保存対象の基準とする.保存された複数の画像を年代別でタグ付けし,学習させる.その後,トレーニング用の画像とは別のページのテスト用画像を用意し,タグによって分類させる.





\subsubsection{CNNについて}

Caffeに学習させた画像は,CNN(Convolution Neural Network)という画像解析に適したモデルで認識される.CNNは設定されたレイヤー構成を基に,画像を小領域(フィルタ)に絞り込む.この領域をから得られる特徴から畳み込みを行い,画像全体の特徴をとら耐えることができ,高精度での学習が可能となる.







\section{現在の進捗状況}

%\begin{wraptable}[行数]{r}{幅}%行数はオプションだが,調整しないとうまくいかない.
\begin{wraptable}[5]{r}{6cm}
 \vspace*{-\intextsep}
 \caption{MNIST結果}\label{サンプル表}
  \begin{tabular}{|l|r|r|}
   \hline
   ライブラリ名 & 学習結果 & 所要時間 \\
   \hline
   \hline
   Caffe & 99,04\% & 約30分\\
   \hline
   Tensorflow & 91,59\% & なし\\
   \hline
  \end{tabular}
\end{wraptable}

\subsection{MNISTの学習}

MNISTという手書き数字の画像が集められたデータセットがある.これをサンプルとし,Caffeでの動作確認を行った.また,比較として主要なディープラーニング用ライブラリであるTensorflowを同じ環境下で動作させ,MNISTを解析させた.結果は表1のとおりである.
Caffeの正解率は99.04%で,学習には30分を要した.Tensorflowの正解率は91.59%で,学習には時間を要さなかった.
Tensorflowの学習所要時間は,Pythonの使用によりレイヤ等の設定を手打ちした後の実行である.総合的な所要時間は,設定の細かな理解を要するため,より多く必要であると感じた.




\subsection{Webデザイン解析}
Amazon・Google・Microsoft・Wikipedia・Yahoo・YouTubeのページをトレーニング画像として使用する.インターネット・アーカイブに保存されているページから計56枚をキャプチャし,学習させた.しかし,プログラムを三日間動かし続けたが,一向に学習は終わらなかった.理由として考えられるのは,Webページのキャプチャであるため,サイズが大きすぎたこと.GPUを使用せずCPUのみで学習したため,処理に時間がかかったことが挙げられる.


\section{今後の計画}

以下のように研究を進める計画である.

\begin{enumerate}
\item レイヤーの設定を学び,学習画像のサイズ設定を変える.
\item 動作環境にGPUを導入する.
\item 検証用データを与え,正解率を出す.
\item その後,必要に応じて学習画像の追加等を行う.
\end{enumerate}

\bibliographystyle{junsrt}
\bibliography{biblio}%「biblio.bib」というファイルが必要.

\end{document}
