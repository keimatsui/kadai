%卒論概要テンプレート ver. 3.0

\documentclass[uplatex,twocolumn,dvipdfmx]{jsarticle}
\usepackage[top=22mm,bottom=22mm,left=22mm,right=22mm]{geometry}
\setlength{\columnsep}{10mm}
\usepackage[T1]{fontenc}
\usepackage{txfonts}
\usepackage[expert,deluxe]{otf}
\usepackage[dvipdfmx,hiresbb]{graphicx}
\usepackage[dvipdfmx]{hyperref}
\usepackage{pxjahyper}
\usepackage{secdot}





%タイトルと学生番号,名前だけ編集すること
\title{\vspace{-5mm}\fontsize{14pt}{0pt}\selectfont リツイート数と売り上げの相関関係}
\author{\normalsize ソフトウェアコース 矢吹研究室 1442043 川崎貴雅}
\date{}
\pagestyle{empty}
\begin{document}
\fontsize{10.5pt}{\baselineskip}\selectfont
\maketitle





%以下が本文

\section{序論}
Twitterは2006年に開始したソーシャル・ネットワーキング・サービス(SNS)である.
日本語版が利用可能となったのは2008年4月からである.
Twitterはツイートと呼ばれる140字以内の短い文字列を投稿するサービスである.自分以外のユーザーのツイートを読むためには,そのユーザーのページにアクセスする方法以外に,そのユーザーをフォローすることでツイートを読むことができる.\nocite{bib002}
その他の機能にリツイート呼ばれるものがあり,他のユーザーの投稿を自分のタイムライン上に再投稿すること.自身のフォロワーにそのツイートを見せることができる.
またリツイート以外にも気に入った投稿にいいね(ハートマーク)を付けることができる.
Twitterを見ていると企業の広告や商品の紹介ツイートが流れてくることがある.主観的だが一時期に比べ商品の紹介ツイート等が増えたように感じた.
Twitterでの商品紹介はリツイートなどによる広告の拡散を目的としているのではないかと予測できるが,実際にリツイートやいいね(ハートマーク)を送られることと売り上げ上昇に関連があるのかは断言ができない.
しかし売れている商品はリツイートやいいね(ハートマーク)が多いというイメージが強いならば定量的な結果をだしてリツイートなどと売り上げの関係を調べてみればわかるのでないかと考えました.

\section{目的}

売れている商品や人気商品がリツイートやいいね(ハートマーク)の数に関係があるのかグラフや数値で判断できるようにする.
またリツイートと売上の関係をグラフで出しパターンがないかを調べる.

\section{手法}
TwitterAPIで300リツイート以上のツイートを集め,その中から人気商品を取り出しリツイート数やいいね(ハートマーク)数をエクセルに記入する.次に人気商品の売り上げを記入します.
出来たファイルをCSVファイルにしてRで解析して売上とリツイートの間に相関関係がないかを出します.
エクセルに記入したデータからグラフを作りパターンがないか確認する.

\section{関連}
今回の研究で相関関係が見られた場合,公式アカウントでのリツイート数などから飲料水なら生産数を調整などの判断材料となるため,コストマネジメントやリスクマネジメントとの関連が見られると考えられる.
またグラフからパターンを見出すことができた場合も同様に,リスクマネジメントにおける定量的なリスク分析に活用できると考えられるため,リスクマネジメントとの関連全体を通して深いと考えられる.
他にも分析などから課題が見つかればステークホルダーとの対話などにもつながるためステークホルダーマネジメントとの関連も考えられる.\nocite{bib001}\nocite{bib002}\nocite{bib003}

\nocite{bib001}
\nocite{bib003}

\bibliographystyle{junsrt}
\bibliography{biblio}%「biblio.bib」というファイルが必要.

\end{document}
