%卒論概要テンプレート ver. 3.0

\documentclass[uplatex,twocolumn,dvipdfmx]{jsarticle}
\usepackage[top=22mm,bottom=22mm,left=22mm,right=22mm]{geometry}
\setlength{\columnsep}{10mm}
\usepackage[T1]{fontenc}
\usepackage{txfonts}
\usepackage[expert,deluxe]{otf}
\usepackage[dvipdfmx,hiresbb]{graphicx}
\usepackage[dvipdfmx]{hyperref}
\usepackage{pxjahyper}
\usepackage{secdot}





%タイトルと学生番号,名前だけ編集すること
\title{\vspace{-5mm}\fontsize{14pt}{0pt}\selectfont MBTIを用いたメンバ間のリスク特定}
\author{\normalsize プロジェクトマネジメントコース 矢吹研究室 1442085 中村真悟}
\date{}
\pagestyle{empty}
\begin{document}
\fontsize{10.5pt}{\baselineskip}\selectfont
\maketitle





%以下が本文
 プロジェクトをマネジメントする上で必ずと言っていいほどリスクが付きまとう.システムの仕様変更やメンバ間の情報伝達など,リスクの多くは人間関係から起こっているものといっても過言ではない.プロジェクトのメンバがほぼ毎回異なるためである.それを改善するにはどうしたらいいか.

 一つはチームメンバを固定してしまうことである.当然そうすることができれば苦労はしないだろうが,プロジェクトの定義上困難な話である.では,プロジェクトマネージャがより早くメンバのことを人間的特徴を把握することができればどうだろうか.

 ユングが提唱した類型論というものがある.外交的か内向的かの2型をもとに,思考型,感情型,感覚型,直感型の4型に細かく分類される.それを自己理解のために発展させたのがMBTI(Myers-Briggs Type Indicator)という.

 MBTIはアセスメントツールではなく,自己肯定を促すためのものである.従来,心理学の観点では人の欠点を理解し,そこを補うという考えが主流だった.だがMBTIはその人の良い点をより強化していこうという考えである.具体的には
\begin{itemize}
 \item 内向:I・外交:E
 \item 感覚:S・直感:N
 \item 思考:T・感情:F
 \item 判断的態度:J・知覚的態度:P
\end{itemize}
の4指標の組み合わせで16タイプに分類される.当然この16タイプで同じだからと言っても性格がすべて同じというわけではない.この技法はキャリアカウンセリングやリーダーシップ開発,チームビルディングなどに使われることが多い.

 このMBTIを用いて,プロジェクトのメンバの大まかな性格を理解し,メンバが原因となって起きる事象の予測を行うことはできないかと考える.実際にチームビルディングにMBTIを用いる場合は個々の強みや弱みを把握するためである.早い段階でメンバの強みや欠点を把握することができればプロジェクト・マネジメントに活かすことができるのではないか.メンバの強みと弱みが分かるのならば、チームの強みと弱みもおのずと把握することができるだろう。

 研究方法としては,プロジェクト開始時にMBTIの性格検査を行い,メンバのタイプを把握する.その後,チーム内で実際に起きた事象を記録し,16タイプごとの強みと弱みとチームに起こった事象とでどのような関連があるのかを調査する.

 調査の結果は,PMBOKにおけるリスク・マネジメント,人的資源マネジメント,コミュニケーションマネジメントに関連する.目的の事象の予測は,リスク特定でありリスク・マネジメントに該当している.MBTIは自己理解メソッドであり、自己を理解することは人材の育成を促すこともできるので人的資源マネジメントに該当する。また,MBTIはチームビルディングに用いることができる.さらにはメンバ間の円滑なコミュニケーション向上が期待できるため,コミュニケーションマネジメントに関連しているといえる\nocite{BB19543658}\nocite{110003745117}\nocite{ylab2015}\nocite{katolab2015}.

\bibliographystyle{junsrt}
\bibliography{biblio}
\end{document}
