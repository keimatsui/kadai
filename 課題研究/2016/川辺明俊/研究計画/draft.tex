%卒論概要テンプレート ver. 3.0

\documentclass[uplatex,twocolumn,dvipdfmx]{jsarticle}
\usepackage[top=22mm,bottom=22mm,left=22mm,right=22mm]{geometry}
\setlength{\columnsep}{10mm}
\usepackage[T1]{fontenc}
\usepackage{txfonts}
\usepackage[expert,deluxe]{otf}
\usepackage[dvipdfmx,hiresbb]{graphicx}
\usepackage[dvipdfmx]{hyperref}
\usepackage{pxjahyper}
\usepackage{secdot}





%タイトルと学生番号,名前だけ編集すること
\title{\vspace{-5mm}\fontsize{14pt}{0pt}\selectfont 各オープンソースゲームエンジンによるSWOT分析}
\author{\normalsize プロジェクトマネジメントコース 矢吹研究室 1442045 川辺 明俊}
\date{}
\pagestyle{empty}
\begin{document}
\fontsize{10.5pt}{\baselineskip}\selectfont
\maketitle





%以下が本文
\section{背景}
ゲームエンジンとは,コンピュータゲームのソフトにおいて,共通して用いられる主要な処理を代行し効率化するソフトウェアの総称である.現在のハイエンドゲームは高度なグラフィックや物理演算を求めるため,作るのに一から再現するためには多くのコストや時間が掛かりすぎてしまうが,その作業を軽減化するのがゲームエンジンである.\\
現在のゲームエンジンの主流は2つある.1つは自分の会社でゲームを開発するために使う自社開発ゲームエンジンである.これは初めにゲームエンジンを開発する際に,時間とコストが掛かかり開発中はゲームエンジン自体では収益を上げることはできない.利点としては,ゲームのソフトウェアを製作の際,問題が発生しても自分たちで作ったエンジンのため問題に対処しやすい環境になることである.他にも,独自開発のエンジンなので独自のグラフィックや物理演算の特徴が出るなどの,ゲームエンジンによってもたらされる他の会社が作るゲームソフトとの差異が図れる利点がある.\\
2つ目は,オープンソースのゲームエンジンである.まずオープンソースとは,人間が理解しやすいプログラミング言語で書かれたコンピュータプログラムであるソースコードを広く一般に公開し,誰でも自由に扱ってよいとする考え方.また,そのような考えに基づいて公開されたソフトウェアのことである.そしてここで述べるゲームエンジンはソフトウェアの総称なのでオープンソースソフトフェア(以後OSS)として説明する.OSSはフリーウェアと一緒で,無償で利用できるソフトウェアである.そこで何が違うのかというとそれはライセンスによる「利用制限の有無」と「対価」の点である.無償で利用ができるという意味ではフリーウェアとOSSに違いはないが,利用制限についてはフリーウェアは作者の要望に応じて過度な条件が課されていることが多い.またフリーウェアではプログラミングのソースコードの入手・改変と再配布が禁止されていることから修正版・変更版の配布が難しいのも大きな違いだ\cite{matsumoto}.\\
現在世界中でOSSを活用してゲーム開発が行われている.4割のシェア率を誇るUnityを筆頭にUnreal EngineやAmazonが今後展開していくLumberyard,日本産のOROCHIがあります.\\
\section{目的}
今後,大手の企業も自社ゲームエンジンではなくオープンソースエンジンも利用し,中小企業ももっとクオリティの高いゲームやスマフォゲームを作るためにオープンソースエンジンを活用するため,各ゲームエンジンのSWOT分析があれば効率よくゲームソフトの開発に臨むことができる.
\section{手法}
ゲームソフトの情報がまとめられているwebサイトからOSSを使用しているゲームソフト情報を抽出して,数が多すぎればそこから適当な数のゲームソフトを選び,選んだゲームソフトの売上本数や開発までにかかった時間やコストの情報から特徴を調べSWOT分析する.






\bibliographystyle{jplain}
\bibliography{biblio}


\end{document}


