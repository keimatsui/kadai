%卒論概要テンプレート ver. 3.0

\documentclass[uplatex,twocolumn,dvipdfmx]{jsarticle}
\usepackage[top=22mm,bottom=22mm,left=22mm,right=22mm]{geometry}
\setlength{\columnsep}{10mm}
\usepackage[T1]{fontenc}
\usepackage{txfonts}
\usepackage[expert,deluxe]{otf}
\usepackage[dvipdfmx,hiresbb]{graphicx}
\usepackage[dvipdfmx]{hyperref}
\usepackage{pxjahyper}
\usepackage{secdot}





%タイトルと学生番号,名前だけ編集すること
\title{\vspace{-5mm}\fontsize{14pt}{0pt}\selectfont 株価の上昇条件}
\author{\normalsize プロジェクトマネジメントコース 矢吹研究室 1442020 大木崇雅}
\date{}
\pagestyle{empty}
\begin{document}
\fontsize{10.5pt}{\baselineskip}\selectfont
\maketitle





%以下が本文
\section{研究の背景}

1999年10月からインターネット上で株式のオンライントレードが可能になってから,株式がとても簡易に取引できるようになり,普及が進んだ.株とは企業の資金調達のために発行している有価証券のことで,企業は投資家が株を買ってくれた資金を使って事業を拡大する.株を買うことは,株を発行している企業に出資して事業資金を提供していることを意味する.株価は企業が今後大きな利益を上げると期待できる要因があると大きく上場する.投資家は,今後の企業の成長性などに期待して株を買うが,その期待値の一つとして上がるのが,経常利益である.なぜなら経常利益は売り上げからコストを差し引いrた営業利益に,財務活動などの損益を加えた数値であり,企業全体の強さが色濃く表れているからである.小規模企業でも,経常利益4億円を実現できれば,東京証券取引所や大阪証券取引所の市場2部への上場の資格を得ることができる.\cite{BA67886013}


四季報に載っている経常利益の増減率と,株価の上がり下がりから,企業の株価が予測できるのではないかと思った事が本研究の背景だ.

過去の四季報から,企業の経常利益がどれほどあるのかわかる.そして翌年と,翌々年の業績予想が載っている.経常利益が前回の決算報告より高ければ,株価が上がる可能性が高いが,次の四半期財務報告の財務情報が四季報に載っている情報と大きく違う場合がある.もちろんリーマンショックや,イギリスのEU離脱など今後どのような事件があるのかはわからないが,業績の予想があまりにも外れている会社を除いた,過去の経常利益報告にあまり差がない企業を調査したい.



\section{研究の目的}

過去の経常利益と,当時の株価の推移からこれからの株価にどう影響するかを調べる.この研究によって何の株をいつ頃購入してどのタイミングで売れば良いのかおおよその目安がつくので,株を始めたばかりの人でも手を出しやすい.



\section{プロジェクトマネジメントの関連}
新商品の開発プロジェクトにおいて,発売日が発表がされると株式市場が反応を起こす場合がある.例えば任天堂のポケットモンスターや,カプコンのモンスターハンターなどの人気ゲームの発売日が発表されると株価が上昇するなど,発表日の公表時期がタイムマネジメントに関連している.

\section{研究方法}

以下の手順で研究を進める.
\begin{itemize}
 \item ①過去の四季報を調べて各株式会社から出されている翌年度の経常利益予測と,実際に発表されたの経常利益をExcelで表にして予測との差が少ない企業を厳選する
 \item ②予測との差が少ない企業の中で,さらに現時点での営業利益予想が前年よりも増加している企業をリストアップする.
 \item ③その企業の株価は上昇する可能性が高い.その中で株価が上昇した企業と,下降した企業の割合を求める.
\end{itemize}
\nocite{yabuki2013}\nocite{yabuki2014}
\bibliographystyle{junsrt}
\bibliography{biblio}
\end{document}