%卒論概要テンプレート ver. 3.0

\documentclass[uplatex,twocolumn,dvipdfmx]{jsarticle}
\usepackage[top=22mm,bottom=22mm,left=22mm,right=22mm]{geometry}
\setlength{\columnsep}{10mm}
\usepackage[T1]{fontenc}
\usepackage{txfonts}
\usepackage[expert,deluxe]{otf}
\usepackage[dvipdfmx,hiresbb]{graphicx}
\usepackage[dvipdfmx]{hyperref}
\usepackage{pxjahyper}
\usepackage{secdot}





%タイトルと学生番号,名前だけ編集すること
\title{\vspace{-5mm}\fontsize{14pt}{0pt}\selectfont 株価上昇条件の調査と実証}
\author{\normalsize プロジェクトマネジメントコース 矢吹研究室 1442020 大木崇雅}
\date{}
\pagestyle{empty}
\begin{document}
\fontsize{10.5pt}{\baselineskip}\selectfont
\maketitle





%以下が本文

株取引には大金が動くため株で資産を失い,路頭に迷うような大損をしてしまった人がいる一方で,億万長者になった者もいる.

株とは企業が事業資金を調達するために発行している有価証券のことだ.企業は,投資家が株を買ってくれた資金等を使って事業を拡大する.「株を買うこと」は,株を発行している企業に出資して事業資金を提供していることを意味する.また,株の大きな特徴として買った株を第三者へ転売することができるという点がある. 株券は企業が発行するが,その株は企業から直接売ってもらえるわけではない.株の取引は証券取引所で行われさらに証券取引所と投資家を仲介する証券会社の存在がある.近年インターネット技術が進み,各家庭にコンピューターが普及したと同時に株取引もわざわざ証券取引所まで足を運ぶことなく自宅のインターネットに繋がったパソコン上で株取引ができるようになった.取引ができるようになっただけでなく,過去の株価の推移や各企業の四季報のデータなどもネット上で見られるようになった.そこで私は過去数年分の膨大なデータから,過去に株価が上昇した原因を探して,今後上昇する株価を予測することを課題研究のテーマにしたい.

株価が急に下がったり上がったりする時には当時の政治的背景に左右されやすいが,主に企業の不正の発覚,異物混入,社員の大量リストラなどの世間を騒がせるニュースが起きている.

株で利益を上げるには大きく分けて2つの要素がある.1つは銘柄だ.上昇する株価の銘柄の特徴がわかればその企業の株券を持っている限りいつかは利益を得られる.しかしそれでは終わりのないマラソンと同じでほぼ毎日株価をチェックしていないといけなくなり精神的にやつれてしまう.もう一つ必要な要素がタイミングだ.株価がいつ上がるのか知ることができれば時間も投資金も無駄が少なくなり,利益を上げられるのではないだろうか.ネット上に残っている株価の推移から大きく下がっている時や大きく上がっている時の背景を考察して当時何が起こって株価が急沸したの原因を列挙していき上昇する条件を調べて将来上がりそうな銘柄を探す.予測結果と,実際の株価の整合性を高めるため,記録したデータをグラフにまとめる.

今回達成したい目標は大きく分けて2つだ.1つめは株価の取得ツールを用いて,株価のデータを集計して株価が上昇する企業の共通条件を発見してパターン化すること.2つめはどのような条件なら株価が上がるのか大まかな予測ができて,利益を上げられるようになることだ.

\nocite{BB19543658}
\nocite{BA78827129}
\nocite{122222}

\bibliographystyle{jplain}
\bibliography{biblio}


\end{document}