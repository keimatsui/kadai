%卒論概要テンプレート ver. 3.0

\documentclass[uplatex,twocolumn,dvipdfmx]{jsarticle}
\usepackage[top=22mm,bottom=22mm,left=22mm,right=22mm]{geometry}
\setlength{\columnsep}{10mm}
\usepackage[T1]{fontenc}
\usepackage{txfonts}
\usepackage[expert,deluxe]{otf}
\usepackage[dvipdfmx,hiresbb]{graphicx}
\usepackage[dvipdfmx]{hyperref}
\usepackage{pxjahyper}
\usepackage{secdot}

%タイトルと学生番号,名前だけ編集すること
\title{\vspace{-5mm}\fontsize{14pt}{0pt}\selectfont 課題研究の計画}
\author{\normalsize プロジェクトマネジメントコース 矢吹研究室 1442068 鈴木 博文}
\date{}
\pagestyle{empty}
\begin{document}
\fontsize{10.5pt}{\baselineskip}\selectfont
\maketitle

%以下が本文

今,インターネット普及の結果として,検索ワード,ネットショッピングの購買履歴,ツイッターのようなソーシャルメディアワーキングサービス(SNS)のメッセージ,スマートフォンの位置情報など,膨大なデータが蓄積されている.これらは「ビッグデータ」と呼ばれ,その利活用によって経済活動や社会活動に変革をもたせると大きな期待が寄せられている.

インターネットの浸透と平行して,社会インフラ,医療,ヘルスケア,交通,農業などのあらゆる領域がセンサーに覆われ,膨大な電子データの集積,解析が可能になりつつある.このビッグデータ時代の到来により,解析基盤である分散処理技術,データベース技術が進化し,さらには,データ解析手法にも新たな変化が生じた\cite{bib1}.

ビッグデータの中でも,個々人が容易に情報の発信を行うことが出来るソーシャルメディアワーキングサービスのデータを用いたデータマイニングが行えるのではないかと考えた.

マイクロブログサービスが普及している現在,膨大な情報がインターネット上に投稿されている.その中でもTwitterは140文字以内のツイートと呼ばれる短文を投稿する為,他のSNSやブログと違い手軽にツイートを投稿することが可能となっている.その為,Twitterには今起こったことや感じたことを気軽に投稿している場合が多い.そして,このようなツイートにはユーザのその時々の感情が現れている場合が多い.\cite{bib2}

ツイッターの投稿の中で炎上という種類のツイートを,時折見かけることがある.日常的に悪質なツイートがされることに対し,悪ふざけや犯罪を自慢するツイート,情報モラル,情報リテラシーが低いツイートを見過ごさず,通報やリツイートをする正義感溢れる人達がいる.彼らはそれ相応の罰を受ける必要があるという気持ちや何度も同じ過ちを繰り返してしまわないようにという正義感から通報やリツイートをする.リツイート数が伸びると便乗してリツイートするユーザが増え,結果事態が大きくなり炎上してしまう場合がある.\cite{bib3}.

ひとつのツイートに対し人が抱く感情は様々である.喜ぶ,怒る,悲しむ,楽しむなど捉え方次第で感じることも全く異なるだろう.それを踏まえて私は,インシデントが発生した際に人々が抱く感情にはどのようなものが多いのか,時間軸で表すとどのような変化を示すのかを調査したいと感じた.

日常で頻繁に発生し,人々の生活に深く影響するインシデントに「列車の遅延」が思い当たった.列車が遅延すると,人々はどのような感情を抱くのだろうか.重要な予定にも遅延が発生する影響から怒りの感情が湧くかもしれない,あるいは講義や授業に影響が発生し休校などの措置を取らざるをえない状況になり生徒や学生の一部は喜びを感じるかもしれない.また,地域的特性についても様々な結果が現れるのではないかと考える.東京都民の性格と千葉県民の性格で抱く感情に差異が出るかもしれない.それに加え,インシデントの発生時間が早朝か深夜かによっての調査もしたいと考える.

この調査の結果はPMBOKにおけるリスク・マネジメントに結びつくものと考える.調査の結果路線ごとにインシデントが発生する特徴的な時間や場所,それがわかれば発生を防ぐ対応策を講じることができリスクを回避・軽減できるからだ.また,ある路線の利用者は遅延に敏感に反応することがわかれば,駅構内の放送を即座に入れるなど利用者の機嫌を損ねないよう工夫することも出来るかもしれない.

このように,ビッグデータの中からTwitterというSNSから得た情報を元に,鉄道輸送における遅延発生時のリスク軽減策を調査することを課題研究の企画としたいと考える.


\bibliographystyle{junsrt}
\bibliography{biblio}%「biblio.bib」というファイルが必要.

\end{document}
