%課題研究レジュメテンプレート ver. 1.2

\documentclass[uplatex]{jsarticle}
\usepackage[top=20mm,bottom=20mm,left=20mm,right=20mm]{geometry}
\usepackage[T1]{fontenc}
\usepackage{txfonts}
\usepackage{wrapfig}
\usepackage[expert,deluxe]{otf}
\usepackage[dvipdfmx,hiresbb]{graphicx}
\usepackage[dvipdfmx]{hyperref}
\usepackage{pxjahyper}
\usepackage{secdot}

\makeatletter
  \renewcommand{\section}{%
    \if@slide\clearpage\fi
    \@startsection{section}{1}{\z@}%
    {\Cvs \@plus.5\Cdp \@minus.2\Cdp}% 前アキ
    {.5\Cvs \@plus.3\Cdp}% 後アキ
    %{\normalfont\Large\headfont\raggedright}}
    {\normalfont\raggedright}}

  \renewcommand{\subsection}{\@startsection{subsection}{2}{\z@}%
    {\Cvs \@plus.5\Cdp \@minus.2\Cdp}% 前アキ
    {.5\Cvs \@plus.3\Cdp}% 後アキ
    %{\normalfont\large\headfont}}
    {\normalfont}}

  \renewcommand{\subsubsection}{\@startsection{subsubsection}{3}{\z@}%
    {\Cvs \@plus.5\Cdp \@minus.2\Cdp}%
    {\z@}%
    %{\normalfont\normalsize\headfont}}
    {\normalfont}}
\makeatother
%ここから上を編集する必要はない.





\title{\vspace{-14mm}鉄道輸送におけるインシデント発生時の対応策提案}
\author{PMコース 矢吹研究室 1442068 鈴木 博文}
\date{}%日付を入れる必要はない.
\pagestyle{empty}%ページ番号は振らない.
\begin{document}
\maketitle


\section{研究の背景}
今日,インターネットが普及するに伴って,検索ワード,ネットショッピングの購買履歴,TwitterのようなSNSのメッセージ,スマートフォンの位置情報など,膨大なデータが蓄積されている.これらは「ビッグデータ」と呼ばれ,その利活用によって経済活動や社会活動に変革をもたせると大きな期待が寄せられている.

インターネットの浸透と並行して,社会インフラ,医療,ヘルスケア,交通,農業などのあらゆる領域において膨大な電子データの集積,解析が可能になりつつある.このビッグデータ時代の到来により,解析基盤である分散処理技術,データベース技術が進化し,さらにはデータ解析手法にも新たな変化が生じた\cite{bib1}.

ビッグデータを利用したデータマイニングを行い,その結果を社会に柔軟に適応していくことによって,わたしたちの生活はより良いものになっていくと考える.

多くのマイクロブログサービスが普及している現在,膨大な情報がインターネット上に投稿されている.その中でもTwitterは140文字以内のツイートと呼ばれる短文を投稿する為,他のSNSやブログと違い手軽にツイートを投稿することが可能となっている.その為,Twitterには今起こったことや感じたことを気軽に投稿している場合が多い.そして,このようなツイートにはユーザのその時々の感情が現れている場合が多い\cite{bib2}.

Twitterの投稿の中で炎上という種類のツイートを,時折見かけることがある.日常的に悪質なツイートがされることに対し,悪ふざけや犯罪を自慢するツイート,情報モラル,情報リテラシーが低いツイートを見過ごさず,通報やリツイートをする正義感溢れる人達がいる.彼らはそれ相応の罰を受ける必要があるという気持ちや何度も同じ過ちを繰り返してしまわないようにという正義感から通報やリツイートをする.リツイート数が伸びると,便乗してリツイートするユーザが増え,結果事態が大きくなり炎上してしまう場合がある\cite{bib3}.

しかし,このひとつのツイートに対し人が抱く感情は様々である.喜ぶ,怒る,悲しむ,楽しむなど捉え方次第で感じることも全く異なるだろう.それを踏まえてわたしは,インシデントが発生した際に人々が抱く感情にはどのようなものが多いのか,時間軸と地域によっての分布を表すとどのような変化を示すのかを調査したいと感じた.

調査の対象を考える中で,日常生活の中で比較的頻繁に発生し,人々の生活に深く影響するインシデントに「列車の遅延」が思い当たった.列車が遅延すると,人々はどのような感情を抱くか.もちろん遅延が発生する影響から怒りの感情が湧くという人が大多数となることは想像できる.しかし,講義や授業に影響が発生し休校などの措置を取らざるをえない状況になり生徒や学生の一部は喜びを感じるかもしれない.また,地域的特性についても様々な結果が現れるのではないかと考える.東京都民の性格と千葉県民の性格で抱く感情に差異が出るかもしれない.それに加え,インシデントの発生時間が早朝か深夜かによっての調査もしたいと考える.



\section{研究の目的}
路線ごとに,インシデントが発生する特徴的な時間や場所を発見する.また,路線別のインシデント発生時の利用者の感情も発見する.この結果を利用して,インシデント発生時における各所において最も最善な対応策を提案する.仮に,ある路線の利用者はインシデントに敏感に反応することがわかれば,駅構内の放送を即座に入れるなど利用者の機嫌を損ねないようにするなどを対応策とする.



\section{プロジェクトマネジメントとの関連}
この研究はPMBOKにおけるリスク・マネジメントに結びつくものと考える.調査の結果として,事前に発生を防ぐ対応策を現状以上に講じることができ,リスクを回避・軽減できる可能性が高まるからだ.


\section{研究の方法}
TwitterAPIを利用し,インシデント発生時の当該路線に関するツイートを収集する.検索の方法は「中央総武線」「総武快速線」などの路線名,発生の原因となる駅が明確にある場合は当該駅名においても検索を行う.また一定段階において検索の結果が不充分であると判断した場合には「総武」や「快速」などのワードでの検索を行うことも視野に入れる.

\section{現在の進捗状況}
TwitterAPIの利用を行うことを計画しているため,必要なアプリケーションの作成と動作環境の構築を行っている.また,実際に検索を行う実験を行っている.

\section{今後の計画}
検索の結果を,最も適した方法で処理できる方法を見つけ処理できるようになる.現在行っているTwitterAPIを用いた検索が調査に必要なスキルまで行えるようになったら,実際に路線名での検索を行ってみる.

\bibliographystyle{junsrt}
\bibliography{biblio}%「biblio.bib」というファイルが必要.

\end{document}
