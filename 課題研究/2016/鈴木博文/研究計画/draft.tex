%卒論概要テンプレート ver. 3.0

\documentclass[uplatex,twocolumn,dvipdfmx]{jsarticle}
\usepackage[top=22mm,bottom=22mm,left=22mm,right=22mm]{geometry}
\setlength{\columnsep}{10mm}
\usepackage[T1]{fontenc}
\usepackage{txfonts}
\usepackage[expert,deluxe]{otf}
\usepackage[dvipdfmx,hiresbb]{graphicx}
\usepackage[dvipdfmx]{hyperref}
\usepackage{pxjahyper}
\usepackage{secdot}

%タイトルと学生番号,名前だけ編集すること
\title{\vspace{-5mm}\fontsize{14pt}{0pt}\selectfont 鉄道輸送におけるインシデント発生時の対応策}
\author{\normalsize プロジェクトマネジメントコース 矢吹研究室 1442068 鈴木 博文}
\date{}
\pagestyle{empty}
\begin{document}
\fontsize{10.5pt}{\baselineskip}\selectfont
\maketitle

%以下が本文
\section{研究の背景}
今日,インターネットが普及するに伴って,検索ワード,ネットショッピングの購買履歴,TwitterのようなSNSのメッセージ,スマートフォンの位置情報など,膨大なデータが蓄積されている.これらは「ビッグデータ」と呼ばれ,その利活用によって経済活動や社会活動に変革をもたせると大きな期待が寄せられている.このビッグデータ時代の到来により,解析基盤である分散処理技術,データベース技術が進化し,さらにはデータ解析手法にも新たな変化が生じた\cite{bib1}.

わたしは,ビッグデータを利用したデータマイニングを行い,その結果を社会に柔軟に適応していくことによって,わたしたちの生活はより良いものになっていくと考える.

現在,多くのマイクロブログサービスが普及している.その中でもTwitterは140文字以内のツイートと呼ばれる短文を投稿する為,他のSNSやブログと違い手軽にツイートを投稿することが可能となっている.その為,Twitterには今起こったことや感じたことを気軽に投稿している場合が多い.そして,このようなツイートにはユーザのその時々の感情が現れている場合が多い\cite{bib2}.

Twitterの投稿の中で炎上という種類のツイートを,時折見かけることがある.日常的に悪質なツイートがされることに対し,悪ふざけや犯罪を自慢するツイート,情報リテラシーが低いツイートを見過ごさず,通報やリツイートをする正義感溢れる人達がいる.彼らはそれ相応の罰を受ける必要があるという気持ちや何度も同じ過ちを繰り返してしまわないようにという正義感から通報やリツイートをする.リツイート数が伸びると,便乗してリツイートするユーザが増え,結果事態が大きくなり炎上してしまう場合がある\cite{bib3}.

しかし,このひとつのツイートに対し人が抱く感情は様々である.喜ぶ,怒る,悲しむ,楽しむなど捉え方次第で感じることも全く異なる.調査の対象を考える中で,日常生活の中で比較的頻繁に発生し,人々の生活に深く影響するインシデントに「列車の遅延」が思い当たった.わたしは,インシデントが発生した際に人々が抱く感情にはどのようなものが多いのか,時間軸と地域によっての分布を表すとどのような変化を示すのかを調査したいと感じた.


\section{研究の目的}
路線ごとに,インシデントが発生する特徴的な時間や場所を発見する.また,路線別のインシデント発生時の利用者の感情も発見する.この結果を利用して,インシデント発生時における各所において最も最善な対応策を提案する.仮に,ある路線の利用者はインシデントに敏感に反応することがわかれば,駅構内の放送を即座に入れるなど利用者の機嫌を損ねないようにするなどを対応策とする.


\section{プロジェクトマネジメントとの関連}
この研究はPMBOKにおけるリスク・マネジメントに結びつくものと考える.調査の結果として,事前に発生を防ぐ対応策を現状以上に講じることができ,リスクを回避・軽減できる可能性が高まるからだ.


\section{研究の方法}
TwitterAPIを利用し,インシデント発生時の当該路線に関するツイートを収集する.検索の方法は「総武快速線」などの路線名,発生の原因となる駅が明確な場合は当該駅名の検索も行う.また検索の結果が不充分であると判断した場合には「総武」や「快速」などで検索も行う.


\bibliographystyle{junsrt}
\bibliography{biblio}%「biblio.bib」というファイルが必要.

\end{document}
