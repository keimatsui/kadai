%卒論概要テンプレート ver. 3.0

\documentclass[uplatex,twocolumn,dvipdfmx]{jsarticle}
\usepackage[top=22mm,bottom=22mm,left=22mm,right=22mm]{geometry}
\setlength{\columnsep}{10mm}
\usepackage[T1]{fontenc}
\usepackage{txfonts}
\usepackage[expert,deluxe]{otf}
\usepackage[dvipdfmx,hiresbb]{graphicx}
\usepackage[dvipdfmx]{hyperref}
\usepackage{pxjahyper}
\usepackage{secdot}





%タイトルと学生番号,名前だけ編集すること
\title{\vspace{-5mm}\fontsize{14pt}{0pt}\selectfont 研究タイトル}
\author{\normalsize プロジェクトマネジメントコース 矢吹研究室 1234567 氏名}
\date{}
\pagestyle{empty}
\begin{document}
\fontsize{10.5pt}{\baselineskip}\selectfont
\maketitle





%以下が本文


この本では主にインターネットの登場によって引き起こされる炎上について記述されている.まずネットの登場から後に炎上を引き起こすネットの文化的側面を要約する.ネット文化を理解するにおいて現実世界を「旧大陸」と呼び,インターネットの世界を「ネット新大陸」と例える.現実社会に居場所がない「ネット原住民」と呼ばれる人と,ネット原住民に遅れてネット新大陸に入植してきた「ネット新住民」との文化的衝突が,ネット上での軋轢を起こす.ネット原住民たちは自分達が見つけたネット新大陸に先住し,自分たちの文化と ルールをもって生活していたのだが,後から来た新住民達はそれを無視して   旧大陸の文化を持ち込もうとする.それに対して仕掛ける攻撃が炎上だ.\\
 炎上の本質とは,我がもの顔で新大陸に踏み込んできた新住民に対しての原住民の反撃であり主導権争いである.そして炎上の温床となったのが匿名掲示板である.誰が書いたのかわからないという秘匿性の高さと,簡易な双方向コミュニケーションが可能になった2ちゃんねるの登場により,悪意のある書き込みが多く見られるようになった.さらにネットはコンテンツをコミュニケーションの道具に変えてしまう構造をもつため,その勢いがブログの登場によって加速するという仕組みだ.掲示板や個人ブログは自身の考えや見解を大勢の人の目に触れさせる.これは論壇空間をネット上に提供したという事実に他ならない.\\
 ネット論壇空間は,これまで情報の最上流にいて決して批判されない存在であったマスメディアの報道に対してノーを突きつけられる存在となり,相対的にマスメディアの地位を低下させる事態を引き起こした.インターネットの世界は完全にオープンな「場」である.取材の過程や記事の内容,その記事に対する感想や批判,さらにそれらの批判に対する執筆者の反論などが混沌と混じり合い,そのままの状態で人々の前に投げ出される.\\
 ネット論壇空間の与えた影響の一例として,梅田望夫氏がネットへの失望を表明して事実上言論から撤退した事件がある.彼は作家水村美苗氏の本を自身のブログで好意的に紹介した結果批判的なコメントが寄せられた.本を読みもせずに批判的なコメントを書いた無知から生じる定見のなさについて梅田氏はバカと Twitter 上で発信した.その後梅田氏が役員を務めていた「はてなブックマーク」は大騒ぎになり,炎上状態になった.この事件はネットにおける集合知についての根深い問題を浮き彫りにした.\\
 梅田氏のブログように個々人が思い思いのままに行動した結果,集合的な動員にまで成長し,巨大な力を持つにいたるという炎上事例は,クラウドファウンディングなどの成功事例と捉えられる現象においても見出すことができる.募金や署名,抗議運動などといったネットの集合行動そのものを組織化してみんなの力を活かそうという試みだ.同じ構造をもつ現象が,場合によってプラスやマイナスに評価される二面性を持つこともある.\\
 社会集団というものは集合沸騰を通じてのみ創り出され,確認される.そのため社会集団が社会集団であるためには,その構成員が周期的に集合し,沸騰しなければならない.そのための仕組みこそが宗教であり,そのための仕組みこそが祭りである.\\
 祭りには「祭り型炎上」と,「血祭り型炎上」の2種類がある.人々が一斉にバルスと叫び,田代まさしを応援するような事を聖なるものとしてあえて祭り上げる.このような遊びを共同で執り行うことを通じて社会を創り出す祭り型炎上が創造的沸騰に当たる.一方で誹謗中傷によるバッシングを伴う攻撃的な血祭り型炎上が破壊的沸騰に当たる.祭りと血祭りは融合と分離,変容と転態を繰り返しながら激しく燃え続け,さまざまなタイプの社会像を創り出していく.\\
 つまり炎上の場とは新しい時代の新しい枠組みの中で我々が新しく社会を創り出していくうえでの迷走の場であり,紛糾の場であり暴走と逡巡と抗争の場である.

\bibliographystyle{jplain}
\bibliography{biblio}


\end{document}