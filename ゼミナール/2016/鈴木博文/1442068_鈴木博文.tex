%卒論概要テンプレート ver. 3.0

\documentclass[uplatex,twocolumn,dvipdfmx]{jsarticle}
\usepackage[top=22mm,bottom=22mm,left=22mm,right=22mm]{geometry}
\setlength{\columnsep}{10mm}
\usepackage[T1]{fontenc}
\usepackage{txfonts}
\usepackage[expert,deluxe]{otf}
\usepackage[dvipdfmx,hiresbb]{graphicx}
\usepackage[dvipdfmx]{hyperref}
\usepackage{pxjahyper}
\usepackage{secdot}

%タイトルと学生番号,名前だけ編集すること
\title{\vspace{-5mm}\fontsize{14pt}{0pt}\selectfont ネット社会が生んだ文化(要約)}
\author{\normalsize プロジェクトマネジメントコース 矢吹研究室 1442068 鈴木 博文}
\date{}
\pagestyle{empty}
\begin{document}
\fontsize{10.5pt}{\baselineskip}\selectfont
\maketitle

%以下が本文
ネット文化の歴史とは,すなわちネット特有の文化の成立過程である.ネットで展開されているすべての出来事がネット文化になるのではなく,ネットに特有でない話を連ねてもネット文化は見えてこない.そこで,ネット的とは何か.ネット的な設計とは端的に,プラットホームに依存せず,どこからでもアクセスが可能で,特別な技術の修練を必要としない,オープンな環境のことである.逆にプラットホームに依存する表現は非ネット的である.
\\ そうしたネット的な設計と運用を生み出した背景には,ネット的な態度や気分がある.インターネット普及以前からあるコンピューター文化,アメリカの「ハッカー文化」の態度や気分.ハッカー文化のさらに源流といえる1960年代の文化.それらの意識の根底にある「DIY文化」などだ.
\\ 今後のネット文化がどのようになるかは誰もわからないが,過去を振り返ることによって予想は可能である.数十年前の理想が,気づかないままインターネットで実現していることも少なくはない.誰も想像していなかったことが起き,インターネットに初めて触れた時以上の高揚感を体験できるか.この先,日本のネット文化に期待をしているのはそれだけなのである.
\\ 日本のインターネットに本格的に言論空間が立ち上がったのは,2ちゃんねるが最初である.同サイトは当初マスメディアからの批判もあったが,秀逸な書き込みだけを拾って読んでいくことに成功すれば,意外にも良くできた言論空間になっていた.一見は荒れて見える場所であっても,やはりインターネットの原理が動いていることには変わりない.ネットで良質な情報を収集するためには母集団がある程度大きくなることが重要であって,最初からノイズを排除して母集団を小さくしようとすると,結果的に情報全体の質が低下する.つまりこうしたノイズの集積こそが,秀逸な言論を生み出す源流になっていたのだ.
\\ ウェブ上で特定の対象への批判コメントが殺到する現象のことを炎上という.炎上を成立させる条件のひとつが,「多数の反感」である.
個人の失態や湿原,団体の不祥事などに憤った人々が,それぞれ批判的なコメントを書き加えていくことで,炎上と認めるような現象をつくり出す.炎上の過程では,ウェブ上に点在していた批判対象に関する情報が「名寄せ」されたり,住所や写真などの新たな情報がアップロードされたりするケースもある.
\\ しかし,現状は結局,自衛が強要されるにとどまっている.過剰な炎上事例については,法的に対処しようという議論が盛り上がる国もあるだろう.ネットが普及してから,まだ十数年.「ネット群衆の暴走と可能性」を巡る議論は,法的な領域ではようやく議論され始めたばかりである.
\\ 炎上の場とは,新しい時代の新しい枠組みの中でわれわれが新しく社会を創り出していくうえでの迷走の場であり,紛糾の場であり,暴走と逡巡と抗争の場である.そしてそうしたことを繰り返しながらわれわれは現在,そこで迷ったり立ち止まったり後戻りしたりしながらもなお,新しい時代への道のりを少しずつ歩みつつあるのではないだろうか.
\\ 日本のインターネット空間では,ネットの外の世界ではあまり意味をなさない.それらの内でも,ネット空間における人間関係の特殊な構造を象徴的に表現しているように思われるのが,21世紀のゼロ世代の半ばから使われるようになった,「リア充」,及び,その対極の「非リア充」である.ネットを頻繁に利用している人でも,相手と直接対面してのコミュニケーションが人間関係の基本であると考え,メールやLINEなどは,普段付き合っている相手と素早く連絡を取るための「手段」としてしか見ておらず,実際そういう使い方しかしていないのであれば,お互いの”リアルな生活”がかなりわかっているはずだ.
\\ ウェブ上での生活は,それが傷つきやすさとコミットメントを消去するがゆえに確かに魅力的だ.しかし,ネットの活用が物理的・社会的な世界へのわれわれの関わりを弱めることになっていることは明らかだろう.実際,ネットを使えば使うほど,われわれは,非現実的で,孤独な,意味を欠いた世界へと引き込まれることになる.そうした世界は,身体によって相続される災難から逃げたがる人々の世界にほかならないのだ.



\end{document}
