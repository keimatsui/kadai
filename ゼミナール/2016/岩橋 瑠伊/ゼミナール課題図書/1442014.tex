%卒論概要テンプレート ver. 3.0

\documentclass[uplatex,twocolumn,dvipdfmx]{jsarticle}
\usepackage[top=22mm,bottom=22mm,left=22mm,right=22mm]{geometry}
\setlength{\columnsep}{10mm}
\usepackage[T1]{fontenc}
\usepackage{txfonts}
\usepackage[expert,deluxe]{otf}
\usepackage[dvipdfmx,hiresbb]{graphicx}
\usepackage[dvipdfmx]{hyperref}
\usepackage{pxjahyper}
\usepackage{secdot}





%タイトルと学生番号,名前だけ編集すること
\title{\vspace{-5mm}\fontsize{14pt}{0pt}\selectfont ネットコミュニティの設計と力}
\author{\normalsize プロジェクトマネジメントコース 矢吹研究室 1442014 岩橋 瑠伊}
\date{}
\pagestyle{empty}
\begin{document}
\fontsize{10.5pt}{\baselineskip}\selectfont
\maketitle





%以下が本文
コミュニティには 2 つの形態がある. ブログなど のインターネット上に公開された公開型と,SNS や LINE のような一部の人に限定して公開する非公開 型のコミュニティである. 日本の初期のネットコ ミュニティの特徴は, まず匿名的であったと言える. 本名を出さずにIDやニックネームでやり取りをし, 実際に相手と会うことは想定していない, つまりリ アル社会と断絶したインターネット完結型コミュ ニティが主であった.どのようにして実名制SNSが 広まっていったかというと, 今までインターネット コミュニティをあまり利用してこなかったリアル 社会の人々がスマートフォンの普及により急速に インターネットサービスを使い始めたという点が 大きいといえる.mixiは招待制のSNSであり元々繋 がりのある人達と交流したいという要望に答え日 本のSNS界のトップに上り詰めた.元々のネットの 本質的で重要な価値に敢えて制限を加える事でオー プンの裏にある不安感を取り除き, ユーザーに価値 をもたらした. 一方,Twitter は1つの投稿に 140 字 という制限を加えた. これにより情報発信のカジュ アル化に成功した. つまり, ブログのようにタイト ルを決める必要もなく, 自由な内容で気軽に投稿で きるという点でユーザーに価値をもたらした. ネッ トコミュニケーションは,匿名性,反応性,平等性,正 確性, 感情性の 5 つの要素から成り立っている. 各 コミュニティサイトはこの 5 つの要素で分析でき, 各パラメーターごとに特徴が生まれ分類できるよ うになる. コミュニテイサイトが成長し続けるため には, 新規性, ユーザビリティ, 熱心なユーザーを如 何にして取り込むかである. コミュイティサイト運 営の上で投稿に対するレスポンスの速さ, 多さは重 要である.TwitterのRTといいねのわかり易さ,速さ がとても優れているといえる. 流行るネットコミュ ニティを作る上で最も大事なのは設計である. 投稿 するユーザーが何をモチベーションにするのか考 え, どれを一番中心に置いて, それを満たすような 設計にするのかが需要である. 書き込める内容の自 由度も決めなければならない. リスクを抑える為と
いって自由度を低くするとユーザーが投稿する気 がなくなってしまう. 過度の低品質な投稿や個人へ の誹謗中傷, 書き込み自体が犯罪なものは制限しな ければならないだろう. 最初から作りこみ過ぎると, ユーザーの創造性を奪ってしまい, 文化や風土を作 ることなく廃れてしまう. コミュニティはユーザー とともに発展していくものなのである. 立ち上げ時 は書き手のユーザーを第一に集めるべき. 読み手が 居なくても書きたくなるように作るのが重要. しか し熱量を上げ過ぎると燃え尽きてしまう. 投稿の削 除はなるべく避ける. 拡大期では書き手重視から読 み手重視に切り替えていく. 書き手はリフレッシュ していかないとならない. 古参ユーザーは影響力を 持ちやすく主になりやすい. また, 独自のコンテキ ストが生まれてしまう.2 年前はこうだったよね, の ような会話をコミュニティ上でしてしまうのだ. こ れによって新参が入りにくくなりコミュニティは 衰退してしまう. ネットコミュニティを新しく作り たいならば, 人間がコミュニケーションに何を求め ているのかという根本的な視点と, サービスを使っ ているユーザーが求めているものを提供するとい う視点, この両方が必要である.

\end{document}
