%卒論概要テンプレート ver. 3.0

\documentclass[uplatex,twocolumn,dvipdfmx]{jsarticle}
\usepackage[top=22mm,bottom=22mm,left=22mm,right=22mm]{geometry}
\setlength{\columnsep}{10mm}
\usepackage[T1]{fontenc}
\usepackage{txfonts}
\usepackage[expert,deluxe]{otf}
\usepackage[dvipdfmx,hiresbb]{graphicx}
\usepackage[dvipdfmx]{hyperref}
\usepackage{pxjahyper}
\usepackage{secdot}





%タイトルと学生番号,名前だけ編集すること
\title{\vspace{-5mm}\fontsize{14pt}{0pt}\selectfont 検索の新地平(要約)}
\author{\normalsize プロジェクトマネジメントコース 矢吹研究室 1442012 岩瀬 翔}
\date{}
\pagestyle{empty}
\begin{document}
\fontsize{10.5pt}{\baselineskip}\selectfont
\maketitle





%以下が本文
私たちが「検索する」と聞いて思い浮かべるものといえば,グーグルやヤフー!などのウェブ検索サービスの検索窓だろう.私たちユーザーは,電子情報空間から自分が欲しい情報の手がかりを探そうと,その検索窓に言葉を入力する.すると,検索エンジンは世界中の膨大なウェブページの中から,入力された言葉を含みユーザーの役に立ちそうなウェブページを瞬時に選び出して,検索結果として返してくれる.またウェブ上には,ウェブ全体を検索対象とするウェブ検索とは別に,特定分野の情報に限定している専門ポータルサイトが多数存在している.

私たちは主にキーワード検索を使用しており,キーワード検索があまりにも自然にインターネットに溶け込んでいるので,その存在や重要性を意識することは少なくなった.現代では情報がインターネット上にさえあれば,その知識を私たち自身のものとすることができる.与えられたキーワードからユーザーの意図を推定し,60兆ともいわれるウェブページの中から最適なものを探し出す.その検索過程では自然言語処理,統計処理,機械学習を組み合わせ,100を超える様々な指標を組み合わせてランキングが決定されている.

今日のインターネットでは膨大な情報が送受信されているが,その中には画像や映像が多く含まれている.写真や動画の共有サイトやクラウドサービスでは,ユーザーが送信した画像や映像が大量に蓄積され,不特定多数またはコミュニティの中で共有されたり,個人的に活用されたりしている.映像や画像はデータを減らしたり,タイトルや公開日時となる付加情報を格納したりするため,標準的なフォーマットを用いてファイルとして蓄積・通信されている.テキストに比べて画像や映像はデータはるかに大きなデータ量となるため,「圧縮」と呼ばれる手段を使って余分な情報を削ぎ落とし,削減されている.映像は画像よりもデータ量が大きいため広がるのが遅くなったが,スマートフォン・デジタルカメラの普及やユーチューブなどの動画共有サイトの登場によりインターネット上に溢れるようになった.

検索エンジンに人物名を入力することで該当する人物の画像または映像を探し出すことのできる画像検索・映像検索という機能がある.現在インターネット上の画像検索は,テキストにもとづく方法,画像にもとづく方法,及びこのふたつを複合的に利用した方法によって実現されている.画像検索・映像検索の技術はどちらもまだまだ発展途上ではあるが,その能力はすでに実用の段階に至っている.検索エンジンで利用されているだけでなく監視カメラなど社会インフラとして利用されているほか,街の写真から位置がわかるシステムが開発されるなど生活に密着したかたちで,さまざまなところに応用されている.

空間を検索するために,まず地図によって表現される.デジタルデータとしての地図を扱うシステムである地理情報システムの研究開発は1960年台から進んできた.こうしたシステムは,当初は政府機関による土地の管理など専門的な利用が主な目的であったが,インターネットの登場がデジタル地図を日常的なものとした.現代ではモバイル機器の普及によって地図の重要性はますます高まりつつあり,インターネットサービスを展開する上で不可欠の存在となっている.

ウェブ検索には一般に人間が意思決定で陥りやすい罠もある.信頼性の保証がなく,コピペで再生産された似非情報が溢れている.無意識に情報を取り込んでしまわないようにするのは至難の業である.

世界中の記憶や物語がインターネットに流れ込んでいる今,オフラインの脳を抱えて生きている私たちの意味は,自分だけの連想,発想,心の動きを育てて,自分なりの考えを紡ぎだすことだと信じたい.検索の地平は,私たちの頭の中へと広がっているのだ.\nocite{takano2015}

\bibliographystyle{junsrt}
\bibliography{biblio}

\end{document}
