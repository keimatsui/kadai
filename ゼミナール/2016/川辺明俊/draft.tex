%卒論概要テンプレート ver. 3.0

\documentclass[uplatex,twocolumn,dvipdfmx]{jsarticle}
\usepackage[top=22mm,bottom=22mm,left=22mm,right=22mm]{geometry}
\setlength{\columnsep}{10mm}
\usepackage[T1]{fontenc}
\usepackage{txfonts}
\usepackage[expert,deluxe]{otf}
\usepackage[dvipdfmx,hiresbb]{graphicx}
\usepackage[dvipdfmx]{hyperref}
\usepackage{pxjahyper}
\usepackage{secdot}





%タイトルと学生番号,名前だけ編集すること
\title{\vspace{-5mm}\fontsize{14pt}{0pt}\selectfont 開かれる国家}
\author{\normalsize プロジェクトマネジメントコース 矢吹研究室 1442045 川辺 明俊}
\date{}
\pagestyle{empty}
\begin{document}
\fontsize{10.5pt}{\baselineskip}\selectfont
\maketitle





%以下が本文
なめらかな世界への夢では二つのポイントがある.まず一つは、この複雑な世界を複雑なまま生きるには、分人民主主義にすることである.複雑な世界とは貨幣,投票,戦争などの社会制度のことであり,分人民主主義とは個人をより細かくし人間の矛盾を許容し分人によって構成される新しい民主主義である.具体的に委任票がネットワークの上を伝播していく投票システムの伝播委任投票システムや国家のような強固な存在でさえ構成的なものであるということを,明確に自覚できるようにするための構成的社会契約論がある.\\
 もう一つは,現在の代表民主制を代替しうる可能性のある創発民主制をとっていかなければならないことである.まず代表民主制とは現在多くの国で行われている選挙などのある一定の方法で代表者を決めて,間接的に投票などをした人々が政治に関わってゆく共和的形態である.だが今のグローバル化したことによって起こる複雑で巨大な問題すべてに対応するには困難を要する.この問題に立ち向かえる創発民主制というのは,一人一人の市民が状況に応じて自己組織的に政治問題を審議し取り組むことで,民主制の質を高めることである.一枚岩的なメディアではなく,ブログなどの新しいツールを使うことによって,少数のネットが多数のネットを下から突き上げ社会の中で重要な役割を果たすことである.\\
 次に溶解する境界では二つのポイントがある.まず一つは,ネット世界の観念連合をそのまま現実世界へと持ち込む者たちをカウンターという形でヘイトスピーチデモやヘイトクライムをやめさせることである.ここで言及している持ち込む者たちは近年ネット世界で人種や国籍,ジェンダーなど特定の属性を有する集団や個人をおとしめたり,差別や暴力行為を煽るいわゆるネット右翼のことである.そしてそこにカウンターを行うのが,民主主義的な目標をもつレイシストをしばき隊などの組織である.\\
 もう一つは,政治にそこそこの快適さを求めてはいけなくデータ駆動型政治という考えを使って批判している.データ駆動型政治とは,「日々,生成されるデジタルデータを高速かつ高度に分析する技術が整ったことにより,データを政治資源の増大や政治的な便益に生かそうとする動き」である.この政治は快適さと合理性を求める文化にとてもマッチしているように見えるが,この技術は大半の人を生活世界に個々の関心向けさせ,政治からは関心を減少させてしまう.\\
 リバタリアニズムと国家では二つのポイントがある.まず一つは,リバタリアンの立場からすれば,国からの徴税を逃れることは,国を発展させることにつながるということである.リバタリアンとは自由原理主義者のことである.リバタリアンからみれば国から徴税される行為は国家が行う暴力であり,個人が工夫して税金逃れは当然のことと考える.なぜ近年タックスヘイブンやスイスの銀行脱税幇助などで,多くの人々が不満あるのに国家の発展につながるのか,一部の国が積極的に相続税や贈与税を非課税にすると同時に,海外で得た所得に課税しないことで富裕層の移住を促し,その国にお金を落とす仕組みを取り入れたり,法人の徴税を少なくすることによって,雇用を増やし国を発展させることからくる考えだからである.\\
 もう一つは,環境管理型権力が社会統治の観点からさらなるデータ化と合理化を進める一方で,ハクティビズムをはじめとした情報環境を利用した政治活動が同時に生じていることである.環境管理型権力とは,主体の意思が存在することを前提に,ある環境を設計することによって個人の自己決定権を促し,強制力を働かせることのない合理的権力のことをいう.国家や企業がこの権力を行使することによって,情報環境を操り市民の主体的な意思を放棄させ行動決定に介入するとの批判がある.ハクティビズムとは,ハックとアクティビズムをかけあわせた造語である.ハッカーたちは,国家や企業に好き勝手させないように技術革新による社会改良の思想もとに反中央集権を進めている.



\bibliographystyle{jplain}
\bibliography{biblio}


\end{document}
