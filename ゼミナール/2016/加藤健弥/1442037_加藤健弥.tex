%卒論概要テンプレート ver. 3.0

\documentclass[uplatex,twocolumn,dvipdfmx]{jsarticle}
\usepackage[top=22mm,bottom=22mm,left=22mm,right=22mm]{geometry}
\setlength{\columnsep}{10mm}
\usepackage[T1]{fontenc}
\usepackage{txfonts}
\usepackage[expert,deluxe]{otf}
\usepackage[dvipdfmx,hiresbb]{graphicx}
\usepackage[dvipdfmx]{hyperref}
\usepackage{pxjahyper}
\usepackage{secdot}





%タイトルと学生番号,名前だけ編集すること
\title{\vspace{-5mm}\fontsize{14pt}{0pt}\selectfont ネットを支えるオープンソースを読んで}
\author{\normalsize プロジェクトマネジメントコース 矢吹研究室 1442037 加藤健弥}
\date{}
\pagestyle{empty}
\begin{document}
\fontsize{10.5pt}{\baselineskip}\selectfont
\maketitle
プログラムの実行とはコンピューターに搭載されているCPU の基本的動作の極めて単純な命令の実行だけにある。プログラマーが書くのはソースコードと呼ばれる一種の文書で、それは機械語に変
換されて実行される。ソースコードを機械語に変換する方法には事前に全部変換する「コンパイル方式」とプログラムを実行するたびに必要に応じて変換する「インタープリター方式」の大きく分けて2 つのやり方がある。すでに多種多様のプログラミング言語が存在しているのに毎年のように新しいプログラミング言語がいくつも生まれているのはハードウェアの処理性能向上とソフトウェアの巨大化という時代の変化に対応するためである。プログラミング言語は生産性と実行速度がトレードオフの関係であるためにその進化は、人間がより理解しやすい表現力を向上させるという歴史となった。進化する過程で採用されたオブジェクト指向という考え方は数値や文字、それらの集まりでしかなかったデータをオブジェクトという単位で扱うことでより人間が現実世界を記述するのに適したプログラミング言語を生み出した。従来プログラミング教育は、学究的な目的に加えて、社会で必要とされる人材の供給や、個人の就職、経済的安定などが重視されてきたが新しい価値を生み出せる情報産業に対する期待は近年ますます高まっている。しかし、ある目的に特化した教育は、視野の狭窄を招くこともある。日本のプログラミング教育の現在はパソコン教室で子供たちが自由な創作の空間に成り得ていない。企業の経営者はハッカーを育てることが企業価値の向上につながることを理解し、共生するインターネット時代を生んだ。つまり、ハッカー中心の企業文化になることで競争力のある企業をつくった。ハッカーたちの業績が組織を動かし、コミュニティを形成した。無料ソフトウェアで儲けるためにオープンソースという言葉を発明し、そのライセンスを定義した。ソースコードを公開すると進化していくというバザールモデルが広くしられるようになった。それによって競合する企業もオープンソース開発については協調する企業もコミュニティの価値を理解しつつある。ソフトウェアライセンスは権利者が、ある一定の制限を設けてユーザーに対してソフトウェアを使用・利用する権利を許可する仕組みである。歴史を遡るとソフトウェアは著作物であるとみなされてはいなかったが、重要なビジネス上の財産を考えられ秘匿される傾向が強まったために作成されたソフトウェア自体は著作物という扱いになった。現在、インターネット上では数えきれないほどのオープンソースソフトウェアプロジェクトが立ち上げられており、多種多様なソフトウェアが開発され、利用されている。ソフトウェアとは、コンピューター上で実行されるプログラムである。その作成過程には大きく2 つの段階ある。その方法はプログラミング言語というソースコードで処理内容を細かく記述し、それをコンパイラーというプログラムでオブジェクトコードに変換する。ソースコードは基本的には人間が読み書きできる形式となっているが、オブジェクトコードはコンピューターが実行するためのものであり、人間が読んだりすることは想定されていない。企業がオープンソースソフトウェアを使うやり方は大きく分けて3 つある。ひとつは開発コミュニティが開発したオープンソースソフトウェアを主な商品として販売、サポートする企業。もうひとつは、オープンソースソフトウェアを部品として使い、より大きなシステムを構築して納入する企業。最後は、オープンソースソフトウェアを使ってシステムを構築して、そのシステムで事業を行う企業である。オープンソースソフトウェアを利用する場合、一から開発するよりコストを安くできるし、技術者がいれば保守することも可能である。オープンソースソフトウェアは、最初のころは主に開発者や、大学の研究所などで開発されたものがほとんどのため企業が中心に開発しているオープンソースソフトウェアはあまり存在しなかった。しかし、時がたちインターネット・ウェブが広まり、企業が中心となって運営するオープンソースプロジェクトが出てくるようになった。企業の技術力をはかるときに、オープンソースソフトウェアをどれだけ公開しているか、その品質はどれくらいかをみたりするようになってきている。
\bibliographystyle{junsrt}
\bibliography{biblio}%「biblio.bib」というファイルが必要.
\end{document}
