%卒論概要テンプレート ver. 3.0

\documentclass[uplatex,twocolumn,dvipdfmx]{jsarticle}
\usepackage[top=22mm,bottom=22mm,left=22mm,right=22mm]{geometry}
\setlength{\columnsep}{10mm}
\usepackage[T1]{fontenc}
\usepackage{txfonts}
\usepackage[expert,deluxe]{otf}
\usepackage[dvipdfmx,hiresbb]{graphicx}
\usepackage[dvipdfmx]{hyperref}
\usepackage{pxjahyper}
\usepackage{secdot}





%タイトルと学生番号,名前だけ編集すること
\title{\vspace{-5mm}\fontsize{14pt}{0pt}\selectfont ネットを支えるオープンソース要約}
\author{\normalsize プロジェクトマネジメントコース 矢吹研究室 1442043 川崎貴雅}
\date{}
\pagestyle{empty}
\begin{document}
\fontsize{10.5pt}{\baselineskip}\selectfont
\maketitle





%以下が本文
今回読んだ本はネットを支えるオープンソースでした.\\
この本の要約を序章,4章と5章それに6章の4つの内容を要約していきました.\\
序章では1章から3章の内容がだいぶ含まれているため重複したるため割愛しました.\\
初めに序章では4つの内容からなっていました.\\
1つ目はアプリがどのように起動しているのか,またどのような仕組みで動作をさせているのかという内容となっていました.\\
2つ目はサーバーがどうやって端末を識別しているのかという内容となっていました.\\
3つ目はプログラミング関連の話でソースコードを機械語に翻訳したる方式,言語自体の種類について言及されていました.\\
またプログラミング言語がどのような進化をしているのかまたプログラミングには向き不向きがあることなどについての説明もありました.\\
序章最後の4つ目にはオープンソースの重要性について書かれていました.\\
なぜ重要なのかというとソースコードを読むことによるノウハウの伝達や教育に対して大きい効果を持っているからという理由が挙げられていました.\\
また4つ目の議題では現在複数の商業ソフトウェアでも少なくはない数がオープンソース化し,ソースコードを公開して,逆に外部開発者からの貢献を募る場合も多いとも記述されていました.\\
次に4章ではハッカー精神とは何かという題になっていました.\\
ここでのハッカー精神とはプログラミングを楽しんでいる.\\または純粋に手早くプログラミングができる,特定のプログラミングのエキしたパートやそれを生業にしている人のことを言いました.\\
世間でいうところのサイバー犯罪者の指したハッカーのことはクラッカーと言いました.\\
このことを踏まえてコンピュータについての話は進んでいき1971年にPCという形になったと説明がなされていました.\\
そして話はハッカー倫理について説明がされていました.\\
内容としてはハッカーの価値観などの話で,情報は全て自由に利用できなければならない,コンピュータは人生をよいほうに変えうるなどのほかにも4つほどありました.\\
3つ目の5章ではソフトウェアライセンスがユーザーに対して何らかの制限を設けてソフトウェアの使用・利用し許可を与える仕組みがソフトウェアライセンスでした.\\
またこのようなライセンスが必要な理由はソフトウェア自体に著作物という扱いになっているからでした.\\
またこの例として上がっているのがWindows PCやApple Store等が本作品でも上がっていました.\\
OSSライセンスはOSDというオープンソースの基準に合致している物の事をいいました.\\
OSDはOISが定めた10項目を満たしたソフトウェアライセンスで配布されているソフトをOSSとして扱うため,OSSの定義としてはOSSライセンスがOISの10項目を満たしていることが重要となるようでした.\\
またOSSライセンスを採用したることの狙いは主にたくさんの人に使われたい,製品の質向上機能拡張を低コしたトで行いたいなど観られました.\\
第6章ではオープンソース化が生んだ変化という題だがここではブラウザー戦争を例に挙げて説明していました.\\
具体的な例を挙げればブラウザー戦争での開発競争の影響でプログラミング内にゴミだらけな状況を改善したるためにオープンソース化し外部の力を借りながら整理出来るのではという思惑があったのではと推測されていました.\\


\bibliographystyle{jplain}
\bibliography{biblio}

\end{document}
