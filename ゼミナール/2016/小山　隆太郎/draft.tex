%卒論概要テンプレート ver. 3.0

\documentclass[uplatex,twocolumn,dvipdfmx]{jsarticle}
\usepackage[top=22mm,bottom=22mm,left=22mm,right=22mm]{geometry}
\setlength{\columnsep}{10mm}
\usepackage[T1]{fontenc}
\usepackage{txfonts}
\usepackage[expert,deluxe]{otf}
\usepackage[dvipdfmx,hiresbb]{graphicx}
\usepackage[dvipdfmx]{hyperref}
\usepackage{pxjahyper}
\usepackage{secdot}





%タイトルと学生番号,名前だけ編集すること
\title{\vspace{-5mm}\fontsize{14pt}{0pt}\selectfont ネットコミュニティの設計と力}
\author{\normalsize プロジェクトマネジメントコース 矢吹研究室 1442031 小山 隆太郎}
\date{}
\pagestyle{empty}
\begin{document}
\fontsize{10.5pt}{\baselineskip}\selectfont
\maketitle





%以下が本文

今後の日本人の暮らしの分類を「都市型コミュニティー」と「農村型コミュニティー」に分けるとするならば,今後は農村型になっていくのではないかと著者は述べた.これは著者の分析によるもので,大学生を対象にして「仕事をするならば都心部に行くか地元がいいか」とアンケートを行ったところ,約 5 割を占める学生が地元に就職したいという結果になった.10 年前の同じアンケート結果より10 ポイント上がっているという結果と,著者が実際に関わった学生の証言からそう述べられた.\\
 戦後の日本の社会において,農村から都市へと移った人々は高度経済成長期を経て都市型コミュニティー社会を成形していった.そこでの人々は,家庭や会社以外のつながりが薄い極めてものとなっていった.人々は技術発展していくことに没頭していくあまりに,人と人との関係においては孤立化するようになった.\\
 しかし今日の日本は,著者の経験だけではなくも農村型に変わりつつある.「2ちゃんねる」の登場と「Twitter」の有り方について著者は,個人の孤立化はいずれ廃れていくと述べている.一概にネットの世界は個人の主張性が強いものといわれている.これはリアルでは話すことはできないが,匿名を利用することでネット上の世界で話すことができるという棘の強い主張ができる自由度の高さからそういえる.2ちゃんねるでは誰かがスレを立てればあっという間に返事が返ってきたり炎上を起こしたりなど,人がすぐ集まり反応する.Twitter においてもユーザーはフォロワーを増やそうとする動きがあり,ネットコミュニ ティーにおいても人間関係を求める傾向があるといっている.フォロワーが多いユーザーほど,その人には信頼性と安心感があるように捉えられるのもTwitterならではであると著者は述べている.
この動きはサル学に通ずるものがある.サルは瞬く間に群れを成して行動を共にする.群れの規模が大きいほどコミュニティーに力があり,力が強ければ自然とそこに加わろうとするサルもいる. 群れを成さずに孤立したサルは餌も与えられず寒さも一人では耐えしのぐことが出来ず近いうちに死に行く.\\
 今日のネットコミュニティーでも集団を成して動いているといえる.現在の日本の状況は,「臨機応変」といった場に自分を合わせる動きが顕著である.集団の内部では周りに気を使ったり同調したりなどする一方で,一歩その集団を離れると誰にも気を使わなくなるといった言動の落差が大きな社会になっている.このことが人々の不安やストレスを高め,年々増加しつつある自殺率といったことも含め,生きづらさや孤独感が背景になっている.\\        
 社会が都市型を成形していく一方で人間関係ではギャップが生じ様々な矛盾を生み出す背景になりつつある.2 ちゃんねるやTwitter が人間関係を求めていく今日の動きは,自由度の高いネットの世界において,人間が表には出さない本来の姿を出しているのではないかと著者は述べている.そして著者が行ったアンケートの結果から 人々が手を取り合う農村型に考えが変わりつつあるのも,こうした現代の背景があってからこそではないのかと述べている.\\
 またこれらSNS サービスにおいてもここ数年の間で急激に変化してきた.2010 年前後では GREE やミクシィが台頭していたが,5 年経っただけでTwitterや LINE が台頭するようになってきた.\\
 何故このようなことが起こったのか著者はサービスの展開は恋愛に似通っている部分があると述べている.SNSも同じようにサプライズがなければサービスは平凡化してしまう.この面でGREEやミクシィはTwitterを利用した「ついぷら」サービスや,LINE のスタンプ機能などといった新規性のあるサービス事業の前で廃れていってしまった.\\
 今日のTwitter や LINE の機能も日に日に充実したものになってきている.また,アンケート機能や他のアプリケーションと共同したサービス展開を行っており,人々はそれに満足するかのように利用している.\\
 第5章「ネットコミュニティーの設計と力」では,人々はネット上でも人間関係を求めていることが述べられている.



\end{document}
