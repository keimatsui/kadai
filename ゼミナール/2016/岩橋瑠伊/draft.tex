%卒論概要テンプレート ver. 3.0

\documentclass[uplatex,twocolumn,dvipdfmx]{jsarticle}
\usepackage[top=22mm,bottom=22mm,left=22mm,right=22mm]{geometry}
\setlength{\columnsep}{10mm}
\usepackage[T1]{fontenc}
\usepackage{txfonts}
\usepackage[expert,deluxe]{otf}
\usepackage[dvipdfmx,hiresbb]{graphicx}
\usepackage[dvipdfmx]{hyperref}
\usepackage{pxjahyper}
\usepackage{secdot}





%タイトルと学生番号,名前だけ編集すること
\title{\vspace{-5mm}\fontsize{14pt}{0pt}\selectfont ネットコミュニティの設計と力}
\author{\normalsize プロジェクトマネジメントコース 矢吹研究室 1442014 岩橋 瑠伊}
\date{}
\pagestyle{empty}
\begin{document}
\fontsize{10.5pt}{\baselineskip}\selectfont
\maketitle





%以下が本文
 コミュニティには2つの形態がある.ブログなどのインターネット上に公開された公開型と,SNSやLINEのような一部の人に限定して公開する非公開型のコミュニティである.日本の初期のネットコミュニティの特徴は,まず匿名的であったと言える.本名を出さずにIDやニックネームでやり取りをし,実際に相手と会うことは想定していない,つまりリアル社会と断絶したインターネット完結型コミュニティが主であった.

 では,どのようにして実名制SNSが広まっていったかというと,今までインターネットコミュニティをあまり利用してこなかったリアル社会の人々がスマートフォンの普及により急速にインターネットサービスを使い始めたという点が大きいといえる.例としてmixiは招待制のSNSであり元々繋がりのある人達と交流したいという要望に答え日本のSNS界のトップに上り詰めた.元々のネットの本質的で重要な価値に敢えて制限を加える事でオープンの裏にある不安感を取り除き,ユーザーに価値をもたらした.一方,Twitterは1つの投稿に140字という制限を加えた.これにより情報発信のカジュアル化に成功した.つまり,ブログのようにタイトルを決める必要もなく,自由な内容で気軽に投稿できるという点でユーザーに価値をもたらした.

 流行るネットコミュニティを作るためにはどうすればいいのか,そのために最も大事な事は設計である.投稿するユーザーが何をモチベーションにするのか考え,どれを一番中心に置いて,それを満たすような設計にするのかが重要である.また,書き込める内容の自由度も決めなければならない.リスクを抑える為といって自由度を低くするとユーザーが投稿する気がなくなってしまう.しかし,過度の低品質な投稿や個人への誹謗中傷,書き込み自体が犯罪なものは制限しなければならないだろう.しかし,SNSを最初から作りこみ過ぎると,ユーザーの創造性を奪ってしまい,文化や風土を作ることなく廃れてしまう.つまり,ネットコミュニティはユーザーとともに発展していくものなのである.
新たなSNSを立ち上げる時は書き手のユーザーを第一に集めるべきである.読み手が居なくても書きたくなるように作るのが重要である.しかし書き手の熱量を上げ過ぎると燃え尽きてしまい長続きしない.投稿の削除は余程の内容でない限りなるべく避けるべきである.拡大期では書き手重視から読み手重視に切り替えていく.書き手はリフレッシュしていかないとならない.何故なら,古参ユーザーは影響力を持ちやすく主になりやすい.そこから,独自のコンテキストが生まれてしまい,「2年前はこうだったよね」のような会話をコミュニティ上でしてしまうのだ.これによって新参が入りにくくなりコミュニティは衰退してしまう.ネットコミュニティを新しく作りたいならば,人間がコミュニケーションに何を求めているのかという根本的な視点と,サービスを使っているユーザーが求めているものを提供するという視点,この両方が必要である.

 運営者の思いが成功につながるという事例もある.初期にサービス開発に取り組んでいたころの私は,そもそもサービスをヒットさせようという意識よりも,こんな仕組みが欲しいと考えていた.最初に開発した人力検索はてなのサービスは,元々,検索に困っていた父を助けたいと思い作り上げたものだ.こうして生まれたはてなグループがヒットして成長すると,私ははてなの企業としての成長を目指し始めた.社員を採用し,売り上げの拡大を目指し出した頃から,ヒットさせたいと考える比率が増えていった.しかし,なかなか大きなヒットにはつながらなかった.こうして自分自身の経験を振り返ってみると,人のために何かをしてあげたいという純粋な動機で開発したものが成長し,ヒットさせたい,売り上げを伸ばしたいといった,コミュニティの本質からそれた動機で開発したものは大方盛り上がらなかったといえる.運営者のそうした意図は,なんらかの方法でユーザーさんにも伝わっていいるのかもしれない.\nocite{kondou}

\bibliographystyle{junsrt}
\bibliography{biblio}%「biblio.bib」というファイルが必要.

\end{document}
