%卒論概要テンプレート ver. 3.0

\documentclass[uplatex,twocolumn,dvipdfmx]{jsarticle}
\usepackage[top=22mm,bottom=22mm,left=22mm,right=22mm]{geometry}
\setlength{\columnsep}{10mm}
\usepackage[T1]{fontenc}
\usepackage{txfonts}
\usepackage[expert,deluxe]{otf}
\usepackage[dvipdfmx,hiresbb]{graphicx}
\usepackage[dvipdfmx]{hyperref}
\usepackage{pxjahyper}
\usepackage{secdot}





%タイトルと学生番号,名前だけ編集すること
\title{\vspace{-5mm}\fontsize{14pt}{0pt}\selectfont ネットが生んだ文化-要約-}
\author{\normalsize プロジェクトマネジメントコース 矢吹研究室 1442069 須山武弘}
\date{}
\pagestyle{empty}
\begin{document}
\fontsize{10.5pt}{\baselineskip}\selectfont
\maketitle


%以下が本文
人間とは,現実に住むものである.そして,インターネット(以下ネット)は単なるツールにすぎない.これが普通の人の感覚である.しかし,世の中にはネットに住むと形容した方が適切であり当人たちも「ネット住民」と自称している人がいる.

ネット文化の歴史の1つとして炎上というものがある.炎上とはネット上での行為や発言が不特定多数から誹謗や中傷を受ける現象のことである.そのメカニズムの1つとして,ネット住民の反撃ということがある.これは,自分たちのルールを守って生活をしていたあとからやってきた新住民はそれを無視して現実社会の文化を持ち込もうとする.それに対して仕掛ける攻撃が炎上なのである.いわば,文化衝突なのである.

「ネット住民とあとからやってきた新住民との文化衝突」の他にも炎上メカニズムとして,「多くの反感を買うこと」がある.個人の失態や失言,ユーザーに歓迎されないアップデートによってAppStore のレビュー欄が炎上することも度々起こる.炎上の影響はネット上だけに留まらず,本人の生活圏,社会的事件までに発展することもある.しかし,炎上してしまったらコントロールが難しい.現実社会の場合,個人は暴力を放棄し,司法や警察にそれを委ねる.しかし,ネット上では,代表が存在せず,統率なき直接攻撃が続けられる.そのためほとんどの炎上では和解が困難になり,これまでのような代表制が成立しないため私たちはそれをコントロールする有効な手段を見いだせていない.

炎上と祭りは表裏一体となすものとして扱われている.炎上も祭りもネットだけの場からリアルな場へと広がりつつある.炎上は誹謗中傷を伴うも攻撃的なもので,祭りは賑やかで盛り上がりを伴う祝賀的なものである.ここで後者を祭りとすると,前者は血祭りと呼ぶことができる.つまり,広い意味での炎上は,祭り型の炎上と血祭り型の炎上が複合した概念として構成されていると言える.

掲示板は,個人サイトに付随する管理人と閲覧者の交流ではなく,閲覧者だけで成り立たせるコミュニティサイトだ.その系譜で誕生したのが「2ちゃんねる」である.今では有名な掲示板だが,初めの2年は匿名ゆえに誹謗中傷を目的として話題になる事が多く,個人サイトからは非常に嫌われていた.しかし,これは諸刃の剣で企業の内部告発などの情報も盛んに書き込まれていた.その後事業展開をスタートさせ,掲示板上で展開されたラブストーリー「電車男」の社会的大ヒットは2ちゃんねるを「匿名ゆえに誰もが気兼ねなく文章をかける場」へと変容させた.

これまでの冷笑的なネットの雰囲気を一変させたのはブログの登場である.専門知識がなくても使える日記と言った感じで始める個人が急増した.ブログが,2ちゃんねると肩を並べるメディアとなり,マスコミの欠損部分を補完し,報道に対し,弁護士などの専門家が記事の誤りを指摘するようなことも起きるようになった.しかし,ブログをジャーナリズムとして捉えようとすると,問題もある.日本のブログの大半は情報ソースがはっきりしないものだ.それを信じて紹介し,蔓延したケースも多く,デマを大量に生み出す結果にもなった.

リア充・非リアの定義として,特に恋愛や結婚を中心とする人間関係の充実が念頭に置かれ,日常生活で充実している人間をリア充と言い,そうでない人間を非リアという.しかし,これは個々の主観によるもので,非リアを自称する者の大半は自分より充実している(ように見える)人間をリア充と言い,それ以外の人と自分を非リアと定義している.ネット上でリア充,非リア充の軋轢が絶えず,自分のほうが哀れまれる存在だという意味の争いが起きている.

サブカルチャーと非リアの間には密接な関わりがある.友達がいない非リアは,サブカルチャーに辿り着きがちなのである.自分で自分のことがなんだかわからなくなった人間が様々な手段を用いて個性を探しまわった挙句,個性的な趣味でお茶を濁そうとする流れはある種王道なのである.\nocite{bib001}

\bibliographystyle{jplain}
\bibliography{biblio}

\end{document}
