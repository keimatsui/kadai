%卒論概要テンプレート ver. 3.0

\documentclass[uplatex,twocolumn,dvipdfmx]{jsarticle}
\usepackage[top=22mm,bottom=22mm,left=22mm,right=22mm]{geometry}
\setlength{\columnsep}{10mm}
\usepackage[T1]{fontenc}
\usepackage{txfonts}
\usepackage[expert,deluxe]{otf}
\usepackage[dvipdfmx,hiresbb]{graphicx}
\usepackage[dvipdfmx]{hyperref}
\usepackage{pxjahyper}
\usepackage{secdot}





%タイトルと学生番号,名前だけ編集すること
\title{\vspace{-5mm}\fontsize{14pt}{0pt}\selectfont 株価の値動きにおける企業業績との相関}
\author{\normalsize プロジェクトマネジメントコース 矢吹研究室 1442069 須山武弘}
\date{}
\pagestyle{empty}
\begin{document}
\fontsize{10.5pt}{\baselineskip}\selectfont
\maketitle





%以下が本文
\section{序論}
近年ではインターネット上で株式取引ができるようになり,手数料が安く,個人投資家も増えてきた.更に通信品質も向上したため,高速トレーディングが可能となっている.\cite{bib01}このようなことから,株式投資はより身近な資産運用の選択肢になっているといえる.

株式というものは基本的に1株からから買えるものではなく,100株単位,1000株単位から買えるものである.その上で一株当たりの株価が存在し,買う人が多ければ値段が上がり,売る人が多ければ値段が下がる.つまりは需要と供給の原則と同様である.「単純に業績が良くなれば株を買いたい人が増え株価が上がり,株価が低い時から持っている人はキャピタルゲインを得ることができる.」株取引をしない人はそう考える人が大半であろう.実際,あながち間違えではないのだが,株価の変動は必ずしも全て業績と連動することはなく,円高円安,海外市場の株高株安,企業の功績,罪過など様々な要因が影響するので決算が良ければ.実際に1986年に起きたチャレンジャー号の事件では,直後に関連している4企業の株価が急落している.\cite{bib02}

このように株価の値動きには他の様々な要因があるが,その中でも企業の決算発表における株価の値動きについての情報を収集し,株価にどのような影響があるのかを調べる.

\section{目的}
どのような場合に株価は上昇するのか,また下降するのかの条件をまとめるため,以下の3点を調査,分析することを目的とする.

・企業の決算発表が株価に与える影響度

・影響の受ける期間

・下降または上昇した際に,その後また上昇へ転じる事はあるのか

\section{プロジェクトマネジメントとの関連}
本研究テーマは,リスクマネジメントに関連性があると考えられる.本研究では,発表された決算情報が,株価の上下にどのような影響を与えるかを調査し,研究結果を出すため,決算情報が発表されてリスクを管理するにはどうしたらよいかの判断基準になるからである.

\section{方法}
複数の銘柄を選び,株価のチャートの情報と,企業の決算情報,決算情報の発表された日を収集し,それらの情報から相関などの関連性を調査する.決算情報は,決算短信が一番初めに発表される決算短信を資料とする.\cite{bib03}

\bibliographystyle{junsrt}
\bibliography{biblio}%「biblio.bib」というファイルが必要.

\end{document}
