%卒論概要テンプレート ver. 3.0

\documentclass[uplatex,twocolumn,dvipdfmx]{jsarticle}
\usepackage[top=22mm,bottom=22mm,left=22mm,right=22mm]{geometry}
\setlength{\columnsep}{10mm}
\usepackage[T1]{fontenc}
\usepackage{txfonts}
\usepackage[expert,deluxe]{otf}
\usepackage[dvipdfmx,hiresbb]{graphicx}
\usepackage[dvipdfmx]{hyperref}
\usepackage{pxjahyper}
\usepackage{secdot}





%タイトルと学生番号,名前だけ編集すること
\title{\vspace{-5mm}\fontsize{14pt}{0pt}\selectfont 要約:デジタル時代の知識創造}
\author{\normalsize プロジェクトマネジメントコース 矢吹研究室 1442085 中村真悟}
\date{}
\pagestyle{empty}
\begin{document}
\fontsize{10.5pt}{\baselineskip}\selectfont
\maketitle





%以下が本文
 インターネットがパソコンやスマートフォン、タブレット端末などで身近になった。多くの情報が多くの人にコピーされ、変わっていき、元の情報の形などが無くなっていく、そのような事態をなくすために著作権法は存在する。現在、著作権の問題としてあるのはインターネットができる以前の著作権法では著作者にも利用する側にも枷となり、インターネットを介した著作物の利用が複雑化している。だが、著作権法を変えるにはベルヌ条約改定という国際的な議題となるため根本的な解決は容易ではないのだ。
\\ そもそも著作権とは、無体である小説や芸術作品等を保護するための法律である。書籍で例えれば、小説自体に著作権があり、本自体には著作権があるわけではない。著作権によって、所有者の許可なく著作物を模倣し、公開することが禁止されている。言わば、形のない情報を個人のものとする財産権に近しいものだ。また、著作権には著作物の私的利用を認める一文がある。
\\ パソコンが普及し始めた1980年代、デジタル的な著作物としてのソフトウェアが生まれた。そしてオープンソースソフトウェアという構造を決め、今までの著作権法にはない形となっている。
\\ 多くの人が情報を得て、情報を発信し、著作者といえるようになった。著作物は過剰となり、過剰となった著作物はやがて管理できなくなるか価値が暴落し、市場は失敗へと至る。どうなるかはわからないが、万人が著作者の時代が来るということだけが言える。
\\ タブレット端末やスマートフォンの進歩でより紙の本に近い感覚の電子書籍が流行り始めている。紙の本とは違い、瞬時に単語の意味を調べることができたり、楽譜を表示しながら実際の音楽を聴くことができたり、紙媒体ではありえないほど拡張性がある。だが、メリットばかりではない。媒体であるタブレット端末やスマートフォンは、充電しなければならないこと、生産が中止される恐れがあること、閲覧するソフトがまちまちだということ、の三点がデメリットである。特に充電に関しては必ずと言っていいほど付きまとうことになる。
\\ 技術の発展に伴い、インターネットを使った情報収集が容易となった昨今では、著作権と現実があわくなってきた。簡単にコピーすることが出来るようになり、それが悪いことだと知らずに著作権を侵害しているといえる。一つ例として、グーグルのブック検索プロジェクトがある。グーグルが電子書籍を実際にシステム化しようとした際にも、書籍の内容だけではなく引用時に出店が必要となるため、著者のデータベースをも作らなくてはならなかった。さらには権利者不明の書籍に関しては事実上グーグルの独占となるため、多くの反対を受けた。このことから、著作権から考え、実際の権利はどこにあるのかという大きな見方が出来なくなってきているのではないだろうか。そのことから「著作権と現実があわない」状態になったのだろう。
昔は、日の目を浴びることのなかった自費出版本もインターネットの普及によって様々な経路を得た。さらに電子書籍やブログの登場により、小説家と本の在り方は多種多様になった。これからもインターネットが生み出す力を失わないのであれば、ほんと小説家の在り方は変わり続けるだろう。
\\ デジタル化の波は何も紙と電子書籍に限った話ではない。図書館や博物館、美術館もデジタルデータによってより長く、正確に残し、多くの人に使ってもらえるようにするデジタルアーカイブというものがある。従来の方法とは違い、一つのデバイスから複数のデータを閲覧し、結びつけることが出来る。ただ、デジタルアーカイブも絶対的というわけではなく、再生する機器と保存する機器がなくなれば価値がなくなってしまう。それは扱う人間にも同様のことが言える。デジタルアーカイブを残すことと、それを扱える人材を育成することが急務といえる。
\\ これらのことは、著作権が出来たころからの問題ではなく、インターネットが出来てからの問題である。最もインターネットに触れているのは、今を生きる私たちだ。いまだに未知数で多くの可能性を秘めているインターネットを使いこなし、知識を創造することが出来るかはこれらにかかっている。

\end{document}
