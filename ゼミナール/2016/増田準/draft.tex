%卒論概要テンプレート ver. 3.0

\documentclass[uplatex,twocolumn,dvipdfmx]{jsarticle}
\usepackage[top=22mm,bottom=22mm,left=22mm,right=22mm]{geometry}
\setlength{\columnsep}{10mm}
\usepackage[T1]{fontenc}
\usepackage{txfonts}
\usepackage[expert,deluxe]{otf}
\usepackage[dvipdfmx,hiresbb]{graphicx}
\usepackage[dvipdfmx]{hyperref}
\usepackage{pxjahyper}
\usepackage{secdot}





%タイトルと学生番号,名前だけ編集すること
\title{\vspace{-5mm}\fontsize{14pt}{0pt}\selectfont 課題研究企画}
\author{\normalsize プロジェクトマネジメントコース 矢吹研究室 1442104 増田準}
\date{}
\pagestyle{empty}
\begin{document}
\fontsize{10.5pt}{\baselineskip}\selectfont
\maketitle





%以下が本文

\section{背景}
現代において,Webサイトは見るものから使うものへと形を変えている.「言語や配信の仕組みに変わりはありませんが,広い意味での「Webデザイン」という行為は,その時々に合わせて変化し続けています\cite{bib002}.」とあるように,時代にあったWebデザインが求められている.また,「ネット界は多並行分散型のネットワークになっているので,より多様化を進める方向でウェブという市場は推移する\cite{bib001}.」とあるように,流行の変化に適応することがウェブ運営にとっても重要だと考える.視覚的な良し悪しだけではなく,使いやすさを追求することもデザインの一環であるといえる.例えば,スマートフォンなどタブレット端末が生活に根付いた昨今では,ユーザーは縦スクロールの機会が増え,それにあったWebデザインの重要性も高まっている.では,現代において流行しているWebデザインとはどのようなものなのか.

\section{目的}
この研究ではWebデザインに流行のパターンがあるのかを明らかにしたい.海外デザインブログDesignmodoで2016年1月4日に公開された「11 Web Design Trends for 2016\cite{bib004}」という記事がある.2016年のWebデザインのトレンドとなるパターンを11個紹介したものだ.例を挙げると,情報整理がしやすくデバイスを問わず動作が可能な「カード型のデザイン」.ユーザーが直感的に移動させることができ,スクロール,クリック,時間経過にも対応した「フルスクリーンスライド」.更には,ヘッダーに映画のような高解像度の動画を用いた「ヒーロービデオヘッダー」では,「Webデザインは映画製作のようになるだろう」とも言われている.Designmodoでは例年,Webデザインのトレンドが紹介され,注目度が高まっている.この研究では2016年現在,世界でアクセス数の多い人気サイトにおいて,記事に紹介されたパターンが,トレンドパターンを用いていないWebサイトよりも多くあるかを検証する.

\section{手法}
この研究はディープラーニングを用いて行われる.Caffeという画像解析用ライブラリとPythonを利用する.スクリーンショットでWebサイトのデザインを保存し,2014年から2016年までのトレンドパターンに分類し,パターンを記憶させる.その際,各年には複数のトレンドパターンが存在するため,ジャンルが近いものを対象とする.例えば,ジャンルを「配色」とするならば,2016年は派手でカラフル(80年代を連想させる)な配色,2015年は単色でアクセントが強調される配色,そして2014年はシンプルで清潔な配色と紹介されており,対象となりえる.研究の方法として,予め各年度のパターンが使用されているWebサイトのデザインを最低10個記憶させる.もちろん,記憶させる画像の数が多いほど画像解析は正確になりうるので可能な限り集める.その後,様々なWebサイトがトレンドパターンに当てはまっているかをディープラーニングで解析する.「データマイニングを利用してヒットの要因を把握する技術は,プロジェクトの新規性を見出す方法のひとつとなる\cite{bib003}.」とあるように,この研究にはPMとの関係性もあるといえる.


\bibliographystyle{junsrt}
\bibliography{biblio}%「biblio.bib」というファイルが必要.

\end{document}
