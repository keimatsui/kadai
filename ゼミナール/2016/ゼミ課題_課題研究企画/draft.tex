%卒論概要テンプレート ver. 3.0

\documentclass[uplatex,twocolumn,dvipdfmx]{jsarticle}
\usepackage[top=22mm,bottom=22mm,left=22mm,right=22mm]{geometry}
\setlength{\columnsep}{10mm}
\usepackage[T1]{fontenc}
\usepackage{txfonts}
\usepackage[expert,deluxe]{otf}
\usepackage[dvipdfmx,hiresbb]{graphicx}
\usepackage[dvipdfmx]{hyperref}
\usepackage{pxjahyper}
\usepackage{secdot}





%タイトルと学生番号,名前だけ編集すること
\title{\vspace{-5mm}\fontsize{14pt}{0pt}\selectfont 課題研究企画}
\author{\normalsize プロジェクトマネジメントコース 矢吹研究室 1442104 増田準}
\date{}
\pagestyle{empty}
\begin{document}
\fontsize{10.5pt}{\baselineskip}\selectfont
\maketitle





%以下が本文
現代において,Webサイトは見るものから使うものへと形を変えている.「言語や配信の仕組みに変わりはありませんが,広い意味での「Webデザイン」という行為は,その時々に合わせて変化し続けています.\cite{bib001}」とあるように,時代にあったWebデザインが求められている.また,「ネット界は多並行分散型のネットワークになっているので,より多様化を進める方向でウェブという市場は推移する.\cite{bib002}」とあるように,流行の変化に適応することがウェブ運営にとっても重要だと考える.視覚的な良し悪しだけではなく,使いやすさを追求することもデザインの一環であるといえる.例えば,スマートフォンなどタブレット端末が生活に根付いた昨今では,ユーザーは縦スクロールの機会が増え,それにあったWebデザインの重要性も高まっている.では,現代におけるユーザー好むデザインとはどのようなものなのか.

この研究では,Webデザインの違いによってユーザーからの支持がどう変わるのか,その違いを明らかにしたい.海外デザインブログDesignmodoで2016年1月4日に公開された「11 Web Design Trends for 2016\cite{bib004}」という記事がある.2016年のWebデザインのトレンドとなるパターンを11個紹介したものだ.例を挙げると,情報整理がしやすくデバイスを問わず動作が可能な「カード型のデザイン」.ユーザーが直感的に移動させることができ,スクロール,クリック,時間経過にも対応した「フルスクリーンスライド」.更には,ヘッダーに映画のような高解像度の動画を用いた「ヒーロービデオヘッダー」では,「Webデザインは映画製作のようになるだろう」とも言われている.Designmodoでは例年,Webデザインのトレンドが紹介され,注目度が高まっている.この研究では2016年現在,この記事に紹介されたパターンが,過去のデザインのトレンドパターンを用いたWebサイトよりもユーザーに好まれるかを検証する.

この研究はアクセス数の統計とその比較によって行われる.まず,同じ内容のサイトを複数立ち上げる.2016年のトレンドパターンを用いたもの,そして比較対象として2015年と2014年のトレンドパターンを用いたものだ.各年には複数のトレンドパターンが存在するため,ジャンルが近いものを比較対象とする.例えば,ジャンルを「配色」とするならば,2016年は派手でカラフル(80年代を連想させる)な配色,2015年は単色でアクセントが強調される配色,そして2014年はシンプルで清潔な配色と紹介されており,比較対象となりえる.それぞれを同日に公開し,期間を決めてアクセス数を解析し,グラフなどにまとめてどのデザインがユーザーに好まれるかを検証する.また,ジャンルを変えたパターンでも同様の検証をし,データの確実性を高める.「データマイニングを利用してヒットの要因を把握する技術は,プロジェクトの新規性を見出す方法のひとつとなる.\cite{bib003}」とあるように,この研究にはPMとの関係性もある.

この研究の問題点は,Webサイト制作の技術面の不安である.Desighmodoが紹介したデザインパターンを再現できるかという点だ.もう一つは,紹介されたデザインの再現率にばらつきがある場合である.ばらつきがあると,デザイン性の違いだけでの比較といえず,データの信頼性に問題があるといえる.よってこの研究をするうえで最も必要なことは,プログラミングの技術を高めることのほか,画像編集や映像編集などの技術の取得であると考える.

\bibliographystyle{junsrt}
\bibliography{biblio}

\end{document}
